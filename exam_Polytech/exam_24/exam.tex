\documentclass[12pt]{article}
\usepackage{tikz}
\usepackage{amsfonts} %% mathbb
\usepackage{amsmath} %% align
\usepackage{listings} %% pour le code
\usepackage{stmaryrd} %% pour llbracket et rrbracket

\begin{document}
\noindent
{\rule{\textwidth}{.2mm}}\\
\renewcommand{\labelenumi}{(\alph{enumi})}
\noindent
Polytech Sorbonne \hfill Ann{\'e}e universitaire 2023--2024\\
Analyse Numérique des EDP \hfill
MAIN4 \hfill mercredi 15 mai 2024
{\rule{\textwidth}{.2mm}}\\


\newtheorem{exercice}{\bf Exercice}
\newenvironment{exo}{\begin{exercice}
\rm}{\end{exercice} }
\newtheorem{corrige}{\bf Corrig{\'e}}
\newenvironment{cor}{\begin{corrige}\rm}{\end{corrige} }

\newcommand{\R}{\mathbb{R}}
\newcommand{\Z}{\mathbb{Z}}
\newcommand{\GRAD}{\nabla}
\newcommand{\SCAL}{\cdot}
\newcommand{\polP}{\mathbb{P}}
\newcommand{\calT}{\mathcal{T}}
\newcommand{\calA}{\mathcal{A}}

\newcommand{\tu}{\widetilde{u}}

\setcounter{MaxMatrixCols}{15} % for matrices of size more than 10

\begin{center}
{\bf Examen}
\end{center}

\begin{exo}
  \textit{Discr\'etisation par \'el\'ements finis (environ 10 pts)}

  On note $\Omega = ]0,1[$. On s'int\'eresse au probl\`eme
  \begin{align}
    \label{eq:pb:1}
    - u'' &= f \text{ dans } \Omega
    \\
    u'(0) &= u(0) ,
    \\
    \label{eq:pb:2}
     u'(1) &= - u(1) ,
  \end{align}
  o\`u $f : \Omega \to \R$ une fonction donn\'ee.
  
  \begin{enumerate}
  \item Prouver que toute solution $u \in C^2(\overline{\Omega})$ de~\eqref{eq:pb:1}--\eqref{eq:pb:2}
    est aussi solution de la formulation variationnelle suivante:
    \begin{align}
      \label{eq:pb_var}
      \text{Trouver } u \in H^1(\Omega) \text{ telle que }
      a(u,v)
      = \int_{0}^1 f v \qquad
      \forall v \in H^{1}(\Omega) ,
    \end{align}
    o\`u $a(u,v) := \int_{0}^1 u'(x) v'(x) dx + u(0) v(0) + u(1) v(1)$.
  \item On se donne l'in\'egalit\'e de Poincar\'e suivante: il existe $C > 0$ telle que
    \[
      \int_{0}^1 (v(x))^2 dx \leq C \Big( v(0)^2 + v(1)^2 + \int_0^1 (v'(x))^2 dx \Big) .
    \]
    On se donne \'egalement l'in\'egalit\'e de trace : il existe une constante $C>0$ telle que
    $\forall v \in H^1(0,1), |v(0)| + |v(1)| \leq C \| v \|_{H^1}$.
    En d\'eduire que si $f \in L^2(\Omega)$, alors
    le probl\`eme~\eqref{eq:pb_var} admet une unique solution.

  \item Proposer une discr\'etisation \'el\'ements finis du probl\`eme consid\'er\'e.
    On rappelera la d\'efinition des points de discr\'etisation et de l'espace de fonctions utilis\'e.

  \item Montrer que le probl\`eme \'el\'ements finis admet une unique solution 
    (on suppose \`a nouveau $f \in L^2(\Omega)$).

  \item Donner l'\'ecriture alg\'ebrique du probl\`eme \'el\'ements finis
    (le syst\`eme lin\'eaire \`a r\'esoudre pour trouver la solution).
    On ne vous demande \textbf{pas} de calculer les coefficients de la matrice.
  \end{enumerate}

\end{exo}

\newpage
\begin{exo}
  \textit{Discr\'etisation par diff\'erences finies (environ 10 pts)}

  On se place dans le domaine temporel $]0,T[$ ($T > 0$) et spatial $]0,1[$.
  On veut discr\'etiser l'\'equation de la chaleur avec conditions aux limites p\'eriodiques
  \begin{align*}
    \frac{\partial u}{\partial t} - \frac{\partial^2 u}{\partial x^2}
    &= 0 \quad \text{ sur } ]0,T[ \times ]0,1[ ,
    \\
    u(t,0)
    &= u(t,1) \qquad \forall t \in [0,T] ,
    \\
    u(0,x)
    &= u_0(x) \qquad \forall x \in [0,1] .
  \end{align*}
  Pour cela, on discr\'etise l'espace par $M+1$ points $x_j = jh_x$ ($h_x = 1/M$)
  et le temps par $N+1$ points $t_n = n h_t$ ($h_t = T/N$).
  On consid\`ere le sch\'ema donn\'e par
  \begin{align*}
    u_{j}^{n+1}
    &= u_j^n + \dfrac{h_t}{2 h_x^2} ( u_{j+1}^{n+1} - 2 u_{j}^{n+1} + u_{j-1}^{n+1} + u_{j+1}^{n} - 2 u_{j}^n + u_{j-1}^{n} ) ,
    %  \quad 0 \leq j \leq M, \; 0 \leq n \leq N-1 ,
    \\
    u_0^n
    &= u_M^n \qquad \qquad \text{ pour tout } n \in \llbracket 0, N \rrbracket
    \\
    u_j^0
    &= u_0(x_j) \qquad \;\;\;\! \text{ pour tout } j \in \llbracket 0,M \rrbracket
  \end{align*}
  % Pour simplifier, on notera dans la suite $c = \frac{a h_t}{h_x}$ le nombre de CFL.

  \begin{enumerate}
  \item Ce sch\'ema est-il explicite ou implicite ? Justifier.
  \item {\bf Stabilit\'e : } On s'int\'eresse \`a la stabilit\'e de Von Neumann de ce sch\'ema.
    \begin{itemize}
    \item Donner l'expression de $\calA_j(k)$ tel que $u_j^{n+1} = \calA_j(k) u_j^n$
      (en supposant que $u_j^n = e^{2i \pi k x_j}$ et $u_j^{n+1} = \calA_j(k) e^{2i \pi k x_j}$)
    \item Calculer $| \calA_j(k) |^2$. En d\'eduire que le sch\'ema est stable au sens
      de Von Neumann. A-t-on besoin d'une condition de CFL ?
    \end{itemize}
  \item {\bf Consistance : } On va montrer que ce sch\'ema est consistant d'ordre 2 en temps
    et 2 en espace. \\
    Pour simplifier, on note $\partial_t u = \dfrac{\partial u}{\partial t}$
    et $\partial_{xx}^2 u = \dfrac{\partial^2 u}{\partial x^2}$.
    \begin{itemize}
    \item Donner l'expression de l'erreur de consistance.
    \item Montrer que $u(t,x_{j+1}) - 2 u(t,x_j) + u(t,x_{j-1}) = h_x^2 \partial_{xx}^2 u(t,x_j) + O(h_x^4)$
      pour $t \in \{ t_n , t_{n+1} \}$.
    \item Montrer que $u(t_{n+1},x_j) - u(t_n,x_j) = h_t \partial_t u(t_{n} + \frac{h_t}2 , x_j)
      + O(h_t^3)$.
    \item Montrer que $\partial_{xx}^2 u(t_{n+1},x_j) + \partial_{xx}^2 u(t_{n},x_j)
      = 2 \partial_{xx}^2 u(t_{n} + \frac{h_t}{2},x_j) + O(h_t^2)$.
    \item En d\'eduire que le sch\'ema est d'ordre 2 en espace et en temps.
    \end{itemize}
    
  \end{enumerate}
\end{exo}

%%%%%%%%%%%%%%%%%%%%%%%%%%%%%%%%%%%%%%%%%%%%%%%%%%%%%%%%%%%%%%%%%%%%%%%%%%%%%%%%%%%%%%%%%%%%%%%%%%%%
%%%%%%%%%%%%%%%%%%%%%%%%%%%%%%%%%%%%%%%%%%%%%%%%%%%%%%%%%%%%%%%%%%%%%%%%%%%%%%%%%%%%%%%%%%%%%%%%%%%%
%%%%%%%%%%%%%%%%%%%%%%%%%%%%%%%%%     CORRECTION    %%%%%%%%%%%%%%%%%%%%%%%%%%%%%%%%%%%%%%%%%%%%%%%%
%%%%%%%%%%%%%%%%%%%%%%%%%%%%%%%%%%%%%%%%%%%%%%%%%%%%%%%%%%%%%%%%%%%%%%%%%%%%%%%%%%%%%%%%%%%%%%%%%%%%
%%%%%%%%%%%%%%%%%%%%%%%%%%%%%%%%%%%%%%%%%%%%%%%%%%%%%%%%%%%%%%%%%%%%%%%%%%%%%%%%%%%%%%%%%%%%%%%%%%%%
\newpage

%%%%%%%%%%%%%%%%%%%%%%%%%%%%%%%%%%%%%%%%%%

\begin{cor} $\quad$
  \\
  \begin{enumerate}
  \item Soit $u$ une solution $C^2(\overline{\Omega})$ de~\eqref{eq:pb:1}--\eqref{eq:pb:2}.
    Pour $v \in H^1(\Omega)$, on multiplie l'\'equation par $v$ et on int\`egre sur le domaine.
    On a
    \begin{align*}
      \int_{\Omega} -u'' v = \int_{\Omega} f v .
    \end{align*}
    Puis en int\'egrant par parties (et en utilisant $u'(0) = u(0)$ et $u'(1) = - u(1)$) on obtient
    \begin{align*}
      \int_{\Omega} u' v' - [u' v]^1_0
      &= \int_{\Omega} f v ,
      \\
      \int_{\Omega} u' v' dx + u(0)v(0) + u(1) v(1)
      &= \int_{\Omega} f v .
    \end{align*}
    De plus, toute fonction fortement d\'erivable est aussi faiblement d\'erivable,
    et toute fonction continue sur un compact $K$ est dans $L^2(K)$.
    On a donc $u \in C^2(\overline{\Omega}) \subset C^1(\overline{\Omega}) \subset H^1(\Omega)$.
    On a donc montr\'e que $u$ est solution du probl\`eme~\eqref{eq:pb_var}.
    
    [2 pts]
  \item
    Montrons que toutes les hypoth\`eses du th\'eor\`eme de Lax-Milgram sont v\'erifi\'ees:

    $\bullet$ On rappelle que $V = H^1(\Omega)$ est un espace de Hilbert
    pour le produit scalaire $(u,v)_{H^1} = \int_{\Omega} (u'v' + u v)$.

    $\bullet$ $\ell(v) = \int_0^1 fv$ est bien une forme lin\'eaire sur $V$
    (pas besoin de d\'etailler cette partie).
    Si $f \in L^2(0,1)$, alors $\ell$ est continue car
    \begin{align*}
      | \ell(v) |
      &= \left| \int_{\Omega} f v \right| \leq \| f \|_{L^2} \| v \|_{L^2}
        \leq \| f \|_{L^2} \| v \|_{H^1} ,
    \end{align*}
    o\`u on a utilis\'e l'in\'egalit\'e de Cauchy-Schwarz.


    $\bullet$ $a$ est une forme bilin\'eaire sur $V$.
    Elle est continue car
    \begin{align*}
      | a(u,v) |
      &= \left| \int_{\Omega} u'v' dx + u(0) v(0) + u(1) v(1) \right|
      \\
      &\leq \left| \int_{\Omega} u'v' \right|
        + | u(0) | |v(0)| + |u(1)| | v(1)|
      \\
      &\leq \| u' \|_{L^2} \| v' \|_{L^2}
        + 2 C^2 \| u \|_{H^1} \| v \|_{H^1}
      \leq (1+2C^2) \| u \|_{H^1} \| v \|_{H^1} ,
    \end{align*}
    o\`u on a utilis\'e l'in\'egalit\'e de Cauchy-Schwarz et l'in\'egalit\'e de trace.

    $\bullet$ $a$ est coercive car
    pour $v \in H^1(0,1)$, on a $a(v,v) = \int_0^1 (v')^2 dx + v(0)^2 + v(1)^2$.
    Donc $\| v' \|_{L^2}^2 = \int_0^1 (v')^2 \leq a(v,v)$.
    De plus, l'in\'egalit\'e de Poincar\'e correspond \`a
    \begin{align*}
      \| v \|_{L^2}^2 \leq C a(v,v) .
    \end{align*}
    On a donc
    \begin{align*}
      \| v \|_{H^1}^2 = \| v \|_{L^2}^2 + \| v' \|_{L^2}^2
      \leq (1+C) a(v,v) .
    \end{align*}
    La forme bilin\'eaire $a$ est donc coercive.
    

    En appliquant le th\'eor\`eme de Lax-Milgram, le probl\`eme~\eqref{eq:pb_var}
    admet une unique solution.
    
    [4 pts]

  \item Le probl\`eme \'el\'ements finis associ\'e est
    \begin{align*}
      \text{Trouver } u_h \in V_{h} \text{ telle que }
      a(u_h , v_h) = \ell(v_h) \quad \forall v_h \in V_{h} ,
    \end{align*}
    avec
    \begin{align*}
      V_{h}
      &:= \{ v_h \in C^0(\overline{\Omega}) \; | \; v_h \vert_{[x_{j-1} , x_{j}]} \text{ est affine pour }
         j \in \llbracket 1,M \rrbracket \} ,
    \end{align*}
    et les points de discr\'etisation sont d\'efinis par $x_j = jh$ et $h = 1/M$
    avec $M$ le nombre de sous-intervalles consid\'er\'es.

    [1 pt]
    
  \item On rappelle que $V_{h}$ est un sous-espace vectoriel de dimension finie de $H^1(0,1)$,
    c'est donc un Hilbert pour le produit scalaire $(\cdot,\cdot)_{H^1}$.
    On a d\'ej\`a montr\'e les autres hypoth\`eses du th\'eor\`eme de Lax-Milgram
    dans les questions pr\'ec\'edentes.
    Le probl\`eme \'el\'ements finis admet donc une unique solution.

    [1 pt]

  \item On a $u_h \in V_{h}$, on peut donc d\'ecomposer
    \begin{align*}
      u_h = \sum_{j=0}^{M} u_h(x_j) \Phi_j ,
    \end{align*}
    avec $(\Phi_j)_{0 \leq j \leq M}$ la base canonique de $V_{h}$ d\'efinie par
    $\Phi_j(x_k) = \delta_{jk}$.

    On a donc, par lin\'earit\'e,
    \begin{align*}
      \sum_{j=0}^{M} u_h(x_j) a(\Phi_j , v_h) = \ell(v_h) \qquad \forall v_h \in V_{h} .
    \end{align*}
    En prenant $v_h = \Phi_i$, on obtient
    \begin{align*}
      \sum_{j=0}^{M} u_h(x_j) a(\Phi_j , \Phi_i) = \ell(\Phi_i) \qquad \forall i \in \llbracket 0 , M \rrbracket .
    \end{align*}
    Ceci revient \`a r\'esoudre le syst\`eme lin\'eaire
    \begin{align*}
      A U = F ,
    \end{align*}
    avec $a_{ij} = a(\Phi_j , \Phi_i)$
    et $F_i = \ell(\Phi_i)$ pour d\'eterminer les coefficients de $u_h$ dans la base :
    $U_i = u_h(x_i)$.

    [2 pts]


  \end{enumerate}
  
\end{cor}


\begin{cor}
  $\quad$
  \\
  \begin{enumerate}
  \item Plusieurs termes au temps $t_{n+1}$ sont pr\'esents dans la d\'efinition du sch\'ema.
    Ainsi, pour calculer $(u_j^{n+1})_{0 \leq j \leq M}$ \`a partir de $(u_j^{n})_{0 \leq j \leq M}$,
    il faut r\'esoudre une \'equation. Le sch\'ema est donc implicite.

    [1 pt]
  \item
    $\bullet$ Pour simplifier, on ne consid\`ere pas les $j \in \{0,M\}$.
    En supposant $u^n_j = e^{2i \pi k x_j}$ et $u^{n+1}_j = \calA_j(k) e^{2i \pi k x_j}$, on a
    \begin{align*}
      u_{j}^{n+1}
      = u_j^n + \dfrac{h_t}{2 h_x^2} \Big( u_{j}^{n+1} e^{2i \pi k h_x} - 2 u_{j}^{n+1} + u_{j}^{n+1} e^{-2i \pi k h_x} + u_{j}^{n} e^{2i \pi k h_x} - 2 u_{j}^n + u_{j}^{n} e^{-2i \pi k h_x} \Big) ,
      \\
      u_{j}^{n+1} \Big[ 1 - \dfrac{h_t}{2 h_x^2} \Big( e^{2i \pi k h_x} - 2 + e^{-2i \pi k h_x} \Big) \Big]
      = u_j^n \Big[ 1 + \dfrac{h_t}{2 h_x^2} \Big( e^{2i \pi k h_x} - 2 + e^{-2i \pi k h_x} \Big) \Big] .
    \end{align*}
    On obtient donc $u_j^{n+1} = \calA_j(k) u_j^n$ avec
    \begin{align*}
      \calA_j(k) = \dfrac{1 + \dfrac{h_t}{2 h_x^2} \Big( e^{2i \pi k h_x} - 2 + e^{-2i \pi k h_x} \Big)}{1 - \dfrac{h_t}{2 h_x^2} \Big( e^{2i \pi k h_x} - 2 + e^{-2i \pi k h_x} \Big)}
      = \dfrac{1 + \dfrac{h_t}{h_x^2} \Big( \cos(2 \pi k h_x) - 1 \Big)}{1 - \dfrac{h_t}{h_x^2} \Big( \cos(2 \pi k h_x) - 1 \Big)} .
    \end{align*}

    [2 pts]
    \\
    $\bullet$ Montrons maintenant que $|\calA_j(k)| \leq 1$ sans aucune condition.
    On a $-1 \leq \cos(2\pi k h_x) \leq 1$ donc $-2 \leq \cos(2\pi k h_x) - 1 \leq 0$.
    Ainsi, $-\frac{2 h_t}{h_x^2} \leq \frac{h_t}{h_x^2} (\cos(2\pi k h_x) - 1) \leq 0$.

    En utilisant $\frac{h_t}{h_x^2} (\cos(2\pi k h_x) - 1) \leq 0$, on obtient
    \begin{align*}
      1 - \frac{h_t}{h_x^2} (\cos(2\pi k h_x) - 1) \geq 1 \geq 1 + \frac{h_t}{h_x^2} (\cos(2\pi k h_x) - 1) .
    \end{align*}
    donc
    \begin{align*}
      \calA_j(k) \leq 1 .
    \end{align*}

    Il reste \`a prouver $\calA_j(k) \geq -1$. Pour cela, on \'etudie $\calA_j(k) + 1$
    \begin{align*}
      \calA_j(k) + 1
      &= \dfrac{1 + \dfrac{h_t}{h_x^2} \Big( \cos(2 \pi k h_x) - 1 \Big)}{1 - \dfrac{h_t}{h_x^2} \Big( \cos(2 \pi k h_x) - 1 \Big)} + 1
      \\
      &= \dfrac{2}{1 - \dfrac{h_t}{h_x^2} \Big( \cos(2 \pi k h_x) - 1 \Big)} \geq 0 .
    \end{align*}
    Ainsi, on a bien $\calA_j(k) \geq -1$. Et donc $| \calA_j(k) | \leq 1$.

    Le sch\'ema est donc inconditionnellement stable au sens de Von Neumann.

    
    [2 pts]

  \item
    $\bullet$ L'erreur de consistance est d\'efinie par
    \begin{align*}
      \varepsilon_j^{n+1} = u(t_{n+1}, x_j) - u(t_n,x_j) - \frac{h_t}{2 h_x^2}
      \Big(& u(t_{n+1}, x_{j+1}) - 2 u(t_{n+1}, x_{j}) + u(t_{n+1}, x_{j-1})
      \\
      &+ u(t_{n}, x_{j+1}) - 2 u(t_{n}, x_{j}) + u(t_{n}, x_{j-1}) \Big) .
    \end{align*}

    [1 pt]
    
    $\bullet$ Pour $t \in \{t_{n+1} , t_n \}$, on consid\`ere les d\'eveloppements de Taylor suivants
    \begin{align*}
      u(t,x_{j+1}) = u(t,x_j) + h_x \partial_x u(t,x_j) + \frac{h_x^2}{2} \partial_{xx}^2 u(t,x_j)
      + \frac{h_x^3}{6} \partial_{xxx}^3 u(t,x_j) + O(h_x^4) ,
      \\
      u(t,x_{j-1}) = u(t,x_j) - h_x \partial_x u(t,x_j) + \frac{h_x^2}{2} \partial_{xx}^2 u(t,x_j)
      - \frac{h_x^3}{6} \partial_{xxx}^3 u(t,x_j) + O(h_x^4) .
    \end{align*}
    En combinant ces deux lignes, on obtient
    \begin{align*}
      u(t, x_{j+1}) - 2 u(t, x_{j}) + u(t, x_{j-1})
      =
      h_x^2 \partial_{xx}^2 u(t,x_j) + O(h_x^4) .
    \end{align*}
    [1 pt]

    $\bullet$ On note $t_{n+\frac12} = t_n + \frac{h_t}2$.
    On utilise les d\'eveloppements de Taylor
    \begin{align*}
      u(t_{n+1} , x_j) = u(t_{n+\frac12} + \frac{h_t}{2}, x_j)
      = u(t_{n+\frac12}, x_j) + \frac{h_t}{2} \partial_t u(t_{n+\frac12}, x_j)
      + \frac{h_t^2}{8} \partial_{tt}^2 u(t_{n+\frac12}, x_j) + O(h_t^3) ,
      \\
      u(t_{n} , x_j) = u(t_{n+\frac12} - \frac{h_t}{2}, x_j)
      = u(t_{n+\frac12}, x_j) - \frac{h_t}{2} \partial_t u(t_{n+\frac12}, x_j)
      + \frac{h_t^2}{8} \partial_{tt}^2 u(t_{n+\frac12}, x_j) + O(h_t^3) .
    \end{align*}
    En soustrayant ces deux lignes, on obtient
    \begin{align*}
      u(t_{n+1} , x_j) - u(t_{n} , x_j) = h_t \partial_t u(t_{n+\frac12}, x_j) + O(h_t^3) .
    \end{align*}
    [1 pt]

    $\bullet$ De fa\c{c}on similaire,
    \begin{align*}
      \partial_{xx}^2 u(t_{n+1},x_j)
      &= \partial_{xx}^2 u(t_{n+\frac12},x_j) + \frac{h_t}2 \partial_{xxt}^3 u(t_{n+\frac12},x_j) + O(h_t^2),
      \\
      \partial_{xx}^2 u(t_{n},x_j)
      &= \partial_{xx}^2 u(t_{n+\frac12},x_j) - \frac{h_t}2 \partial_{xxt}^3 u(t_{n+\frac12},x_j) + O(h_t^2).
    \end{align*}
    Et en sommant ces deux lignes
    \begin{align*}
      \partial_{xx}^2 u(t_{n+1},x_j) + \partial_{xx}^2 u(t_{n},x_j)
      = 2 \partial_{xx}^2 u(t_{n+\frac12},x_j) + O(h_t^2) .
    \end{align*}
    [1 pt]

    $\bullet$ Pour conclure sur l'ordre du sch\'ema, combinons les r\'esultats
    de toutes les \'etapes pr\'ec\'edentes.
    \begin{align*}
      \varepsilon_j^{n+1}
      &= u(t_{n+1}, x_j) - u(t_n,x_j) - \frac{h_t}{2 h_x^2}
      \Big(
      & u(t_{n+1}, x_{j+1}) - 2 u(t_{n+1}, x_{j}) + u(t_{n+1}, x_{j-1})
      \\
      &&+ u(t_{n}, x_{j+1}) - 2 u(t_{n}, x_{j}) + u(t_{n}, x_{j-1}) \Big)
      \\
      &= u(t_{n+1}, x_j) - u(t_n,x_j) - \frac{h_t}{2 h_x^2}
        \Big(
        &h_x^2 \partial_{xx}^2 u(t_{n+1},x_j) + h_x^2 \partial_{xx}^2 u(t_{n},x_j)
        + O(h_x^4) \Big)
      \\
      &= h_t \partial_t u(t_{n+\frac12}, x_j) + O(h_t^3)
        - \frac{h_t}2 \Big(
      &\partial_{xx}^2 u(t_{n+1},x_j) + \partial_{xx}^2 u(t_{n},x_j) + O(h_x^2) \Big)
      \\
      &= h_t \partial_t u(t_{n+\frac12}, x_j) + O(h_t^3)
      &- h_t \partial_{xx}^2 u(t_{n+\frac12},x_j) + O(h_t h_x^2)
      \\
      &= O(h_t^3 + h_t h_x^2) .
    \end{align*}
    Le sch\'ema est donc d'ordre deux en temps et en espace.
    
    [1 pt]
  \end{enumerate}
  
\end{cor}

\end{document}
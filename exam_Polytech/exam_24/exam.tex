\documentclass[12pt]{article}
\usepackage{tikz}
\usepackage{amsfonts} %% mathbb
\usepackage{amsmath} %% align
\usepackage{listings} %% pour le code
\usepackage{stmaryrd} %% pour llbracket et rrbracket

\begin{document}
\noindent
{\rule{\textwidth}{.2mm}}\\
\renewcommand{\labelenumi}{(\alph{enumi})}
\noindent
Polytech Sorbonne \hfill Ann{\'e}e universitaire 2023--2024\\
Analyse Numérique des EDP \hfill
MAIN4 \hfill mercredi 15 mai 2024
{\rule{\textwidth}{.2mm}}\\


\newtheorem{exercice}{\bf Exercice}
\newenvironment{exo}{\begin{exercice}
\rm}{\end{exercice} }
\newtheorem{corrige}{\bf Corrig{\'e}}
\newenvironment{cor}{\begin{corrige}\rm}{\end{corrige} }

\newcommand{\R}{\mathbb{R}}
\newcommand{\Z}{\mathbb{Z}}
\newcommand{\GRAD}{\nabla}
\newcommand{\SCAL}{\cdot}
\newcommand{\polP}{\mathbb{P}}
\newcommand{\calT}{\mathcal{T}}
\newcommand{\calA}{\mathcal{A}}

\newcommand{\tu}{\widetilde{u}}

\setcounter{MaxMatrixCols}{15} % for matrices of size more than 10

\begin{center}
{\bf Examen}
\end{center}

\begin{exo}
  \textit{Discr\'etisation par \'el\'ements finis (environ 9 pts)}

  On note $\Omega = ]0,1[$. On s'int\'eresse au probl\`eme
  \begin{align}
    \label{eq:pb:1}
    - u'' &= f \text{ dans } \Omega
    \\
    u'(0) &= u(0) ,
    \\
    \label{eq:pb:2}
     u'(1) &= - u(1) ,
  \end{align}
  o\`u $f : \Omega \to \R$ une fonction donn\'ee.
  
  \begin{enumerate}
  \item Prouver que toute solution $u \in C^2(\overline{\Omega})$ de~\eqref{eq:pb:1}--\eqref{eq:pb:2}
    est aussi solution de la formulation variationnelle suivante:
    \begin{align}
      \label{eq:pb_var}
      \text{Trouver } u \in H^1(\Omega) \text{ telle que }
      a(u,v)
      = \int_{0}^1 f v \qquad
      \forall v \in H^{1}(\Omega) ,
    \end{align}
    o\`u $a(u,v) := \int_{0}^1 u'(x) v'(x) dx + u(0) v(0) + u(1) v(1)$.
  \item On se donne l'in\'egalit\'e de Poincar\'e suivante: il existe $C > 0$ telle que
    \[
      \int_{0}^1 (v(x))^2 dx \leq C \Big( v(0)^2 + v(1)^2 + \int_0^1 (v'(x))^2 dx \Big) .
    \]
    En d\'eduire que si $f \in L^2(\Omega)$, alors
    le probl\`eme~\eqref{eq:pb_var} admet une unique solution.

  \item Proposer une discr\'etisation \'el\'ements finis du probl\`eme consid\'er\'e.
    On rappelera la d\'efinition des points de discr\'etisation et de l'espace de fonctions utilis\'e.

  \item Montrer que le probl\`eme \'el\'ements finis admet une unique solution.

  \item Donner l'\'ecriture alg\'ebrique du probl\`eme \'el\'ements finis
    (le syst\`eme lin\'eaire \`a r\'esoudre pour trouver la solution).
    On ne vous demande \textbf{pas} de calculer les coefficients de la matrice.
  \end{enumerate}

\end{exo}

\newpage
\begin{exo}
  \textit{Discr\'etisation par diff\'erences finies (environ 11 pts)}

  On se place dans le domaine temporel $]0,T[$ ($T > 0$) et spatial $]0,1[$.
  On veut discr\'etiser l'\'equation de la chaleur avec conditions aux limites p\'eriodiques
  \begin{align*}
    \frac{\partial u}{\partial t} - \frac{\partial^2 u}{\partial x^2}
    &= 0 \quad \text{ sur } ]0,T[ \times ]0,1[ ,
    \\
    u(t,0)
    &= u(t,1) \qquad \forall t \in [0,T] ,
    \\
    u(0,x)
    &= u_0(x) \qquad \forall x \in [0,1] .
  \end{align*}
  Pour cela, on discr\'etise l'espace par $M+1$ points $x_j = jh_x$ ($h_x = 1/M$)
  et le temps par $N+1$ points $t_n = n h_t$ ($h_t = T/N$).
  On consid\`ere le sch\'ema donn\'e par
  \begin{align*}
    u_{j}^{n+1}
    &= u_j^n + \dfrac{h_t}{2 h_x^2} ( u_{j+1}^{n+1} - 2 u_{j}^{n+1} + u_{j-1}^{n+1} + u_{j+1}^{n} - 2 u_{j}^n + u_{j-1}^{n} ) ,
    %  \quad 0 \leq j \leq M, \; 0 \leq n \leq N-1 ,
    \\
    u_0^n
    &= u_M^n \qquad \qquad \text{ pour tout } n \in \llbracket 0, N \rrbracket
    \\
    u_j^0
    &= u_0(x_j) \qquad \;\;\;\! \text{ pour tout } j \in \llbracket 0,M \rrbracket
  \end{align*}
  % Pour simplifier, on notera dans la suite $c = \frac{a h_t}{h_x}$ le nombre de CFL.

  \begin{enumerate}
  \item Ce sch\'ema est-il explicite ou implicite ? Justifier.
  \item {\bf Stabilit\'e : } On s'int\'eresse \`a la stabilit\'e de Von Neumann de ce sch\'ema.
    \begin{itemize}
    \item Donner l'expression de $\calA_j(k)$ tel que $u_j^{n+1} = \calA_j(k) u_j^n$
      (en supposant que $u_j^n = e^{2i \pi k x_j}$ et $u_j^{n+1} = \calA_j(k) e^{2i \pi k x_j}$)
    \item Calculer $| \calA_j(k) |^2$. En d\'eduire que le sch\'ema est stable au sens
      de Von Neumann. A-t-on besoin d'une condition de CFL ?
    \end{itemize}
  \item {\bf Consistance : } On va montrer que ce sch\'ema est consistant d'ordre 2 en temps
    et 2 en espace. \\
    Pour simplifier, on note $\partial_t u = \dfrac{\partial u}{\partial t}$
    et $\partial_{xx}^2 u = \dfrac{\partial^2 u}{\partial x^2}$.
    \begin{itemize}
    \item Donner l'expression de l'erreur de consistance.
    \item Montrer que $u(t,x_{j+1}) - 2 u(t,x_j) + u(t,x_{j-1}) = h_x^2 \partial_{xx}^2 u(t,x_j) + O(h_x^4)$
      pour $t \in \{ t_n , t_{n+1} \}$.
    \item Montrer que $u(t_{n+1},x_j) - u(t_n,x_j) = \partial_t u(t_{n} + \frac{h_t}2 , x_j)
      + O(h_t^3)$.
    \item Montrer que $\partial_{xx}^2 u(t_{n+1},x_j) + \partial_{xx}^2 u(t_{n},x_j)
      = 2 \partial_{xx}^2 u(t_{n} + \frac{h_t}{2},x_j) + O(h_t^2)$.
    \item En d\'eduire que le sch\'ema est d'ordre 2 en espace et en temps.
    \end{itemize}
    
  \end{enumerate}
\end{exo}

%%%%%%%%%%%%%%%%%%%%%%%%%%%%%%%%%%%%%%%%%%%%%%%%%%%%%%%%%%%%%%%%%%%%%%%%%%%%%%%%%%%%%%%%%%%%%%%%%%%%
%%%%%%%%%%%%%%%%%%%%%%%%%%%%%%%%%%%%%%%%%%%%%%%%%%%%%%%%%%%%%%%%%%%%%%%%%%%%%%%%%%%%%%%%%%%%%%%%%%%%
%%%%%%%%%%%%%%%%%%%%%%%%%%%%%%%%%     CORRECTION    %%%%%%%%%%%%%%%%%%%%%%%%%%%%%%%%%%%%%%%%%%%%%%%%
%%%%%%%%%%%%%%%%%%%%%%%%%%%%%%%%%%%%%%%%%%%%%%%%%%%%%%%%%%%%%%%%%%%%%%%%%%%%%%%%%%%%%%%%%%%%%%%%%%%%
%%%%%%%%%%%%%%%%%%%%%%%%%%%%%%%%%%%%%%%%%%%%%%%%%%%%%%%%%%%%%%%%%%%%%%%%%%%%%%%%%%%%%%%%%%%%%%%%%%%%
\newpage

%%%%%%%%%%%%%%%%%%%%%%%%%%%%%%%%%%%%%%%%%%

\begin{cor} $\quad$
  \\
  \begin{enumerate}
  \item Soit $u$ une solution $C^2(\overline{\Omega})$ de~\eqref{eq:pb:1}--\eqref{eq:pb:2}.
    Pour $v \in H^1_0(\Omega)$, on multiplie l'\'equation par $v$ et on int\`egre sur le domaine.
    On a
    \begin{align*}
      \int_{\Omega} (-u'' + u' + u) v = \int_{\Omega} f v .
    \end{align*}
    Puis en int\'egrant par parties (et en utilisant $u(0) = u(1) = 0$) on obtient
    \begin{align*}
      \int_{\Omega} \Big( u' v' + u' v + u v \Big) = \int_{\Omega} f v .
    \end{align*}
    De plus, toute fonction fortement d\'erivable est aussi faiblement d\'erivable,
    et toute fonction continue sur un compact $K$ est dans $L^2(K)$.
    On a donc $u \in C^2(\overline{\Omega}) \subset C^1(\overline{\Omega}) \subset H^1(\Omega)$.
    On a donc montr\'e que $u$ est solution du probl\`eme~\eqref{eq:pb_var}.
    
    [2 pts]
  \item
    Pour $v \in H^1_0(0,1)$, on a
    \begin{align*}
      \int_0^1 v' v = \int_0^1 \left(\frac12 v^2 \right)' = \left[\frac12 v^2 \right]_0^1 = 0 ,
    \end{align*}
    o\`u on a utilis\'e $v(0) = v(1) = 0$.


    Montrons maintenant que toutes les hypoth\`eses du th\'eor\`eme de Lax-Milgram sont v\'erifi\'ees:

    $\bullet$ On rappelle que $V = H_0^1(\Omega)$ est un espace de Hilbert
    pour le produit scalaire $(u,v)_{H^1} = \int_{\Omega} (u'v' + u v)$.

    $\bullet$ $\ell(v) = \int_0^1 fv$ est bien une forme lin\'eaire sur $V$
    (pas besoin de d\'etailler cette partie).
    Si $f \in L^2(0,1)$, alors $\ell$ est continue car
    \begin{align*}
      | \ell(v) |
      &= \left| \int_{\Omega} f v \right| \leq \| f \|_{L^2} \| v \|_{L^2}
        \leq \| f \|_{L^2} \| v \|_{H^1} ,
    \end{align*}
    o\`u on a utilis\'e l'in\'egalit\'e de Cauchy-Schwarz.


    $\bullet$ $a$ est une forme bilin\'eaire sur $V$.
    Elle est continue car
    \begin{align*}
      | a(u,v) |
      &= \left| \int_{\Omega} (u'v' + u'v + u v) \right|
        \leq \left| \int_{\Omega} (u'v' + u v) \right|
        + \left| \int_{\Omega} u'v \right|
      \\
      &\leq \| u \|_{H^1} \| v \|_{H^1}
        + \left(\int_{\Omega} (u')^2 \right)^{\frac12} \left(\int_{\Omega} v^2 \right)^{\frac12}
      \leq 2 \| u \|_{H^1} \| v \|_{H^1} ,
    \end{align*}
    o\`u on a utilis\'e l'in\'egalit\'e de Cauchy-Schwarz.

    $\bullet$ $a$ est coercive car
    pour $v \in H^1_0(0,1)$, on a $a(v,v) = \int_0^1 (v')^2 + v'v + v^2 = \| v \|_{H^1}^2$.
    

    En appliquant le th\'eor\`eme de Lax-Milgram, le probl\`eme~\eqref{eq:pb_var}
    admet une unique solution.
    
    [3 pts]

  \item Le probl\`eme \'el\'ements finis associ\'e est
    \begin{align*}
      \text{Trouver } u_h \in V_{h0} \text{ telle que }
      a(u_h , v_h) = \ell(v_h) \quad \forall v_h \in V_{h0} ,
    \end{align*}
    avec
    \begin{align*}
      V_{h0}
      &:= \{ v_h \in C^0(\overline{\Omega}) \; | \; v_h \vert_{[x_{j-1} , x_{j}]} \text{ est affine pour }
         j \in \llbracket 1,M \rrbracket
        \text{ et } v_h(0) = v_h(1) = 0 \} ,
    \end{align*}
    et les points de discr\'etisation sont d\'efinis par $x_j = jh$ et $h = 1/M$
    avec $M$ le nombre de sous-intervalles consid\'er\'es.

    [1 pt]
    
  \item On rappelle que $V_{h0}$ est un sous-espace vectoriel de dimension finie de $H_0^1(0,1)$,
    c'est donc un Hilbert pour le produit scalaire $(\cdot,\cdot)_{H^1}$.
    On a d\'ej\`a montr\'e les autres hypoth\`eses du th\'eor\`eme de Lax-Milgram
    dans les questions pr\'ec\'edentes.
    Le probl\`eme \'el\'ements finis admet donc une unique solution.

    [1 pt]

  \item On a $u_h \in V_{h0}$, on peut donc d\'ecomposer
    \begin{align*}
      u_h = \sum_{j=1}^{M-1} u_h(x_j) \Phi_j ,
    \end{align*}
    avec $(\Phi_j)_{1 \leq j \leq M-1}$ la base canonique de $V_{h0}$ d\'efinie par
    $\Phi_j(x_k) = \delta_{jk}$.

    On a donc, par lin\'earit\'e,
    \begin{align*}
      \sum_{j=1}^{M-1} u_h(x_j) a(\Phi_j , v_h) = \ell(v_h) \qquad \forall v_h \in V_{h0} .
    \end{align*}
    En prenant $v_h = \Phi_i$, on obtient
    \begin{align*}
      \sum_{j=1}^{M-1} u_h(x_j) a(\Phi_j , \Phi_i) = \ell(\Phi_i) \qquad \forall i \in \llbracket 1 , M-1 \rrbracket .
    \end{align*}
    Ceci revient \`a r\'esoudre le syst\`eme lin\'eaire
    \begin{align*}
      A U = F ,
    \end{align*}
    avec $a_{ij} = a(\Phi_j , \Phi_i)$ (attention : ici la matrice n'est pas sym\'etrique)
    et $F_i = \ell(\Phi_i)$ pour d\'eterminer les coefficients de $u_h$ dans la base :
    $U_i = u_h(x_i)$.

    [2 pts]


  \end{enumerate}
  
\end{cor}


\begin{cor}
  $\quad$
  \\
  Dans toute la suite, on note
  $c = \frac{a h_t}{h_x}$ le nombre de cfl.
  \begin{enumerate}
  \item
    Comme $u_0^n = u_M^n$, on s'int\'eresse aux $u_j^n$ ($1 \leq j \leq M$).
    Le sch\'ema est d\'efini par
    \begin{align*}
      u_{j}^{n+1}
      &= u_j^n - \dfrac{c}{2} ( u_{j+1}^n - u_{j-1}^n ) + \frac{u_{j+1}^n - 2 u_j^n + u_{j-1}^n}{2}
      \\
      &= \left( \frac{1-c}2 \right) u_{j+1}^n + \left( \frac{1+c}2 \right) u_{j-1}^n
    \end{align*}
    Cette relation est vraie m\^eme pour $j \in \{1,M\}$, en posant
    $u_0^n = u_M^n$ et $u_{M+1}^n = u_1^n$.
    On obtient donc la relation $U^{n-1} = A U^n$ pour la matrice
    \begin{align*}
      \begin{pmatrix}
        0 & \frac{1-c}2 & (0) & \frac{1+c}2
        \\
        \frac{1+c}2 & \ddots & \ddots &  (0)
        \\
        (0)& \ddots & \ddots & \frac{1-c}2
        \\
        \frac{1-c}2 & (0) & \frac{1+c}2 & 0
      \end{pmatrix} .
    \end{align*}
    (Ne pas oublier les conditions p\'eriodiques)

    [2 pts]

  \item
    On se place dans le cas o\`u $0 < c \leq 1$, donc $1+c > 0$ et $1-c \geq 0$
    On a
    \begin{align*}
      u_{j}^{n+1} = \left( \frac{1-c}2 \right) u_{j+1}^n + \left( \frac{1+c}2 \right) u_{j-1}^n
    \end{align*}
    donc
    \begin{align*}
      | u_{j}^{n+1} |
      &= \left( \frac{1-c}2 \right) | u_{j+1}^n | + \left( \frac{1+c}2 \right) |u_{j-1}^n|
      \\
      &\leq \left( \frac{1-c}2 \right) \| U^n \|_{\infty}
        + \left( \frac{1+c}2 \right) \| U^n \|_{\infty}
      \\
      &\leq \| U^n \|_{\infty} .
    \end{align*}
    Ainsi, $\| U^{n+1} \|_{\infty} \leq \| U^n \|_{\infty}$ donc $||| A |||_{\infty} \leq 1$
    et le sch\'ema est donc stable en norme $\| \cdot \|_{\infty}$.
    
    [2 pts]
    
  \item
    Soit $k \in \Z$.
    On consid\`ere une condition initiale de la forme
    $u^0_j = e^{2i\pi k x_j}$.
    On a alors
    \begin{align*}
      u_j^1
      &= \frac{u_{j+1}^0 + u_{j-1}^0}{2} - \frac{c}2 (u_{j+1}^0 - u_{j-1}^0)
      \\
      &= \frac{e^{2i\pi k h_x} + e^{-2i\pi k h_x}}{2} u_j^0 - \frac{c}2 (e^{2i\pi k h_x} - e^{-2i\pi k h_x})u_j^0
      \\
      &= (\cos(2\pi k h_x) - ci \sin(2\pi k h_x)) u_j^0
    \end{align*}
    On peut donc \'ecrire $u_j^1 = \calA_j(k) u_j^0$
    avec $\calA_j(k) = \cos(2\pi k h_x) - ci \sin(2\pi k h_x)$.

    Montrons maintenant que, pour $c > 1$, ce sch\'ema n'est pas stable au sens de Von Neumann
    (i.e. $|\calA_j(k)| > 1$ pour un certain $j$ et un certain $k$).
    On a
    \begin{align*}
      |\calA_j(k)|^2
      = \cos^2(2\pi k h_x) + c^2 \sin^2(2\pi k h_x)
      = 1 + (c^2-1) \sin^2(2\pi k h_x) .
    \end{align*}
    Il existe un $k \in \Z$ tel que $\sin^2(2\pi k h_x) > 0$ (pour $h_x$ assez petit)
    et alors, pour $c > 1$, $|\calA_j(k)|^2 > 1$ et le sch\'ema est instable au sens de Von Neumann.

    [3 pts]

  \item
    On \'etudie l'erreur de consistance
    \begin{align*}
      \varepsilon_j^{n+1}
      &= u(t_{n+1},x_j)
      - \left[
      \frac12 u(t_n,x_{j+1}) + \frac12 u(t_n , x_{j-1}) - \frac{c}2 (u(t_n,x_{j+1}) - u(t_n,x_{j-1}))
      \right]
    \end{align*}
    On suppose que $u$ est aussi r\'eguli\`ere que l'on veut.
    On a les d\'eveloppements limit\'es
    \begin{align*}
      u(t_n,x_{j+1})
      &= u(t_n,x_j) + h_x \frac{\partial u}{\partial x}(t_n,x_j)
        + \frac{h_x^2}{2} \frac{\partial^2 u}{\partial x^2}(t_n,x_j) + O(h_x^3)
      \\
      u(t_n,x_{j-1})
      &= u(t_n,x_j) - h_x \frac{\partial u}{\partial x}(t_n,x_j)
        + \frac{h_x^2}{2} \frac{\partial^2 u}{\partial x^2}(t_n,x_j) + O(h_x^3)
    \end{align*}
    et donc
    \begin{align*}
      \frac{u(t_n,x_{j+1}) - u(t_n,x_{j-1})}{2}
      &= h_x \frac{\partial u}{\partial x}(t_n , x_{j}) + O(h_x^3)
      \\
      \frac{u(t_n,x_{j+1}) + u(t_n,x_{j-1})}{2}
      &= u(t_n , x_j) + O(h_x^2)
    \end{align*}
    De plus, on a aussi le d\'eveloppement limit\'e
    \begin{align*}
      u(t_{n+1},x_j) = u(t_n,x_j) + h_t \frac{\partial u}{\partial t}(t_n,x_j)
      + O(h_t^2)
    \end{align*}
    En combinant tout, on obtient
    \begin{align*}
      \varepsilon_j^{n+1}
      &= h_t \frac{\partial u}{\partial t}(t_n,x_j) + O(h_t^2 + h_x^2)
        + c\left(h_x \frac{\partial u}{\partial x}(t_n,x_j) +O(h_x^3) \right)
      \\
      &= O(h_t^2 + h_t h_x^2 + h_x^2)
    \end{align*}
    En utilisant la condition $h_x \leq C h_t$, le $O(h_x^2)$ peut \^etre r\'e\'ecrit comme un $O(h_t h_x)$.
    On perd un $h_t$ en sommant toutes les erreurs de consistance.
    On a donc un sch\'ema d'ordre 1 en temps et 1 en espace.

    [3 pts]
    
  \item
    Ayant prouv\'e la stabilit\'e et la consistance du sch\'ema en norme $\| \cdot \|_{\infty}$, on peut invoquer le th\'eor\`eme de Lax.
    Sous la condition de CFL $a h_t \leq h_x$ et sous la condition $h_x \leq C h_t$ (pour n'importe quelle constante $C>0$),
    on a l'estim\'ee d'erreur
    \begin{align*}
      \max_{\substack{0 \leq n \leq N \\ 0 \leq j \leq M}} |u(t_n,x_j) - u_j^n| \leq \tilde{C}(h_t + h_x) ,
    \end{align*}
    avec $\tilde{C} > 0$.

    [1 pt]
  \end{enumerate}
  
\end{cor}

\end{document}
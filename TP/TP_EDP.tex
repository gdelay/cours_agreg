\documentclass[12pt]{article}
\usepackage[utf8]{inputenc}
\usepackage[T1]{fontenc}
\usepackage[frenchb]{babel}
\usepackage{tikz}
\usetikzlibrary{arrows}
\usepackage{times}    
\usepackage{amsfonts,amssymb,amsmath,amsthm}
%
\newtheorem{exercice}{\color{coul2}{\bf Exercice}} 
\newenvironment{exo}{
%\vskip .1cm
\begin{exercice}\smallskip\normalfont}{\end{exercice}
%\vskip .1cm
} 
 
% \definecolor{coul2}{rgb}{.0,0.2,0.8}
% \definecolor{coul1}{rgb}{.49,0.98,0.96}
% \def\titre{\centerline{
% \includegraphics[width=3cm]{su.png}
% \hfill \begin{tabular}{r}
% {\color{coul1}{\bf Master Math\'ematiques et Applications}}\\ 
% {\color{coul2}{\bf Approfondissement C/C++}} \\ \\{}\end{tabular}     } 
% }


\newcommand*{\N}{\mathbb{N}}
\newcommand*{\R}{\mathbb{R}}

\newif\ifcorrige\corrigetrue
%\newif\ifcorrige\corrigefalse 
\begin{document}
% \titre
\begin{center}
  \underline{\underline{\LARGE TP EDP : Laplacien et \'equation de transport}}
\end{center}
\vskip .5cm 

%%%%%%%%%%%%%%%%%%%%%%%%
\section{Probl\`eme de Laplace et diff\'erentes conditions aux limites}

On se place dans le cadre d'un domaine unidimensionnel $\Omega = (0,1)$.
On s'int\'eresse \`a l'\'equation de Laplace
\begin{equation}
  \label{eq:Laplace}
  - u''(x) = f(x) ,
\end{equation}
o\`u on cherche $u$ et $f$ est une fonction connue.
Cette \'equation est compl\'et\'ee par des conditions aux limites (on consid\'erera
plusieurs possibilit\'es).
Le domaine $(0,1)$ est maill\'e en $M > 0$ sous-divisions de longueur $h = 1/M$.
On d\'efinit les points $x_j = jh$ ($0 \leq j \leq M$).
On veut ensuite construire une suite $(u_j)$ dont le terme g\'en\'eral approche la valeur
de $u$ au point $x_j$.
Le sch\'ema de diff\'erences finies choisi consiste \`a approcher
$u''(x_j) \simeq \frac{u(x_{j+1}) - 2 u(x_j) + u(x_{j-1})}{h^2}$.

Pour chaque condition aux limites consid\'er\'ee ci-dessous, on vous demande
de coder la matrice $A$ et le vecteur $F$.
On pourra ensuite calculer num\'eriquement $U$ et afficher sur un graphe la solution obtenue $(u_j)$
(en ordonn\'ees) en fonction de la position des points $(x_j)$ (en abscisses).
On pourra aussi comparer ce r\'esultat aux valeurs exactes $(u(x_j))$ (sur le m\^eme graphe).


%%%%%%%
\subsection{Conditions de Dirichlet homog\`enes}


On consid\`ere les conditions aux limites
\begin{equation}
  \label{eq:Dir_hom}
  u(0) = 0 , \qquad u(1) = 0 .
\end{equation}
La m\'ethode de diff\'erences finies consiste alors \`a r\'esoudre le probl\`eme
$A U = F$ avec
\begin{equation}
  \label{eq:matDir}
  A = \dfrac{1}{h^2} 
  \begin{pmatrix} 
    2 & -1 & 0 & \cdots
    \\
    -1 & 2 & -1 & 0 & \cdots
    \\
    0 & -1 & 2 & \ddots &
    \\
    \vdots & & \ddots & \ddots & \ddots
    \\
    & & &  \ddots & 2 & -1
    \\
    & &  & & -1 & 2
  \end{pmatrix} , \qquad U = 
  \begin{pmatrix}
    u_1 \\ u_2 \\ \vdots \\ u_{M-1}
  \end{pmatrix} , \qquad F = 
  \begin{pmatrix}
    f(x_1) \\ f(x_2) \\ \vdots \\ f(x_{M-1})
  \end{pmatrix} .
\end{equation}
La suite $(u_j)$ est alors donn\'ee par le vecteur $U$ ainsi que $u_0 = u_M = 0$.


On pourra par exemple consid\'erer $f(x)=x$ avec pour solution associ\'ee
$u(x) = \frac{x}{6}(1-x^2)$ (dans ce cas v\'erifier que l'on retrouve la solution exacte
par la m\'ethode des DF).
On pourra aussi consid\'erer $f(x) = x^2$ et $u(x) = \frac{x}{12}(1-x^3)$
(dans ce cas v\'erifier que l'on n'obtient pas exactement la solution exacte du probl\`eme).

%%%%%%%
\subsection{Conditions de Dirichlet non homog\`enes}
On consid\`ere les conditions aux limites
\begin{equation}
  \label{eq:Dir_non_hom}
  u(0) = \alpha , \qquad u(1) = \beta ,
\end{equation}
o\`u $\alpha$ et $\beta$ sont des r\'eels connus.
On a vu (cf cours) que, dans ce cas, la solution exacte est celle du probl\`eme homog\`ene
\`a laquelle on ajoute le rel\`evement $\alpha (1-x) + \beta x$.

La m\'ethode des diff\'erences finies consiste \`a r\'esoudre le probl\`eme
$A U = F$ avec $A$ et $U$ donn\'es par~\eqref{eq:matDir} et $F$ donn\'e par
\begin{equation}
  \label{eq:F_Dir_non_hom}
  F = 
  \begin{pmatrix}
    f(x_1) + \frac{\alpha}{h^2}\\ f(x_2) \\ \vdots \\ f(x_{M-2}) \\ f(x_{M-1}) + \frac{\beta}{h^2}
  \end{pmatrix} .
\end{equation}
Les valeurs de la suite sont alors celles de $U$ que l'on compl\`ete par
$u_0 = \alpha$ et $u_M = \beta$.

On pourra, par exemple, consid\'erer
$\alpha = 0$, $\beta = \frac1{12}$, $f(x) = x^2$ et $u(x) = \frac{x}{12}(1-x^3) + \frac{x}{12}$.
(N'h\'esitez pas \`a essayer d'autres valeurs.)

%%%%%%%
\subsection{Conditions mixtes}
On consid\`ere les conditions aux limites
\begin{equation}
  \label{eq:mixte}
  u(0) = 0 , \qquad u'(1) = 0 .
\end{equation}
On parle de conditions mixtes car il y a une condition de Dirichlet \`a gauche
et une condition de Neumann \`a droite.
On impose $u_0 = 0$ et $\frac{u_{M-1} - u_{M}}{h} = 0$.

On r\'esout le probl\`eme $A U = F$ avec
\begin{equation}
  \label{eq:matMixte}
  A = \dfrac{1}{h^2} 
  \begin{pmatrix} 
    2 & -1 & 0 & \cdots
    \\
    -1 & 2 & -1 & 0 & \cdots
    \\
    0 & -1 & 2 & \ddots &
    \\
    \vdots & & \ddots & \ddots & \ddots
    \\
    & & &  \ddots & 2 & -1
    \\
    & &  & & -1 & 1
  \end{pmatrix} , \qquad U = 
  \begin{pmatrix}
    u_1 \\ u_2 \\ \vdots \\ u_{M-1}
  \end{pmatrix} , \qquad F = 
  \begin{pmatrix}
    f(x_1) \\ f(x_2) \\ \vdots \\ f(x_{M-1})
  \end{pmatrix} .
\end{equation}
La suite $(u_j)$ est alors donn\'ee par le vecteur $U$ ainsi que $u_0 = 0$
et $u_M = u_{M-1}$.

On pourra, par exemple, consid\'erer $f(x) = x^2$ et $u(x) = \frac{x}{3}(1-\frac{x^3}{4})$.
%%%%%%%
\subsection{Conditions de Neumann (facultatif)}
(facultatif : \`a faire seulement apr\`es avoir trait\'e les autres sections)

A voir si on l'ajoute ...

%%%%%%%%%%%%%%%%%%%%%%%%
\section{\'Equation de transport}

On s'int\'eresse au probl\`eme de transport-Dirichlet
\begin{equation}
  \label{eq:transport}
  \left\{
    \begin{array}{l}
      \partial_t u + a \partial_x u = 0 ,
      \\
      u(t,0) = \alpha ,
      \\
      u(0,x) = u_0(x) ,
    \end{array}
  \right.
\end{equation}
o\`u $u$ est une fonction que l'on cherche \`a d\'eterminer et
$\alpha$ et $u_0$ sont donn\'es. On consid\`ere, de plus, que $a > 0$.

On se propose de comparer le comportement du sch\'ema centr\'e (instable)
avec le sch\'ema d\'ecentr\'e amont (stable sous condition CFL).
On pourra se reporter au cours pour plus d'informations.


Dans les deux cas, on divise l'intervalle d'espace $\Omega = (0,1)$ en
$M$ sous-intervalles (comme pour le Laplacien) et l'intervalle de temps
$(0,T)$ en $N$ sous-intervalles.
On pose, de plus, $h_x = 1/M$, $x_j = jh_x$ ($0 \leq j \leq M$),
$h_t = T/N$ et $t_n = nh_t$ ($0 \leq n \leq N$).
On cherche \`a construire une suite $(u_j^n)$ approchant $u(t_n,x_j)$.
On initialise cette suite par $u_j^0 = u_0(x_j)$ ($0 \leq j \leq M$).
On utilise ensuite l'un des deux sch\'emas propos\'es ci-dessous pour
calculer les pas de temps suivants.


Pour juger de la qualit\'e du r\'esultat, on affichera la solution au temps final $T$.
Pour simplifier, on pourra par exemple consid\'erer $\alpha = 0$, $a=1$, $T=1$ et $u_0(x) = x$.

%%%%%%%
\subsection{Sch\'ema centr\'e}

Pour tout $n \geq 0$, on calcule la solution au pas de temps suivant par
\begin{equation}
  \label{eq:transport_centre}
  u_j^{n+1} = u_j^n - \dfrac{a h_t}{2 h_x} (u_{j+1}^n - u_{j-1}^n) , \qquad 1 \leq j \leq M-1 ,
\end{equation}
avec $u_0^{n+1} = \alpha$ et $u_M^{n+1} = u_M^n - \dfrac{a h_t}{2 h_x} (u_{M}^n - u_{M-1}^n)$.

Ce sch\'ema est cens\'e \^etre instable.
V\'erifier que lorsque l'on rafine $(h_t,h_x)$, la solution se d\'egrade.

%%%%%%%
\subsection{Sch\'ema d\'ecentr\'e}

%%%%%%%%%%%%%%%%%%%%%%%%
\section{\'Equation de la chaleur et des ondes (facultatif)}
(facultatif : \`a faire seulement apr\`es avoir trait\'e les autres sections)

Utiliser le code d\'evelopp\'e pour le Laplacien pour simuler l'\'equation des ondes
et de la chaleur (se reporter au cours pour la discr\'etisation).

\end{document}

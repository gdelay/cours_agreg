\documentclass[12pt]{article}
\usepackage[french]{babel}
\usepackage{times}
\usepackage{xcolor}
\usepackage{graphicx,epic,eepic}
\usepackage{amsfonts,amsmath} 
\usepackage{titlesec,color}
\def\begincorrige #1.{\noindent{\small \bf Corrig{\'e} de l'exercice #1}}
\def\endcorrige{ 
\small

}
%\pagestyle{empty}
% \oddsidemargin+2mm
% \evensidemargin+2mm
% \topmargin-2.5cm
% \textheight 242mm
%\textwidth160mm


\usepackage{listings}
\let\lst=\lstinline
\lstset
{
    basicstyle=\small\ttfamily,
    commentstyle=\color{magenta},
    keywordstyle=\color{couleurpython},
    frame=single,
    language=python,
    morekeywords={True, False},
    numbers=left,
    numbersep=10pt,
    numberstyle=\footnotesize\color{black},
    showstringspaces=false,
    stringstyle=\color{magenta},
    tabsize=3,
    numbers=none,
} 

\definecolor{couleurpython}{rgb}{.0,0. ,1}
\newcommand\python[1]{{\color{couleurpython}{\texttt{#1}}}}

%\setlength{\parindent}{0cm} \setlength{\parskip}{5mm}
\newcommand{\R}{\mathbb{R}}
\newcommand{\C}{\mathbb{C}}
\newcommand{\PP}{\mathbb{P}}
\newcommand{\N}{\mathbb{N}}
%\include{abrev}
 
\newtheorem{corrige}{\bf Corrig{{\'e} de
l'exercice}}
%\newtheorem{corrige}{} 
\newtheorem{exercice}{Exercice} 


\newenvironment{exo}{
\begin{exercice}\normalfont}{\end{exercice}
} 

\newenvironment{cor}{
\begin{corrige}\normalfont}{\end{corrige}
} 
 
\renewcommand{\thecorrige}{\empty{}} 
\newcommand{\NV}[1]{\left|\!\left| #1 \right|\!\right|}

\begin{document}

\newif\ifcorrige  
\corrigetrue%
\corrigefalse
 

\centerline{\Huge \bf   Travaux pratiques} 
 
\section{Listes, modules numpy, scipy, \ldots}
\begin{enumerate}
 \item  Liste, array
 
\begin{lstlisting}
import numpy  
L=range(7,10)
S= numpy.array(L)         # transforme la liste L en array
print(S)
S= numpy.zeros(7)         # S contient 7 fois le nombre 0
print(S)
S= numpy.ones(7)          # que contient S ?
print(S)
S= 24*numpy.ones(7)       # que contient S ?
print(S)
S= numpy.eye(7)           # que contient S ?   
print(S) 

import numpy as np
X=np.linspace(10,30)       # que contient X ?   
print(X)
X=np.linspace(10,30,10)    # que contient X ?  
print(X)
u = np.array(range(4,9))
type(u)
np.size(u)
np.shape(u) 
A = np.array([[1,2,3],[4,5,6]])
type(A)
np.size(A)
np.shape(A)
\end{lstlisting} 

 \item Produit de matrices 
 
\begin{lstlisting}
A = np.array([[1,2], [3,4]])
B = np.array([[1,0], [-1,1]])
C=A*B                #   C=?
D=np.dot(A,B)        #   D=?
\end{lstlisting} 

 \item  Exemple de fonction 
 
\begin{lstlisting} 
def ExempleA(n):
    A=np.zeros((n,n))
    for i in range(1,n-1):
        A[i,i-1], A[i,i], A[i,i+1] = -1,2,-1
    i=0
    A[i,i], A[i,i+1] = 2,-1
    i=n-1
    A[i,i-1], A[i,i] = -1,2
    return A 

n=6
X=ExempleA(n)
print(X)
\end{lstlisting} 

 \item Graphiques
 
\begin{lstlisting}
import matplotlib.pyplot as plt
from math import sin, exp
x=[z/20. for z in range(21)]
y=[sin(3*xk) for xk in x]  
z=[exp(-xk) for xk in x]  
plt.xlabel('x') ; plt.ylabel('sin(3x),exp(-x)') ;   # 'd' comme 'diamond'
plt.plot(x,y, '-r',label='sin(3x)', linewidth=2, marker='d',markersize=12)
plt.plot(x,z,'-b',label='exp(-x)', linewidth=2, marker='o', markersize=12)
plt.legend(title='Legende', loc=8)
plt.grid()
plt.show() 
\end{lstlisting}

 \item  Alg\`ebre matricielle
 
 \begin{enumerate}
 \item  Inverse (d\'econseillé pour des matrices de grande taille!), 
norme, rang, d\'eterminant, spectre, \ldots 

\begin{lstlisting}
from numpy.linalg import inv
n=7
A=ExempleA(n)
B=inv(A)
C=np.dot(A,B)
print("||AB-I|| = ",np.linalg.norm(C-np.eye(n))) 

A=ExempleA(5)
r=np.linalg.matrix_rank(A)

from scipy.linalg import expm
A=np.zeros((2,2))
print('np.exp(A) = \n', np.exp(A))      # Comparer
print('expm(A)   = \n', expm(A))        # np.exp(A) et expm(A)


from numpy.linalg import eig  # pour v.p 
n=3
A=ExempleA(n)
D, V = eig(A)
k=2
d,u=D[k],V[:,k]
z=np.dot(A,u)-d*u
print(np.linalg.norm(z))
\end{lstlisting}
 
 \item  Syst\`emes lin\'eaires 
  
\begin{lstlisting}
import numpy as np
from numpy import linalg   # sous librairie algebre lineaire
n=5
A=np.random.rand(n,n)
s=np.random.rand(n)
b=np.dot(A,s)
x=linalg.solve(A,b)
normex=linalg.norm(s-x) 
\end{lstlisting} 
 \end{enumerate}
 

 \item[$\bullet$] \'Equations non lin\'eaires
 \begin{enumerate}
 \item Cas scalaire
 
\begin{lstlisting}
# resoudre exp(-x)=sin(x) 
import matplotlib.pyplot as plt
x=np.linspace(0,10)
plt.plot(x,np.exp(-x),'b+',x,np.sin(x),'or')
from scipy.optimize import fsolve
from math import exp, sin 
def f(x):
    return exp(-x)-sin(x)

x=fsolve(f, 3)
print('   x = ', x) 
print('f(x) = ', f(x)) 
\end{lstlisting} 

 \item Cas vectoriel 
 
\begin{lstlisting}
# resoudre le systeme : 
# exp(x) + x y^2 =1, 
# sin(x+y) + x^2 =1

def f(p):
    x, y = p[0], p[1]
    q=[exp(x)+x*y*y-1,sin(x+y)+x*x-1]
    return q

x=  fsolve(f, (0,0))
print('   x = ', x) 
print('f(x) = ', f(x)) 
\end{lstlisting} 
\end{enumerate}
\end{enumerate}
\end{document}

\documentclass[12pt]{article}
\usepackage[french]{babel}
\usepackage{amssymb,times,amsmath,amsfonts}
\usepackage{a4,epic,eepic}
\usepackage{color,graphicx,alltt}
\usepackage{hyperref,eurosym}
\usepackage{caption}


\textwidth=17cm
\textheight=23cm
\topmargin=-1cm
\oddsidemargin=-1cm

\newcommand{\R}{\mathbb{R}}
\newcommand{\C}{\mathbb{C}}
 
 
\def\annee{Ann\'{e}e {2023}-{2024}}
\def\semestre{ }%{p{\'e}riode 2}
\def\cours{Option B} 
\def\coursbis{Pr\'eparation \`a l'agr\'egation de math\'ematiques}
%\def\titre{\centerline{\textbf{\Large  Micro-projets math\'ematiques}}}
\def\univ{Sorbonne Universit{\'e} - Université Paris Cit\'e}

\newtheorem{exercice}{\small \bf Exercice}
\newenvironment{exo}{

\begin{exercice}\normalfont}{\end{exercice}
\vskip .1cm
%\hrule 
} 


\newtheorem{exemple}{\small \bf Exemple}
\newenvironment{exa}{


\begin{exemple}\normalfont}{\end{exemple}
\vskip .1cm
%\hrule 
} 

\newtheorem{corrige}{\small \bf Corrig{\'e} de l'exercice}
\newenvironment{cor}{
\begin{corrige}\normalfont}{\end{corrige}
\vskip .1cm
\hrule 
} 


\begin{document}



\setcounter{page}{1}\noindent 
\univ\hfill
\annee \\
\cours\hfill \\
\coursbis \hfill
\par\bigskip

%\titre
\par\bigskip
\begin{center}
\Large{ \bf TP : Equations diff\'erentielles}
\end{center}
%%%%%%%%%%%%%%%%%%%%%%%%%

Dans ce TP, on aura besoin des modules suivants : \\ 
\texttt{
import numpy as np \\
import scipy as sc\\
import numpy.linalg as LA\\
import matplotlib.pyplot as plt \\
import scipy.integrate as integrate \vspace{0.3cm}\\
}
Le dernier module sera utilis\'e pour r\'esoudre des \'equations diff\'erentielles avec la commande \texttt{odeint}. 
\begin{exo}
  \label{ex:sch_exp}
  \textbf{Sch\'ema d'Euler, calcul d'erreur} \\
On consid\`ere le probl\`eme suivant :
$$
\left\{
\begin{array}{rl}
x'(t)&=x(t)+2y(t)\\
y'(t)&=2x(t)+y(t)
\end{array}
\right.
$$
sur l'intervalle $[0,1]$ avec la condition initiale $x(0)=5$ et $y(0)=0$.
\begin{enumerate}
\item Faire un programme afin de d\'eterminer la solution approch\'ee de ce syst\`eme \`a l'aide d'un sch\'ema d'Euler explicite. 
\item En prenant un pas de temps $\Delta t = 0.01$, tracer la solution approch\'ee de ce syst\`eme en fonction du temps.
\item On rappelle que la solution exacte de ce syst\`eme est donn\'ee par
  $x(t) = \frac52 (e^{-t} + e^{3t})$ et $y(t) = \frac52 (-e^{-t} + e^{3t})$. Comparer graphiquement la solution exacte et la solution approch\'ee pour diff\'erentes valeurs de $\Delta t$.
\item Pour $0\leq n \leq N$, on note $t_n=n\Delta t$ les points de discr\'etisation et $X_n$ la solution approch\'ee au temps $t_n$. \'Ecrire une routine calculant l'erreur $\max_{0\leq n \leq N}\|X(t_n)-X_n\|_{\infty}$. On pourra utiliser la commande \texttt{norm} du module \texttt{numpy.linalg} (avec l'option \texttt{np.inf}). 
\item Tracer  l'erreur en fonction de $\Delta t$ en \'echelle logarithmique. 
\item Utiliser la commande \texttt{np.polyfit} pour d\'eterminer la droite qui approche au sens des moindres carr\'es les points de la courbe tracée à la question précédente  (il s'agit de la droite de r\'egression). 
\item On observe que la pente de la droite est proche de $1$. Comment l'interpr\'eter ?
\item Reprendre les trois derni\`eres questions
  pour le sch\'ema du point milieu d\'efini dans l'exercice~\ref{ex:pt_milieu}.
\end{enumerate}
\end{exo}
\newpage

\begin{exo}\textbf{\'Etude d'un pendule : Portrait de phase}
  \label{ex:pendule}
  \\
  On consid\`ere un pendule de longueur $l$. L'\'evolution de $\theta$ l'angle du pendule par rapport \`a la verticale est donn\'ee par l'\'equation suivante :
$$
\theta''(t)+\frac{g}{l}\sin(\theta(t))=0.
$$
On d\'efinit $\omega=\theta'$, on a ainsi le syst\`eme 
\begin{equation*}
\left\{
\begin{array}{l}
\theta'(t)=\omega(t) \vspace{0.1cm}\\
\omega'(t)=-\displaystyle\frac{g}{l}\sin(\theta(t)).
\end{array}
\right.
\end{equation*}
On suppose que $\theta(0)$ et  $\omega(0)$ sont donn\'es. 
On veut tracer le portrait de phase de ce probl\`eme pour les param\`etres
$l=5$ et $g=9.81$.

\begin{enumerate}
\item Tracer la solution de ce probl\`eme dans le plan des phases pour la donn\'ee initiale
  $(\theta(0),\omega(0)) = (\theta_0,0)$ avec $\theta_0 \in [0 , 2\pi]$
  (prendre de nombreuses valeurs de $\theta_0$ et tracer toutes les solutions sur le m\^eme graphe).
  On pourra r\'esoudre le syst\`eme en utilisant la commande \texttt{odeint} du module \texttt{scipy.integrate}. \\

\begin{tabular}{|l|}
 \hline
La syntaxe pour la commande  \texttt{odeint}  est :\\
 \hspace{2cm}\texttt{sol=integrate.odeint(f,X0,t)}\\
o\`u \\
$\bullet$ \texttt{f} a \'et\'e d\'efinie de la fa\c con suivante : \\
 \texttt{ def f(X,t): ... }\\
(m\^eme si  \texttt{f}  ne d\'epend pas de \texttt{t}, il faut que l'argument apparaisse en seconde position en entr\'ee).\\ 
$\bullet$ \texttt{X0} correspond \`a la condition initiale \\
$\bullet$ \texttt{t}  est le vecteur contenant les temps en lesquels la solution est \'evalu\'ee (le premier c\oe fficient\\  du vecteur est le 
temps initial)\\
$\bullet$ \texttt{sol} est le vecteur contenant les valeurs prises par la solution \'evalu\'ee aux temps du vecteur \texttt{t}. \\
 \hline
\end{tabular}
 \vspace{0.3cm}
\item Pour compl\'eter le portrait de phase, ajouter plusieurs
  courbes issues des donn\'ees initiales
  $(-\pi , \omega_0)$ avec $\omega_0 > 0$ et $(3\pi , \omega_0)$ avec $\omega_0 < 0$.
\item Ajouter les points critiques sur cette figure.
\item Faisons maintenant un zoom pr\`es du point $(0,0)$.
  Sur une autre figure, repr\'esenter le diagramme des phases pr\`es du point $(0,0)$.
  Ajouter les isoclines.
\item Enfin, nous allons repr\'esenter $f$ sur cette figure.
  Pour cela, on pourra utiliser les commandes \texttt{np.meshgrid}, \texttt{np.hypot}
  et \texttt{plt.quiver}.

  \textbf{NB :} si vous essayez de repr\'esenter $f$ comme ici sur la premi\`ere figure,
  vous pourriez avoir des difficult\'es dues au fait que $\omega$ et $\theta$
  n'ont pas le m\^eme ordre de grandeur.
\end{enumerate}
\end{exo}


%
%\begin{exo}
%On consid\`ere le syst\`eme sur l'intervalle $[0,50]$ :
%$$
%\left\{
%\begin{array}{rl}
%x'&=-y-x(x^2+y^2)\\
%y'&=x-y(x^2+y^2)
%\end{array}
%\right.
%$$
%compl\'et\'e par la condition initiale $x(0)=1$ et $y(0)=0$.
%Tracer la solution en fonction du  temps et dans le plan de phase.
%\end{exo}

\newpage
\begin{exo}\textbf{Sch\'emas implicites}\\
  Dans cet exercice, nous allons utiliser les sch\'emas d'Euler implicite et de Crank--Nicolson
  pour r\'esoudre le probl\`eme de l'exercice~\ref{ex:sch_exp}.
\begin{enumerate}
\item Coder une routine \texttt{Newton(F,dF,X0,tol,maxiter)}
  qui renvoie la solution $X \in \R^2$ de l'\'equation $F(X) = 0$
  calcul\'ee par la m\'ethode de Newton (\texttt{dF} est la jacobienne de \texttt{F}).
\item En utilisant la question pr\'ec\'edente, coder le sch\'ema d'Euler implicite donn\'e par
  $X_{n+1} = X_n + h f(t_{n+1} , X_{n+1})$.
\item Coder le sch\'ema de Crank--Nicolson donn\'e par
  $X_{n+1} = X_n + \frac{h}2 (f(t_{n+1} , X_{n+1}) + f(t_{n} , X_{n}))$.
\item Retrouver num\'eriquement les taux de convergence de ces deux sch\'emas
  (voir la fin de l'exercice~\ref{ex:sch_exp}).
\end{enumerate}
\end{exo}


\begin{exo}\textbf{\'Etude d'un pendule : simulations num\'eriques.}
  \label{ex:pt_milieu}
  \\
  Reprenons le syst\`eme consid\'er\'e dans l'exercice~\ref{ex:pendule}.
Pour r\'esoudre ce probl\`eme sur $[0,T]$, on prend $N \in \mathbb{N}^*$, puis on d\'efinit le pas de discr\'etisation $h=\frac TN$ et  les temps de discr\'etisation $t_n=nh$ pour $0 \leq n \leq N$.  


\begin{enumerate}
\item D\'eterminer la solution approch\'ee de ce syst\`eme \`a l'aide d'un sch\'ema d'Euler explicite :
\begin{equation*}
\left\{
\begin{array}{l}
\theta_{n+1}=\theta_n+h\omega_{n}\vspace{0.1cm}\\
\omega_{n+1}=\omega_n-h\displaystyle\frac{g}{l}\sin(\theta_{n}).
\end{array}
\right.
\end{equation*}

On prendra $T=12,\, N=200,\,l=5,\,g=9.81$
et les conditions initiales 
$\theta(0)=\frac{\pi}{3},\,$ \\ $\omega(0)=0.$
\item Tracer $\theta$ et $\omega$ en fonction du temps.
\item Mettre en \oe uvre le sch\'ema du point-milieu. \vspace{0.2cm}\\
\begin{tabular}{|c|}
 \hline
Le \textbf{sch\'ema du point-milieu} pour approcher l'EDO $y'(t)=f(t,y(t))$ est donn\'e par :\\
$y_{n+1}=y_n+h f\displaystyle\left(t_n+\frac h2,y_n+\frac h2 f(t_n,y_n)\right)$\\
 \hline
\end{tabular}
\item Comparer  la solution donn\'ee par la fonction \texttt{odeint} et les solutions donn\'ees par ces deux méthodes (on pr\'esentera $\theta$ et $\omega$ sur deux graphes diff\'erents). 
\item Comparer les solutions sur le portrait de phases.
\item Refaire les questions pr\'ec\'edentes avec la m\'ethode d'Euler implicite (et comparer).
\end{enumerate}



 
\end{exo}


\begin{exo}\textbf{Syst\`eme de Lotka-Volterra}\\
On consid\`ere le syst\`eme de Lotka-Volterra sur l'intervalle $[0,30]$ :
$$
\left\{
\begin{array}{rl}
x'(t)&=x(t)(2-y(t)/10)\\
y'(t)&=y(t)(x(t)/10-4)
\end{array}
\right.
$$
avec les conditions initiales $x(0)=100$ et $y(0)=70$.
\begin{enumerate}
\item Tracer $x$ et $y$ en fonction du temps.
\item Tracer la solution dans l'espace des phases.
\item Tracer les isoclines dans l'espace des phases. 
\item Ajouter d'autres orbites dans l'espace des phases.
\item Repr\'esenter $f$ pour compl\'eter le portrait de phase.
\end{enumerate}
\end{exo}





\end{document}

















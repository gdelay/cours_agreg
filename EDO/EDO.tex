\documentclass[12pt,a4paper,twoside]{article}
\addtolength{\textheight}{80pt} \addtolength{\topmargin}{-50pt}
\textwidth 164mm \oddsidemargin -2.25mm \evensidemargin -2.25mm
\usepackage{amssymb}
\usepackage{amsmath}
\usepackage{amsthm} % theoremes
\usepackage[T1]{fontenc}
\usepackage[utf8]{inputenc} 
\usepackage[francais]{babel}
\usepackage{enumitem}
\usepackage{url}
\usepackage{graphicx}
\usepackage{comment}
\usepackage{xcolor}

%% pour les figures
\usepackage{tikz}
\usepackage{forloop}


%% pour les exercices
%\usepackage{exercise}

\begin{document}

%%%%%%%%%%%%%%%%%%%%%%%%%%%%%%%%%%%%%


\newcommand*{\R}{\mathbb{R}}

\begin{center}
{\bf \Huge Analyse du sch\'ema d'Euler explicite}
\end{center}

%\vfill
%===================================================================================

%%%%%%%%%%%%%%%%%%%%%%%%%%%%%%%%%%%%%%%%%%%%%%%%%%%%%%%%%%%%%%%%%%%%%%%%%%%%%%%%%%%%
\section{Pr\'esentation du sch\'ema d'Euler explicite}

On cherche \`a approcher num\'eriquement une EDO de la forme
\begin{equation}
  \label{eq:pb}
  \left\{
    \begin{array}{l}
      y'(t) = f(t,y(t)) \text{ sur } [t_0,T] ,
      \\
      y(t_0) = y_0 ,
    \end{array}
  \right.
\end{equation}
o\`u $f$ et $y_0$ sont des donn\'ees du probl\`eme.
Ici on consid\`ere que le probl\`eme est scalaire : $y(t) \in \R$.
On peut raisonner de mani\`ere analogue s'il est vectoriel:
$y(t) \in \R^d$.

Pour approcher num\'eriquement le probl\`eme \eqref{eq:pb}, on d\'ecoupe l'intervalle
$[t_0,T]$ en $N$ sous-intervalles de longueur uniforme que l'on appelle
pas de discr\'etisation
\begin{align*}
  h = \frac{T - t_0}{N} .
\end{align*}
On d\'efinit ainsi une discr\'etisation 
de cet intervalle $t_k = t_0 + kh$ avec $0 \leq k \leq N$.


Le sch\'ema d'Euler consiste \`a approcher la d\'eriv\'ee par un taux d'accroissement:
\begin{align}
  \label{eq:taux_acc}
  y'(t_k) \simeq \dfrac{y(t_{k+1}) - y(t_k)}{t_{k+1} - t_k } = \dfrac{y(t_{k+1}) - y(t_k)}{h} .
\end{align}
On cherche donc \`a calculer les termes d'une suite $(y_k)_{0 \leq k \leq N}$
en rempla\c{c}ant dans \eqref{eq:pb} la d\'eriv\'ee par le taux d'accroissement
\eqref{eq:taux_acc}.
Ceci revient donc \`a calculer la suite $(y_k)_{0 \leq k \leq N}$ par r\'ecurrence comme:
\begin{align*}
  \left\{
  \begin{array}{l}
    y_{k+1} = y_k + h f(t_k , y_k), \qquad 0 \leq k \leq N-1 ,
    \\
    y_0 = y(t_0) .
  \end{array}
  \right.
\end{align*}

Nous voyons que la valeur de $y_{k+1}$ peut \^etre calcul\'ee de mani\`ere explicite 
\`a partir de la valeur de $y_k$. C'est pour cela que l'on parle de sch\'ema
d'Euler explicite.


Dans l'ensemble de ce document on utilise les notations standards qui consistent
\`a noter $(y_k)$ la suite calcul\'ee par le sch\'ema d'Euler explicite
et $y(t)$ la solution exacte (solution de \eqref{eq:pb}) au temps $t \in [t_0,T]$.
Le but d'une telle approche est de construire une suite $(y_k)$ qui s'approche
des $(y(t_k))$. Nous allons maintenant \'etudier sous quelles conditions
$y_k$ est une bonne approximation de $y(t_k)$.
%%%%%%%%%%%%%%%%%%%%%%%%%%%%%%%%%%%%%%%%%%%%%%%%%%%%%%%%%%%%%%%%%%%%%%%%%%%%%%%%%%%%
\section{Analyse du sch\'ema}

!! consistance !!

!! stabilit\'e !!

!! convergence !!

!! ordre !!


%%%%%%%%%%%%%%%%%%%%%%%%%%%%%%%%%%%%%%%%%%%%%%%%%%%%%%%%%%%%%%%%%%%%%%%%%%%%%%%%%%%%
\section{Un exemple}

!! but : 3--4 pages !!

%===================================================================================

\end{document}
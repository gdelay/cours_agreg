\documentclass[12pt,a4paper,twoside]{article}
\addtolength{\textheight}{80pt} \addtolength{\topmargin}{-50pt}
\textwidth 164mm \oddsidemargin -2.25mm \evensidemargin -2.25mm
\usepackage{amssymb}
\usepackage{amsmath}
\usepackage{amsthm} % theoremes
\usepackage[T1]{fontenc}
\usepackage[utf8]{inputenc} 
\usepackage[francais]{babel}
\usepackage{enumitem}
\usepackage{url}
\usepackage{graphicx}
\usepackage{comment}
\usepackage{xcolor}

%% pour les figures
\usepackage{tikz}
\usepackage{forloop}


%% pour les exercices
%\usepackage{exercise}

\begin{document}

%%%%%%%%%%%%%%%%%%%%%%%%%%%%%%%%%%%%%
\newtheorem{theorem}{Th\'eor\`eme}
\newtheorem{definition}{D\'efinition}
\newcommand*{\R}{\mathbb{R}}

\begin{center}
{\bf \Huge Analyse du sch\'ema d'Euler explicite}
\end{center}

%\vfill
%===================================================================================

%%%%%%%%%%%%%%%%%%%%%%%%%%%%%%%%%%%%%%%%%%%%%%%%%%%%%%%%%%%%%%%%%%%%%%%%%%%%%%%%%%%%
\section{Pr\'esentation du sch\'ema d'Euler explicite}

On cherche \`a approcher num\'eriquement une EDO de la forme
\begin{equation}
  \label{eq:pb}
  \left\{
    \begin{array}{l}
      y'(t) = f(t,y(t)) \text{ sur } [t_0,T] ,
      \\
      y(t_0) = y_0 ,
    \end{array}
  \right.
\end{equation}
o\`u $f$ et $y_0$ sont des donn\'ees du probl\`eme.
Ici on consid\`ere que le probl\`eme est scalaire : $y(t) \in \R$.
On peut raisonner de mani\`ere analogue s'il est vectoriel:
$y(t) \in \R^d$.

Pour approcher num\'eriquement le probl\`eme \eqref{eq:pb}, on d\'ecoupe l'intervalle
$[t_0,T]$ en $N$ sous-intervalles de longueur uniforme.
La longueur de ces sous-intervalles est appel\'ee pas de discr\'etisation.
On la note
\begin{align*}
  h = \frac{T - t_0}{N} .
\end{align*}
On d\'efinit ainsi une discr\'etisation 
de l'intervalle $[t_0,T]$ avec les temps $t_k = t_0 + kh$ avec $0 \leq k \leq N$.


Le sch\'ema d'Euler consiste \`a approcher la d\'eriv\'ee par un taux d'accroissement:
\begin{align}
  \label{eq:taux_acc}
  y'(t_k) \simeq \dfrac{y(t_{k+1}) - y(t_k)}{t_{k+1} - t_k } = \dfrac{y(t_{k+1}) - y(t_k)}{h} .
\end{align}
On cherche donc \`a calculer les termes d'une suite $(y_k)_{0 \leq k \leq N}$
en rempla\c{c}ant dans \eqref{eq:pb} la d\'eriv\'ee par le taux d'accroissement
\eqref{eq:taux_acc}.
Ceci revient donc \`a calculer la suite $(y_k)_{0 \leq k \leq N}$ par r\'ecurrence comme:
\begin{align*}
  \left\{
  \begin{array}{l}
    y_{k+1} = y_k + h f(t_k , y_k), \qquad 0 \leq k \leq N-1 ,
    \\
    y_0 = y(t_0) .
  \end{array}
  \right.
\end{align*}

Nous voyons que la valeur de $y_{k+1}$ peut \^etre calcul\'ee de mani\`ere explicite 
\`a partir de la valeur de $y_k$. C'est pour cela que l'on parle de sch\'ema
d'Euler explicite.


Dans l'ensemble de ce document on utilise les notations standards qui consistent
\`a noter $(y_k)$ la suite calcul\'ee par le sch\'ema d'Euler explicite
et $y(t)$ la solution exacte (solution de \eqref{eq:pb}) au temps $t \in [t_0,T]$.
Le but d'une telle approche est de construire une suite $(y_k)$ qui s'approche
des $(y(t_k))$. Nous allons maintenant \'etudier sous quelles conditions
$y_k$ est une bonne approximation de $y(t_k)$.


%%%%%%%%%%%%%%%%%%%%%%%%%%%%%%%%%%%%%%%%%%%%%%%%%%%%%%%%%%%%%%%%%%%%%%%%%%%%%%%%%%%%
\section{Analyse du sch\'ema}

Pour introduire le sch\'ema d'Euler explicite, nous avons approch\'e une d\'eriv\'ee
par un taux d'accroissement. Cette approximation est correcte dans la limite
$h \to 0$ (ou de mani\`ere \'equivalente $N \to +\infty$). 
Nous allons maintenant \'etudier le comportement de ce sch\'ema dans cette limite.
Pour cela, nous d\'efinissons un certain nombre de notions.


Si l'on se place dans le cadre plus g\'en\'eral des sch\'emas explicites \`a un pas de temps,
le probl\`eme consiste (en consid\'erant que $y_0 = y(t_0)$) 
\`a construire une suite $(y_k)$ \`a partir de la relation de r\'ecurrence
\begin{equation}
  \label{eq:1pas}
  y_{k+1} = y_k + h F(t_k,y_k,h) ,
\end{equation}
o\`u $h$ est le pas de discr\'etisation et $F$ est une fonction \`a trois variables
donn\'ee qui d\'etermine le sch\'ema utilis\'e.
Par exemple, dans le cas du sch\'ema d'Euler explicite on a $F(t,y,h) = f(t,y)$.


\begin{definition}[Erreur de consistance]
  On appelle erreur de consistance la quantit\'e
  \begin{align*}
    \varepsilon_k = y(t_{k+1}) - y(t_k) - h F(t_k , y(t_k) , h ) .
  \end{align*}
\end{definition}
Il s'agit de la diff\'erence entre la solution exacte au temps $t_{k+1}$
et la solution g\'en\'er\'ee par le sch\'ema \eqref{eq:1pas}
\`a partir de la solution exacte au temps $t_k$.
Nous voyons donc que cette erreur correspond \`a l'erreur g\'en\'er\'ee
par le sch\'ema au cours de la r\'esolution dans l'intervalle
$[t_k , t_{k+1}]$.

\begin{definition}[Consistance]
  On dit que le sch\'ema \eqref{eq:1pas} est consistant si 
  \begin{align*}
    \lim_{h \to 0} \sum_{k=0}^{N-1} | \varepsilon_k | = 0 ,
  \end{align*}
  o\`u $\varepsilon_k$ est l'erreur de consistance d\'efinie pr\'ec\'edemment.
\end{definition}
Notons que la notion de consistance se fait par rapport \`a un probl\`eme continu
(ici \eqref{eq:pb}) puisqu'on utilise la solution exacte $y(t)$.
Cette notion consiste donc \`a dire que l'erreur du sch\'ema g\'en\'er\'ee
au cours de toutes les it\'erations tend vers 0 quand le pas de discr\'etisation
tend vers 0.

\begin{definition}[Stabilit\'e]
  Le sch\'ema \eqref{eq:1pas} est stable s'il existe une constante $C > 0$
  ind\'ependante de $N$ (et donc de $h$) telle que pour toute suite 
  $(\eta_k)_{0 \leq k \leq N}$, les suites $(y_k)$ et $(z_k)$ d\'efinies comme
  \begin{align*}
    y_0 \in \R \quad \text{et} \quad y_{k+1} = y_k + h F(t_k,y_k,h) ,
    \\
    z_0 = y_0 + \eta_0 \quad \text{et} \quad z_{k+1} = z_k + h F(t_k,z_k,h) + \eta_{k+1} ,
  \end{align*}
  v\'erifient
  \begin{align*}
    \sup_{0 \leq k \leq N} | y_k - z_k | \leq C \sum_{n=0}^N | \eta_n | .
  \end{align*}
\end{definition}

Cette notion correspond \`a dire que si une erreur de calcul s'introduit
(ici l'erreur de calcul est le $\eta_k$) alors la solution sera perturb\'ee
de mani\`ere "continue" par rapport \`a cette erreur de calcul.

En pratique, des erreurs de calcul apparaissent lors des op\'erations informatiques.
Ces erreurs sont tr\`es petites par rapport \`a la solution du probl\`eme.
Ainsi, si le sch\'ema est stable, ces erreurs seront n\'egligeables et ne d\'egraderont
pas le r\'esultat final.


\begin{definition}[Convergence]
  Le sch\'ema \eqref{eq:1pas} converge (ou est convergent) si
  \begin{align*}
    \lim_{h \to 0} \sup_{0 \leq k \leq N} | y_k - y(t_k) | = 0 .
  \end{align*}
\end{definition}

!! ordre !!

\begin{theorem}[...]
  a
\end{theorem}


%%%%%%%%%%%%%%%%%%%%%%%%%%%%%%%%%%%%%%%%%%%%%%%%%%%%%%%%%%%%%%%%%%%%%%%%%%%%%%%%%%%%
\section{Un exemple}

!! but : 3--4 pages !!

%===================================================================================

\end{document}
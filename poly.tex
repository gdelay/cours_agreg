\documentclass[12pt,a4paper,twoside]{article}
\addtolength{\textheight}{80pt} \addtolength{\topmargin}{-50pt}
\textwidth 164mm \oddsidemargin -2.25mm \evensidemargin -2.25mm
\usepackage{amssymb}
\usepackage{amsmath}
\usepackage[T1]{fontenc}
\usepackage[utf8]{inputenc} 
\usepackage[francais]{babel}
\usepackage{enumitem}
\usepackage{url}
\usepackage{graphicx}
\usepackage{comment}
\usepackage{xcolor}

\begin{document}

%%%%%%%%%%%%%%%%%%%%%%%%%%%%%%%%%%%%%

\newcommand{\Ni}[1]{{\left\|#1\right\|}_\infty}
\newcommand*{\R}{\mathbb{R}}
\newcommand*{\N}{\mathbb{N}}
\newcommand*{\Z}{\mathbb{Z}}
\newcommand*{\C}{\mathbb{C}}

\newcommand*{\DIV}{\nabla \cdot}
\newcommand*{\GRAD}{\nabla}
\newcommand*{\deriv}{\;\mathrm{d}}
\newcommand*{\SCAL}{\cdot}

\newcommand*{\dans}{\text{ dans }}
\newcommand*{\sur}{\text{ sur }}

\newcommand*{\st}{\; | \;}

\newcommand*{\calA}{\mathcal{A}}

%% maths
\newcommand*{\Stab}{\mathcal{S}}
\newcommand*{\tu}{\widetilde{u}}
\newcommand*{\tn}{|\!|\!|}
\newcommand*{\cfl}{\mathrm{cfl}}
\newcommand*{\co}{\mathrm{c}}
\newcommand*{\calF}{\mathcal{F}}
\newcommand*{\hu}{\hat u}
\newcommand*{\hf}{\hat f}

%% environnements
\newtheorem{theorem}{Th\'eor\`eme}
\newtheorem{proposition}{Proposition}
\newtheorem{remark}{Remarque}
\newtheorem{definition}{D\'efinition}
\newtheorem{corrolary}{Corrolaire}
\newtheorem{exercise}{Exercice}
\newtheorem{lemma}{Lemme}
% \noindent
% {\rule{\textwidth}{.2mm}}\\
% \renewcommand{\labelenumi}{(\alph{enumi})}
% \noindent
Polytech Sorbonne \hfill Ann{\'e}e universitaire 2023--2024\\
Analyse Numérique des EDP \hfill
MAIN4 \hfill mercredi 15 mai 2024
% {\rule{\textwidth}{.2mm}}\\

\begin{center}
{\bf \Huge Introduction aux \'equations aux d\'eriv\'ees partielles}
\end{center}

% \title{Introduction aux \'equations aux d\'eriv\'ees partielles}

%\vfill
%===================================================================================

%%%%%%%%%%%%%%%%%%%%%%%%%%%%%%%%%%%%%%%%%%%%%%%%%%%%%%%%%%%%%%%%%%%%%%%%%%%%%%%%%%%%
\section*{Avant-propos}

Le but de ce cours est de vous proposer une introduction \`a la th\'eorie
des \'equations aux d\'eriv\'ees partielles (EDP dans la suite).
Nous \'etudierons plusieurs \'equations ainsi que leur discr\'etisation
par la m\'ethode des diff\'erences finies.

Dans le cadre du programme officiel de la pr\'eparation \`a l'agr\'egation
de math\'ematiques (\'epreuve de mod\'elisation), nous aborderons notamment:
\begin{itemize}
\item des notions \'el\'ementaires portant sur les EDP classiques en dimension 1,
\item l'\'equation de transport lin\'eaire avec la m\'ethode des caract\'eristiques,
\item l'\'equation des ondes et de la chaleur; une r\'esolution par s\'erie de Fourier
  et transform\'ee de Fourier sera propos\'ee ainsi qu'une m\'ethode de s\'eparation
  des variables.
  Les aspects qualitatifs seront abord\'es.
\item les \'equations elliptiques avec l'utilisation du th\'eor\`eme de Lax--Milgram
\item des exemples de discr\'etisation des EDP en dimension 1 avec la m\'ethode des
  diff\'erences finies. L'\'etude des propri\'et\'es de ces discr\'etisations
  sera propos\'ee : notions de consistance, stabilit\'e, convergence et d'ordre.
\end{itemize}


Vous \^etes par ailleurs invit\'es \`a lire le rapport du jury (disponible sur internet).
Vous vous rendrez compte que le jury insiste notamment sur le fait que :
\begin{itemize}
\item l'\'epreuve de mod\'elisation, comme les autres, requiert de la rigueur math\'ematique,
\item il faut \'equilibrer sa pr\'esentation entre une pr\'esentation du mod\`ele \'etudi\'e,
  des preuves math\'ematiques rigoureuses, des illustrations informatiques,
\item il attend une prise de recul de la part des candidats.
  Il faudra donc notamment \^etre capable de critiquer les limites du mod\`ele pr\'esent\'e
  dans le texte, d'expliquer le comportement qualitatif de celui-ci 
  (par exemple expliquer ce qu'il se passe quand la valeur d'un param\`etre change)
  et \^etre capable de conclure sur la probl\'ematique de d\'epart.
\end{itemize}


Ce cours sera compos\'e de
\begin{itemize}
\item cinq s\'eances de cours de deux heures chacune,
\item une s\'eance de programmation de deux heures.
\end{itemize}
Dans une premi\`ere partie, nous pr\'esenterons les \'equations \'etudi\'ees
dans ce cours en donnant une id\'ee des probl\`emes physiques associ\'es.
Chacune des parties suivantes sera consacr\'ee \`a l'\'etude plus approfondie
d'une EDP. Nous y traiterons notamment les principales caract\'eristiques de cette EDP,
les outils utilis\'es pour mener des preuves ainsi qu'une discr\'etisation par diff\'erences finies.
Les EDP \'etudi\'ees dans la suite seront les \'equations elliptiques,
l'\'equation de transport, l'\'equation de la chaleur et enfin
l'\'equation des ondes.

%%%%%%%%%%%%%%%%%%%%%%%%%%%%%%%%%%%%%%%%%%%%%%%%%%%%%%%%%%%%%%%%%%%%%%%%%%%%%%%%%%%%
\section{Introduction}

a

%%%%%%%%%%%%%%%%%%%%%%%%%%%%%%%%%%%%%%%%%%%%%%%%%%%%%%%%%%%%%%%%%%%%%%%%%%%%%%%%%%%%
\section{\'Equations elliptiques}

%%%%%%%%%%%%%%%%%%%%%%%%%%%%%%%%%%%%%%%%%%%%%%%%%%%%%%%%%%%%%%%%%%%%%%%%%%%%%%%%%%%%
\section{L'\'equation de transport}

%%%%%%%%%%%%%%%%%%%%%%%%%%%%%%%%%%%%%%%%%%%%%%%%%%%%%%%%%%%%%%%%%%%%%%%%%%%%%%%%%%%%
\section{L'\'equation de la chaleur}

%%%%%%%%%%%%%%%%%%%%%%%%%%%%%%%%%%%%%%%%%%%%%%%%%%%%%%%%%%%%%%%%%%%%%%%%%%%%%%%%%%%%
\section{L'\'equation des ondes}


%===================================================================================

\end{document}
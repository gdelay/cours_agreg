\documentclass[12pt,a4paper,twoside]{article}
\addtolength{\textheight}{80pt} \addtolength{\topmargin}{-50pt}
\textwidth 164mm \oddsidemargin -2.25mm \evensidemargin -2.25mm
\usepackage{amssymb}
\usepackage{amsmath}
\usepackage{amsthm} % theoremes
\usepackage[T1]{fontenc}
\usepackage[utf8]{inputenc} 
\usepackage[francais]{babel}
\usepackage{enumitem}
\usepackage{url}
\usepackage{graphicx}
\usepackage{comment}
\usepackage{xcolor}

%% pour les figures
\usepackage{tikz}
\usepackage{forloop}

\begin{document}

%%%%%%%%%%%%%%%%%%%%%%%%%%%%%%%%%%%%%

\newcommand{\Ni}[1]{{\left\|#1\right\|}_\infty}
\newcommand*{\R}{\mathbb{R}}
\newcommand*{\N}{\mathbb{N}}
\newcommand*{\Z}{\mathbb{Z}}
\newcommand*{\C}{\mathbb{C}}

\newcommand*{\DIV}{\nabla \cdot}
\newcommand*{\GRAD}{\nabla}
\newcommand*{\deriv}{\;\mathrm{d}}
\newcommand*{\SCAL}{\cdot}

\newcommand*{\dans}{\text{ dans }}
\newcommand*{\sur}{\text{ sur }}

\newcommand*{\st}{\; | \;}

\newcommand*{\calA}{\mathcal{A}}

%% maths
\newcommand*{\Stab}{\mathcal{S}}
\newcommand*{\tu}{\widetilde{u}}
\newcommand*{\tn}{|\!|\!|}
\newcommand*{\cfl}{\mathrm{cfl}}
\newcommand*{\co}{\mathrm{c}}
\newcommand*{\calF}{\mathcal{F}}
\newcommand*{\hu}{\hat u}
\newcommand*{\hf}{\hat f}

%% environnements
\newtheorem{theorem}{Th\'eor\`eme}
\newtheorem{proposition}{Proposition}
\newtheorem{remark}{Remarque}
\newtheorem{definition}{D\'efinition}
\newtheorem{corrolary}{Corrolaire}
\newtheorem{exercise}{Exercice}
\newtheorem{lemma}{Lemme}
% \noindent
% {\rule{\textwidth}{.2mm}}\\
% \renewcommand{\labelenumi}{(\alph{enumi})}
% \noindent
Polytech Sorbonne \hfill Ann{\'e}e universitaire 2023--2024\\
Analyse Numérique des EDP \hfill
MAIN4 \hfill mercredi 15 mai 2024
% {\rule{\textwidth}{.2mm}}\\

\begin{center}
{\bf \Huge Introduction aux \'equations aux d\'eriv\'ees partielles}
\end{center}

% \title{Introduction aux \'equations aux d\'eriv\'ees partielles}

%\vfill
%===================================================================================

%%%%%%%%%%%%%%%%%%%%%%%%%%%%%%%%%%%%%%%%%%%%%%%%%%%%%%%%%%%%%%%%%%%%%%%%%%%%%%%%%%%%
\section*{Avant-propos}

Le but de ce cours est de vous proposer une introduction \`a la th\'eorie
des \'equations aux d\'eriv\'ees partielles (EDP dans la suite).
Nous \'etudierons plusieurs \'equations ainsi que leur discr\'etisation
par la m\'ethode des diff\'erences finies.

Dans le cadre du programme officiel de la pr\'eparation \`a l'agr\'egation
de math\'ematiques (\'epreuve de mod\'elisation), nous aborderons notamment:
\begin{itemize}
\item des notions \'el\'ementaires portant sur les EDP classiques en dimension 1,
\item l'\'equation de transport lin\'eaire avec la m\'ethode des caract\'eristiques,
\item l'\'equation des ondes et de la chaleur; une r\'esolution par s\'erie de Fourier
  et transform\'ee de Fourier sera propos\'ee ainsi qu'une m\'ethode de s\'eparation
  des variables.
  Les aspects qualitatifs seront abord\'es.
\item les \'equations elliptiques avec l'utilisation du th\'eor\`eme de Lax--Milgram
\item des exemples de discr\'etisation des EDP en dimension 1 avec la m\'ethode des
  diff\'erences finies. L'\'etude des propri\'et\'es de ces discr\'etisations
  sera propos\'ee : notions de consistance, stabilit\'e, convergence et d'ordre.
\end{itemize}


Vous \^etes par ailleurs invit\'es \`a lire le rapport du jury (disponible sur internet).
Vous vous rendrez compte que le jury insiste notamment sur le fait que :
\begin{itemize}
\item l'\'epreuve de mod\'elisation, comme les autres, requiert de la rigueur math\'ematique,
\item il faut \'equilibrer sa pr\'esentation entre une pr\'esentation du mod\`ele \'etudi\'e,
  des preuves math\'ematiques rigoureuses, des illustrations informatiques,
\item il attend une prise de recul de la part des candidats.
  Il faudra donc notamment \^etre capable de critiquer les limites du mod\`ele pr\'esent\'e
  dans le texte, d'expliquer le comportement qualitatif de celui-ci 
  (par exemple expliquer ce qu'il se passe quand la valeur d'un param\`etre change)
  et \^etre capable de conclure sur la probl\'ematique de d\'epart.
\end{itemize}


Ce cours sera compos\'e de
\begin{itemize}
\item cinq s\'eances de cours de deux heures chacune,
\item une s\'eance de programmation de deux heures.
\end{itemize}
Dans une premi\`ere partie, nous pr\'esenterons les \'equations \'etudi\'ees
dans ce cours en donnant une id\'ee des probl\`emes physiques associ\'es.
Chacune des parties suivantes sera consacr\'ee \`a l'\'etude plus approfondie
d'une EDP. Nous y traiterons notamment les principales caract\'eristiques de cette EDP,
les outils utilis\'es pour mener des preuves ainsi qu'une discr\'etisation par diff\'erences finies.
Les EDP \'etudi\'ees dans la suite seront les \'equations elliptiques,
l'\'equation de transport, l'\'equation de la chaleur et enfin
l'\'equation des ondes.

%%%%%%%%%%%%%%%%%%%%%%%%%%%%%%%%%%%%%%%%%%%%%%%%%%%%%%%%%%%%%%%%%%%%%%%%%%%%%%%%%%%%
\section{Pr\'esentation des EDP du cours}

Nous pr\'esentons dans cette section les EDP \'etudi\'ees dans la suite du cours.
Nous essayons de donner une signification physique aux diff\'erents termes.

Nous nous int\'eressons \`a des EDP de la forme
\begin{align*}
  a \dfrac{\partial^2 u}{\partial x^2} + b \dfrac{\partial^2 u}{\partial x \partial y}
  + c \dfrac{\partial^2 u}{\partial y^2} + d \dfrac{\partial u}{\partial x}
  + e \dfrac{\partial u}{\partial y} + f u = F .
\end{align*}
Pour d\'eterminer la nature de l'EDP, on associe \`a chaque d\'eriv\'ee
la variable qui correspond \`a la direction de d\'erivation.
L'\'equation pr\'ec\'edente devient donc
\begin{align*}
  a x^2 + b xy + c y^2 + d x + e y + f = A ,
\end{align*}
avec $A$ un r\'eel tel que l'ensemble des solutions soit non vide.
S'il s'agit de l'\'equation:
\begin{itemize}
\item d'une ellipse, on dira que l'\'equation est elliptique;
\item d'une parabole, on dira que l'\'equation est parabolique;
\item d'une hyperbole, on dira que l'\'equation est hyperbolique.
\end{itemize}
Cette d\'enomination n'est pas juste esth\'etique.
En effet, comme nous le verrons plus loin dans ce cours,
chacun de ces types d'\'equations dispose de propri\'et\'es sp\'ecifiques.

Notons \'egalement que les \'equations pr\'ec\'edentes d\'ependent de deux
variables d'espace. La d\'enomination pr\'ec\'edente se g\'en\'eralise 
dans le cas o\`u on aurait une seule variable ou strictement plus de deux variables.
Dans le cadre de ce cours, nous nous concentrerons sur l'\'etude d\'equations
avec une seule dimension d'espace.
On consid\'erera donc une seule variable $x$ dans le cas d'un probl\`eme stationnaire
et deux variables $t$ (le temps) et $x$ (l'espace) dans le cas d'un probl\`eme
instationaire.

Dans la suite de ce cours, on notera $\Omega$ un ouvert born\'e de $\R^d$
avec $d=1,2,3$.

%%%%%%%%%%%%%%%%%%%%%%%%%%%%%%%%%%%%%%%
\subsection{\'Equations elliptiques}
\label{subsec:elliptique}

Les \'equations elliptiques apparaissent principalement dans deux contextes que nous
allons maintenant aborder.
Le premier est le cas o\`u des particules circulent dans un domaine.
Ce probl\`eme est repr\'esent\'e sur la figure \ref{fig:flux}.
Dans la partie droite de cette figure, nous nous int\'eressons \`a un probl\`eme
o\`u des particules circulent dans un milieu unidimensionnel. La position est
rep\'er\'ee par la coordonn\'ee d'espace $x$. On note $u(x)$ la densit\'e
de particules en $x$. Certaines particules entrent ou sortent du domaine
en $x$, on les note $f(x)$ le terme source les repr\'esentant.
De plus, les particules se d\'eplacent \`a travers le domaine,
on note $q(x)$ le flux de particules en $x$
(le nombre de particules qui traversent l'axe vertical d'abscisse $x$).
Ce flux est n\'egatif si les particules vont vers la gauche ($x$ d\'ecroissants)
et positif si elles vont vers la droite ($x$ croissants).

\begin{figure}
\begin{tikzpicture}[scale = 3]
  \def\h{0.1}
\def\N{11.0}
\newcounter{itx}
\newcounter{ity}

%% pente
\forloop{itx}{0}{\value{itx} < \N}{
\forloop{ity}{1}{\value{ity} < \value{itx} }{
\draw (\arabic{itx}*\h,\arabic{ity}*\h) node{$\circ$};
}
}

%% plateau
\forloop{itx}{0}{\value{itx} < 9.0}{
\forloop{ity}{1}{\value{ity} < 11.0 }{
\draw (1.0+\h+\arabic{itx}*\h,\arabic{ity}*\h) node{$\circ$};
}
}

%% axe des x
\draw[->] (0.0,0.0) -- (2.5 , 0.0);
\draw (2.5,-0.1) node{$x$};

%% terme source 1
\forloop{itx}{0}{\value{itx} < 4.0}{
\draw[->] (1.2 + 2*\h * \arabic{itx}, 1.5) -- (1.2 + 2*\h * \arabic{itx}, 1.1);
\draw (1.2 + 2*\h * \arabic{itx}, 1.3) node{$\circ$};
\draw (1.2 + 2*\h * \arabic{itx}, 1.5) node{$\circ$};
}
\draw (2.0, 1.3) node{$f(x)$};

%% terme source 2
\forloop{itx}{0}{\value{itx} < 2.0}{
\draw[->] (0.2 + 2*\h * \arabic{itx}, 0.4) -- (0.2 + 2*\h * \arabic{itx}, 0.8);
\draw (0.2 + 2*\h * \arabic{itx}, 0.6) node{$\circ$};
\draw (0.2 + 2*\h * \arabic{itx}, 0.4) node{$\circ$};
}
\draw (0.0, 0.6) node{$f(x)$};


%% densite u
\draw[<->] (2.0, \h/2.0) -- (2.0, 1.05);
\draw (2.2, 0.6) node{$u(x)$};

%% flux q
\draw[->] (0.95, 0.9) -- (0.6, 0.9);
\draw (0.8, 1.0) node{$q(x)$};
\end{tikzpicture}
\begin{tikzpicture}[scale = 2.5]
  \def\h{0.1}
\def\N{11.0}

%% rectangle
\draw (0,0) -- (2,0) -- (2,1) -- (0,1) -- (0,0);
\draw (1.0, 0.5) node{$u(x)$};

%% source term
\forloop{itx}{0}{\value{itx} < 9.0}{
\draw[->] (0.2 + 2*\h * \arabic{itx}, 1.5) -- (0.2 + 2*\h * \arabic{itx}, 1.1);
}
\draw (2.0, 1.3) node{$f(x)$};

%% flux
\draw[->] (-0.6,0.5) -- (-0.1,0.5);
\draw (-0.4, 0.3) node{$q(x)$};
\draw[->] (2.1,0.5) -- (2.8,0.5);
\draw (2.5, 0.3) node{$q(x+\delta x)$};

%% axe des abscisses
\draw[->] (-0.4,-0.1) -- (2.3,-0.1);
\draw (0.0, -0.1) -- (0.0, -0.2);
\draw (0.0, -0.3) node{$x$};
\draw (2.0, -0.1) -- (2.0, -0.2);
\draw (2.0, -0.3) node{$x+\delta x$};
\end{tikzpicture}
\caption{Un flux de particules. Gauche: repr\'esentation du probl\`eme.
  Droite: \'equilibre des flux.}
\label{fig:flux}
\end{figure}


On s'int\'eresse au cas o\`u les flux sont \`a l'\'equilibre, il n'y
a donc pas d'accumulation de particules en aucun point de l'espace.
Le probl\`eme ne d\'epend pas du temps.


Si l'on consid\`ere un \'el\'ement du domaine de taille $\delta x$ comme
sur la droite de la figure \ref{fig:flux}, le nombre de particules doit
rester constant au cours du temps.
On obtient donc la relation de conservation
$q(x) - q(x+\delta x) + f(x) \delta x = 0$,
ce qui donne
\begin{align}
  \label{eq:eq_flux}
  \dfrac{\partial q}{\partial x} = f .
\end{align}

De plus, on consid\`ere que les particules fuient les zones de forte densit\'e:
le flux $q(x)$ est orient\'e dans le sens inverse du gradient de $u$.
On note donc 
\begin{align}
  \label{eq:def_flux}
  q(x) = - k(x) \dfrac{\partial u}{\partial x}(x) . 
\end{align}
Ici $k$ est un coefficient positif qui peut d\'ependre de l'espace.
Il traduit le rapport de proportionnalit\'e entre le gradient de la densit\'e
et le flux qui en r\'esulte. Ainsi, pour une densit\'e fix\'ee,
si $k$ est grand alors les particules circuleront facilement et le flux sera important;
\`a l'inverse, un $k$ petit traduit le fait que les particules ont du mal \`a circuler dans
le milieu. On dit que $k$ est un coeeficient de diffusion.

L'\'equation finale sur $u$ est donc
\begin{align*}
  - \dfrac{\partial}{\partial x} \left(k \dfrac{\partial u}{\partial x} \right) = f .
\end{align*}

Si l'on consid\`ere plusieurs dimensions d'espace, cette \'equation devient
\begin{align*}
  - \DIV \left(k \GRAD u \right) = f ,
\end{align*}
o\`u $\DIV$ et $\GRAD$ sont respectivement les op\'erateurs divergence et gradient.


En pratique les particules peuvent r\'eellement repr\'esenter des particules physiques,
dans ce cas $u$ sera une densit\'e de particules, $q$ un flux de particules, 
$f$ un terme repr\'esentant l'apparition ou la disparition de particules et
$k$ sera un coefficient de diffusion des particules.
On peut aussi dire que les particules repr\'esentent de l'\'energie thermique.
Dans ce cas, $u$ sera la temp\'erature, $q$ un flux thermique, $f$ une source
ou un puit de chaleur et $k$ sera la conductivit\'e thermique du mat\'eriau consid\'er\'e.
Nous citerons une derni\`ere possibilit\'e selon laquelle les particules sont des
individus (humains ou animaux).
Dans ce cas, $u$ correspond \`a une densit\'e de population,
$q$ \`a un flux de population, $f$ repr\'esente des naissances ou des morts
dans la population et $k$ est un coefficient de diffusion repr\'esentant la
facilit\'e avec laquelle la population peut se d\'eplacer. 


!! TABLEAU !!


!! CONDITIONS FRONTIERES !!

Un autre cas o\`u ces \'equations apparaissent est le cas d'un mat\'eriau 
soumis \`a des contraintes m\'ecaniques.
Par exemple, on repr\'esente sur la figure \ref{fig:barre} une barre \'elastique
en \'equilibre.


\begin{figure}
\begin{tikzpicture}[scale = 1.5]
  %% barre
\draw (0,0) -- (6,0) -- (6,1) -- (0,1) -- (0,0);

%% fixation
\draw (0, -0.5) -- (0, 1.5);
\forloop{itx}{0}{\value{itx} < 8}{
\draw (0, -0.25+0.25*\arabic{itx}) -- (-0.25, -0.5+0.25*\arabic{itx});
}


%% forces
\forloop{itx}{0}{\value{itx} < 8}{
\draw[->] (0.5+0.7*\arabic{itx},0.5) node{$\times$} -- (0.9+0.7*\arabic{itx},0.5);
}
\draw (4.2,0.25) node{$f$};

%% axe des abscisses
\draw[->] (-0.5, 2.0) -- (6.0,2.0);
\draw (6.0,1.7) node{$x$};


%% barre de reference
\draw[dashed] (0.0,2.5) -- (4,2.5) -- (4,3.5) -- (0,3.5) -- (0,2.5);
\draw[dashed] (0.0,2.5) -- (0.0,1.0);
\draw[dashed] (4,2.5) -- (6,1);


%% deplacement u
\draw[dashed] (2,2.5) -- (2,-1);
\draw (2,-1.2) node{$x$};
\draw[dashed] (2,2.5) -- (2.8,1);
\draw[dashed] (2.8,1) -- (2.8,-1);
\draw[->] (2,-0.5) -- (2.8,-0.5);
\draw (2.4,-0.7) node{$u(x)$};

\end{tikzpicture}
\begin{tikzpicture}[scale = 2]
  \def\h{0.1}
\def\N{11.0}

%% rectangle 1
\draw[dashed] (-0.8,1.0) -- (1.2,1.0) -- (1.2,1.5) -- (-0.8,1.5) -- (-0.8,1.0);

%% rectangle 2
\draw (0,0) -- (2,0) -- (2,0.5) -- (0,0.5) -- (0,0);

%% axe des abscisses 1
\draw[dashed,->] (-1.2,0.9) -- (1.8,0.9);
\draw[dashed] (-0.8, 0.9) -- (-0.8, 0.8);
\draw (-0.8, 0.7) node{$x$};
\draw[dashed] (1.2, 0.9) -- (1.2, 0.8);
\draw (1.2, 0.7) node{$x+\delta x$};

%% axe des abscisses 2
\draw[->] (-0.4,-0.1) -- (2.3,-0.1);
\draw (0.0, -0.1) -- (0.0, -0.2);
\draw (0.0, -0.3) node{$x+u(x)$};
\draw (2.0, -0.1) -- (2.0, -0.2);
\draw (2.0, -0.3) node{$x + \delta x + u(x+\delta x)$};


%% autres traits
\draw[dashed] (-0.8,1.0) -- (0,0.5);
\draw[dashed] (1.2,1.0) -- (2,0.5);

%% bilan des forces
\draw[->] (-0.5,0.3) -- (-0.05,0.3) node[below left]{$q(x)$};
\draw[->] (2.1,0.3) -- (3,0.3) node[below left]{$q(x+\delta x)$};
\draw[->] (0.4,0.3) node{$\times$} -- (0.6,0.3);
\draw[->] (0.9,0.3) node{$\times$} -- (1.1,0.3);
\draw[->] (1.4,0.3) node{$\times$} -- (1.6,0.3);
\draw (1,0.15) node{$f$};
\end{tikzpicture}
\caption{Un flux de particules. Gauche: repr\'esentation du probl\`eme.
  Droite: \'equilibre des flux.}
\label{fig:barre}
\end{figure}


La barre dans son \'etat initial est repr\'esent\'ee en pointill\'es.
Sous l'effet d'une force lin\'eique $f$, cette barre s'allonge
et atteint l'\'etat d'\'equilibre repr\'esent\'e en bas de la figure.
On note $u(x)$ le d\'eplacement de mati\`ere qui a eu lieu entre
l'\'etat sous la charge $f$ et l'\'etat de r\'ef\'erence en pointill\'es.


On consid\`ere que la barre est \`a l'\'equilibre m\'ecanique.
On note $q(x)$ la force qu'exerce la section de gauche sur la section de droite
en $x$. En faisant un bilan de force comme sur la partie droite de la figure \ref{fig:flux},
les forces s'exer\c{c}ant sur une portion infinit\'esimale de barre sont
la force $q(x)$ \`a gauche, la force $-q(x+\delta x)$ \`a droite
et la force lin\'eique $f(x) \delta x$. La barre \'etant \`a l'\'equilibre la somme de ces forces
est nulle. On retrouve donc \eqref{eq:eq_flux}.


De plus, la force s'exer\c{c}ant en $x$ \`a travers la section de la barre est 
proportionnelle \`a l'\'elongation de la barre et s'oppose au mouvement impos\'e.
Ceci est intuitif, pensez \`a un \'elastique si vous l'allongez il s'exerce une force
qui tend \`a le faire revenir vers sa position initiale. De plus, plus l'\'elongation
est importante, plus l'intensit\'e de la force est grande.
On obtient donc la loi d'\'elasticit\'e \eqref{eq:def_flux}
o\`u $k$ est un coefficient de raideur:
plus $k$ est grand, plus la barre est raide, plus il faut forcer pour la d\'eformer.
En pratique, le coefficient de raideur d\'epend du mat\'eriau choisi et
de la g\'eom\'etrie de la section de la barre.

Notons que dans \eqref{eq:def_flux}, la d\'eriv\'ee en espace correspond \`a 
l'\'elongation de la barre en $x$. Pour s'en convaincre, on regardera la 
partie droite de la figure \ref{fig:barre}. Un \'el\'ement de mati\`ere de longueur
$\delta x$ dans sa position de r\'ef\'erence a pour longueur 
$x + \delta x + u(x+\delta x) - u(x) - x$ sous charge $f$.
La nouvelle longueur est donc de $\delta x + u(x+\delta x) - u(x) 
\simeq \left(1 + \dfrac{\partial u}{\partial x} \right) \delta x$
et la d\'eriv\'ee partielle en $x$ est donc bien une \'elongation lin\'eique.


!! CONDITIONS FRONTIERES !!
La partie hachur\'ee \`a gauche du dessin repr\'esente le fait que la barre est encastr\'ee
dans un mur. Ainsi son d\'eplacement est forc\'ement nul \`a gauche.


!! autres exemples de cette physique : electromagnetisme !!

!! solution pour diff\'erents $k$ !!

!! EDP elliptique !! A faire en exo !! -- A exprimer sous la forme d'une proposition !!

!! dire clairement ce qui est exigible !!

!! dire ce que sont les equations de Poisson et de Laplace !!
%%%%%%%%%%%%%%%%%%%%%%%%%%%%%%%%%%%%%%%
\subsection{\'Equation de transport}

La deuxi\`eme \'equation que nous \'etudions est l'\'equation de transport
(unidimensionelle).
Elle correspond \`a une quantit\'e qui est transport\'ee \`a vitesse
constante dans une direction.
Cela peut \^etre par exemple un polluant transport\'e par une rivi\`ere.
Dans cette section, nous nous int\'eressons au cas de tas de sable
sur un tapis roulant se d\'epla\c{c}ant \`a vitesse constante $a$.


\begin{figure}
\centering
\begin{tikzpicture}[scale = 0.5]
  %% longueur du tapis roulant
\def\L{15};

%% tapis roulant
\draw (1,0) arc (0:360:1);
\draw (0,1.1) arc (90:270:1.1);
\draw (\L+1.0,0) arc (0:360:1);
\draw (\L,1.1) arc (90:-90:1.1);
\draw (0,1.1) -- (\L,1.1);
\draw (0,-1.1) -- (\L,-1.1);

%% tas de sables (pointilles)
\draw[dashed] (1,1.1) -- (5,3.5) -- (10,2.3) -- (15,5.5);

%% distance parcourue
\def\a{3}

%% tas de sables (trait continu)
\draw (1+\a,1.1) -- (5+\a,3.5) -- (10+\a,2.3) -- (15+\a,5.5);

%% vecteur a
\draw[->] (5,3.9) -- (5+\a,3.9);
\draw (5+\a/2,4.5) node{$a$};

%% hauteur u
\draw[<->] (14,1.2) -- (14,2.8);
\draw (15.3,2.0) node{$u(t,x)$};

%% axe des abscisses
\draw[->] (-1.5,0.0) -- (19.0,0.0);
\draw (19.0,-1.0) node{$x$};

\end{tikzpicture}
\caption{Des tas de sable transport\'es sur un tapis roulant.
  La ligne discontinue repr\'esente les tas en $t=0s$ et la ligne
  continue en $t=1s$.}
\label{fig:transport}
\end{figure}

Sur la figure \ref{fig:transport}, nous repr\'esentons des tas de sables
qui se d\'eplacent \`a vitesse constante $a$ vers la droite.
On note $u(t,x)$ la hauteur du sable en $x$ \`a l'instant $t$.

Si l'on consid\`ere un temps infinit\'esimal $\delta t$,
le sable aura avanc\'e d'une distance $\delta x = a \delta t$.
La hauteur en $(t+\delta t , x + \delta x)$ sera la m\^eme qu'elle
\'etait en $(t,x)$, on traduit cela par $u(t+\delta t, x + a \delta t) = u(t,x)$.
On obtient ainsi l'\'equation
\begin{align}
  \dfrac{\partial u}{\partial t} + a \dfrac{\partial u}{\partial x} = 0 .
\end{align}
Ici, $a$ correspond \`a la vitesse du transport.
Si $a$ est positif, le sable bouge vers la droite; si $a$ est n\'egatif,
le sable bouge vers la gauche; plus $a$ est grand en valeur absolue, 
plus le mouvement est rapide.



%%%%%%%%%%%%%%%%%%%%%%%%%%%%%%%%%%%%%%%
\subsection{\'Equation de la chaleur}

L'\'equation de la chaleur est obtenue en consid\'erant une propagation 
de quantit\'es d'\'energie comme sur la figure \ref{fig:flux}.
La diff\'erence avec ce qui a \'et\'e fait dans la section \ref{subsec:elliptique}
est qu'ici les flux de chaleur ne sont pas n\'ecessairement \`a l'\'equilibre.
On permet donc \`a la temp\'erature de changer au cours du temps.

Nous allons maintenant refaire le raisonnement de la section \ref{subsec:elliptique}
avec cette fois-ci une temp\'erature $u(t,x)$ qui d\'epend du temps.
Si on consid\`ere la partie droite de la figure \ref{fig:flux}, l'accumulation
d'\'energie en $(t,x)$ est due au fait que les flux ne sont pas \'equilibr\'es
("ce qui entre n'est pas \'egal \`a ce qui sort").
Nous traduisons cela par l'\'equation
$c (x) \delta x \dfrac{\partial u}{\partial t}(t,x) = f(t,x) \delta x 
  + q(t,x) - q(t,x + \delta x)$, ce qui donne
\begin{align}
  \label{eq:flux_t}
  c (x) \dfrac{\partial u}{\partial t}(t,x) = f(t,x) - \dfrac{\partial q}{\partial x} (t,x) .
\end{align}
o\`u $c(x)$ est la capacit\'e thermique lin\'eique du mat\'eriau,
$q(t,x)$ et $f(t,x)$ sont respectivement le flux de chaleur 
et la source de chaleur en $(t,x)$.
Comme pr\'ec\'edemment, le flux de chaleur est proportionnel au gradient 
de temp\'erature (voir \eqref{eq:def_flux}).
Pour simplifier la pr\'esentation, nous consid\'erons que la capacit\'e
thermique $c$ et la conduction thermique $k$ sont des constantes 
(qui ne d\'ependent ni du temps, ni de l'espace).
En combinant les \'equations \eqref{eq:def_flux} et \eqref{eq:flux_t},
on obtient l'\'equation de la chaleur
\begin{align}
  \dfrac{\partial u}{\partial t} - \nu \dfrac{\partial^2 u}{\partial x^2} = f ,
\end{align}
o\`u $\nu = k/c$ est la diffusion thermique.


!! courbes avec $\nu$ qui varie !!

!! exercice : prouver qu'il s'agit d'une equation parabolique !!

De mani\`ere similaire \`a ce qui a \'et\'e expliqu\'e dans la section \ref{subsec:elliptique},
cette l'\'equation de la chaleur, en plus de repr\'esenter l'\'evolution de la temp\'erature
dans un milieu peut aussi repr\'esenter l'\'evolution de la densit\'e d'une population
ou encore l'\'evolution d'une concentration chimique d'une esp\`ece au sein d'un diluant.


!! rappel sur les conditions fronti\`eres !!

%%%%%%%%%%%%%%%%%%%%%%%%%%%%%%%%%%%%%%%
\subsection{\'Equation des ondes}

L'\'equation des ondes est obtenue en consid\'erant un mod\`ele m\'ecanique
comme celui de la figure \ref{fig:barre} o\`u cette fois-ci les forces ne
sont pas n\'ecessairement \`a l'\'equilibre et o\`u les d\'eplacements
$u(t,x)$ peuvent varier au cours du temps.

La principe fondamental de la dynamique appliqu\'e \`a une tranche de mati\`ere
de longueur $\delta x$ nous dit que l'acc\'el\'eration de la position de cette tranche
($x+u(t,x)$) multipli\'ee par sa masse est \'egale \`a la somme des forces qui
agissent sur elle.
Ainsi, $\rho(x) \delta x \dfrac{\partial^2 u}{\partial t^2} = 
f(t,x) \delta x + q(t,x) - q(t, x + \delta x)$, ce qui nous donne l'\'equation
\begin{align}
  \label{eq:barre_t}
  \rho(x) \dfrac{\partial^2 u}{\partial t^2} + \dfrac{\partial q}{\partial x}(t,x) = f(t,x) ,
\end{align}
o\`u $\rho(x)$ correspond \`a la masse de la barre par unit\'e de longueur.
Plus $\rho$ est grand, plus la barre a d'inertie et plus elle acc\'el\`ere lentement.
Notons \`a ce stade qu'il y a, dans cette \'equation, une d\'eriv\'ee seconde en temps
\`a la place de la d\'eriv\'ee premi\`ere qu'il y avait dans \eqref{eq:flux_t}.

La force de compression \`a travers la barre est donn\'ee par la loi d'\'elasticit\'e
\eqref{eq:def_flux}.
Pour simplifier la pr\'esentation, on consid\`ere que la raideur $k$ et la masse
lin\'eique $\rho$ sont constantes (ne d\'ependent ni de $t$ ni de $x$).
En combinant \eqref{eq:def_flux} et \eqref{eq:barre_t} on obtient l'\'equation des ondes
\begin{align}
  \label{eq:ondes}
  \dfrac{\partial^2 u}{\partial t^2} - c^2 \dfrac{\partial^2 u}{\partial x^2} = f ,
\end{align}
o\`u $c>0$ d\'efini par $c^2 = k/\rho$ correspond \`a la vitesse de propagation
des ondes dans ce milieu.

!! differentes valeurs de $c$ -> utiliser des valeurs reelles (eau/air) !!

!! prouver que l'\'equation est hyperbolique !!

!! autres physiques : electromagnetisme, corde !!

!! rappel sur les conditions fronti\`eres !!


%%%%%%%%%%%%%%%%%%%%%%%%%%%%%%%%%%%%%%%%%%%%%%%%%%%%%%%%%%%%%%%%%%%%%%%%%%%%%%%%%%%%
\section{\'Equations elliptiques}

Comme \'evoqu\'e pr\'ec\'edemment, on s'int\'eresse au probl\`eme
\begin{equation}
  \label{eq:ell_Dir}
  \left\{
    \begin{array}{l}
      - \DIV (k \GRAD u) = f \dans \Omega ,
      \\
      u = 0 \sur \partial \Omega ,
    \end{array}
  \right.
\end{equation}
avec $0 < k_0 \leq k(x) \leq k_1$.

%%%%%%%%%%%%%%%%%%%%%%%%%%%%%%%%%%%%%%%
\subsection{Propri\'et\'es g\'en\'erales de l'\'equation}

\begin{theorem}[Existence et unicit\'e de la solution]
  \label{thm:ell_existence}
  Si $f \in C^0(\Omega)$, alors il existe une unique solution
  $u \in C^2(\Omega)$ au probl\`eme \eqref{eq:ell_Dir}.
  De plus, pour $\ell \in \N$, si $f \in C^{\ell}(\Omega)$, alors
  $u \in C^{\ell+2}(\Omega)$.
\end{theorem}

!! commenter + donner lien vers preuve !!

\begin{proposition}[Principe du maximum]
  On se place dans le cadre du th\'eor\`eme \ref{thm:ell_existence}.
  Si $f \leq 0$, alors $u \leq 0$. En particulier,
  $\max_{x\in \Omega} u(x) = \max_{x\in \partial \Omega} u(x) = 0$.
  Si de plus il existe $m\in \Omega$ tel que $u(m) = 0$,
  alors $\forall x \in \Omega$, $u(x) = 0$ 
  (principe du maximum fort).
\end{proposition}

!! commenter + ref livre !!

\begin{remark}
  D'autres conditions fronti\`ere sont possibles.
  Condition de Neumann, de Robin, conditions mixtes, conditions de Dirichlet non homog\`enes.
\end{remark}

!! expliquer le principe d'un rel\`evement !!

\begin{definition}[Fonction harmonique]
  Si $u \in C^2(\Omega)$ et $- \Delta u = 0$ dans $\Omega$,
  on dit que $u$ est une fonction harmonique.
\end{definition}

!! lien avec les fonctions holomorphes !!

%%%%%%%%%%%%%%%%%%%%%%%%%%%%%%%%%%%%%%%
\subsection{Formulation variationnelle, th\'eor\`eme de Lax--Milgram}

Le but de cette section est de pr\'esenter une preuve du th\'eor\`eme \ref{thm:ell_existence}
en utilisant le th\'eor\`eme de Lax--Milgram.

\begin{theorem}[Lax--Milgram]
  \label{thm:Lax-Milgram}
  On fait les hypoth\`eses suivantes.
  \begin{itemize}
  \item Soit $V$ un espace de Hilbert.
  \item Soit $\ell$ une forme lin\'eaire continue sur $V$
    (il existe $C_{\ell} > 0$ tel que $\forall v \in V, |\ell(v)| \leq C_{\ell} \| v \|_{V}$).
  \item Soit $a$ une forme bilin\'eaire continue sur $V$
    (il existe $C_{a} > 0$ tel que $\forall v,w \in V, |a(v,w)| \leq C_{a} \| v \|_{V} \| w \|_{V}$).
  \item On suppose de plus que $a$ est coercive:
    il existe $\alpha > 0$ tel que $\forall v \in V, a(v,v) \geq \alpha \| v \|_{V}^2$.
  \end{itemize}
  Sous ces hypoth\`eses le probl\`eme suivant est bien pos\'e:
  \begin{equation}
    \label{eq:Lax-Milgram}
    \text{Trouver $u \in V$, tel que $\forall v \in V$, } a(u,v) = \ell(v) .
  \end{equation}
  Ceci signifie que le probl\`eme \eqref{eq:Lax-Milgram} admet une unique solution $u \in V$
  et que celle-ci v\'erifie $\| u \|_{V} \leq \dfrac{C_{\ell}}{\alpha}$.
\end{theorem}

!! lien vers un livre !!

Pour utiliser ce th\'eor\`eme, \'ecrivons le probl\`eme \eqref{eq:ell_Dir} sous
sa forme variationnelle.
On introduit l'espace de Sobolev
\begin{align}
  H_0^1(\Omega) := \{ v \in L^2(\Omega) \st \GRAD v \in L^2(\Omega) 
  \text{ et } v_{| \partial \Omega} = 0 \sur \partial \Omega \} ,
\end{align}
o\`u $\GRAD v$ est d\'efini au sens des distributions et $v_{\partial \Omega}$
est la trace de $v$ sur le bord du domaine.


Si $u$ est une solution de \eqref{eq:ell_Dir}, alors pour tout $v \in H^1_0(\Omega)$
on peut \'ecrire
\begin{align*}
  \int_{\Omega} - \DIV (k \GRAD u) v \deriv x = \int_{\Omega} f v \deriv x .
\end{align*}
En int\'egrant par parties on obtient 
\begin{align*}
  \int_{\Omega} k \GRAD u : \GRAD v \deriv x 
  - \int_{\partial \Omega} k (\GRAD u \SCAL n) v \deriv s = \int_{\Omega} f v \deriv x .
\end{align*}
Puisque $v_{\partial \Omega} = 0$, le deuxi\`eme terme de cette expression est nul.
Nous avons donc \'etabli la proposition suivante 
(la r\'eciproque utilise des arguments similaires).
\begin{proposition}
  Le probl\`eme \eqref{eq:ell_Dir} est \'equivalent au probl\`eme suivant.
  \begin{equation}
    \label{eq:LM2}
    \text{Trouver $u \in H_0^1(\Omega)$, tel que $\forall v \in H_0^1(\Omega)$, } 
    \int_{\Omega} k \GRAD v : \GRAD v \deriv x = \int_{\Omega} f v \deriv x .
  \end{equation}
\end{proposition}

\begin{proposition}
  Le probl\`eme \eqref{eq:LM2} est bien pos\'e.
\end{proposition}
\begin{proof}
  Le probl\`eme \eqref{eq:LM2} correspond au probl\`eme \eqref{eq:Lax-Milgram}
  avec $V = H_0^1(\Omega)$, 
  $\forall v, w \in H_0^1(\Omega), a(v,w) = \int_{\Omega} k \GRAD v : \GRAD v \deriv x$
  et $\ell(v) = \int_{\Omega} f v \deriv x$.
  Nous allons montrer que toutes les hypoth\`eses du th\'eor\`eme de Lax--Milgram
  sont v\'erifi\'ees.
  On admet le fait que $H_0^1(\Omega)$ \'equip\'e de la norme 
  $\| v \|_{H_0^1(\Omega)} := (\int_{\Omega} v^2 + \GRAD v : \GRAD v \deriv x)^{1/2}$ 
  est un espace de Hilbert (se reporter \`a un cours sur les distributions).
  D'apr\`es l'in\'egalit\'e de Cauchy--Schwarz, on a 
  $|\ell(v)| \leq (\int_{\Omega} f^2 \deriv x)^{1/2} (\int_{\Omega} v^2 \deriv x)^{1/2}
  \leq (\int_{\Omega} f^2 \deriv x)^{1/2} \| v \|_{H_0^1(\Omega)}$.
  La forme lin\'eaire $\ell$ est donc continue. 
  De la m\^eme fa\c{c}on, l'in\'egalit\'e de Cauchy--Schwarz permet de prouver
  que la forme bilin\'eaire $a$ est continue.
  Pour finir, nous utilisons l'in\'egalit\'e de Poincar\'e: il existe $c>0$
  tel que $\forall v \in H_0^1(\Omega), 
  \int_{\Omega} v^2 \deriv x \leq c \int_{\Omega} \GRAD v : \GRAD v \deriv x$.
  Avec cette in\'egalit\'e, on peut prouver que $a$ est coercive.
  Toutes les hypoth\`eses du th\'eor\`eme de Lax--Milgram sont r\'eunies,
  le probl\`eme \eqref{eq:LM2} est donc bien pos\'e.
\end{proof}



%%%%%%%%%%%%%%%%%%%%%%%%%%%%%%%%%%%%%%%
\subsection{Discr\'etisation par la m\'ethode des diff\'erences finies}
\label{subsec:ell_DF}

Le but de la m\'ethode des diff\'erences finies est d'approcher la solution 
d'une EDO (\'equation aux d\'eriv\'ees ordinaires) ou EDP
(\'equation aux d\'eriv\'ees partielles) par des valeurs cens\'ees repr\'esenter
cette fonction en certains points. Dans toutes les discr\'etisations abord\'ees dans ce cours,
nous ne consid\'ererons que le cas de probl\`emes \`a une dimension en espace
(les probl\`emes instationnaires auront une deuxi\`eme dimension correspondant au temps).

Par exemple, int\'eressons nous au probl\`eme \eqref{eq:ell_Dir} avec $\Omega = (0,1)$
et $k = 1$.
Notre probl\`eme est donc
\begin{equation}
  \label{eq:ell_1d}
  \left\{
    \begin{array}{l}
      - \dfrac{\!\deriv^2 u}{\!\deriv x^2} = f \dans (0,1) ,
      \\
      u(0) = u(1) = 0 .
    \end{array}
  \right.
\end{equation}

Nous discr\'etisons l'intervalle $(0,1)$ en $N>0$ sous-intervalles.
On d\'efinit donc les points $x_j = j h$ ($0 \leq j \leq N$) o\`u $h = 1/N$.
On a bien $x_0 = 0$ et $x_N = 1$ (voir figure \ref{fig:disc_x}).

\begin{figure}
\centering
\begin{tikzpicture}[scale=1]
\newcounter{it}
% intervalle
\draw (0,0) -- (10,0);
\forloop{it}{0}{\value{it} < 11}{
\draw (\arabic{it},-0.3) -- (\arabic{it},0.3);
}

% temps
\draw (0,-0.6) node {$0$};
\draw (10,-0.6) node {$1$};
\draw (0,0.6) node {$x_0$};
\draw (1,0.6) node {$x_1$};
\draw (10,0.6) node {$x_N$};
\draw (5,0.6) node {$.......$};
\end{tikzpicture} 
\caption{Repr\'esentation de la discr\'etisation de l'intervalle $(0,1)$.}
\label{fig:disc_x}
\end{figure}


\'Etant donn\'e que nous ne disposons que de valeurs en certains points, on ne peut
pas d\'efinir de d\'eriv\'ees au sens usuel. On utilise donc des d\'eriv\'ees discr\`etes.
Pour cela, on rappelle qu'une d\'eriv\'ee est la limite d'un taux d'accroissement.
Par exemple, on peut montrer que 
$\dfrac{\!\deriv^2 u}{\!\deriv x^2}(x) 
= \lim\limits_{h \to 0} \dfrac{u(x+h) - 2 u(x) + u(x-h)}{h^2}$.
Avec les notations introduites pr\'ec\'edemment, cela signifie que,
pour $h$ suffisamment petit, 
\begin{align}
  \label{eq:dx2}
  \dfrac{\!\deriv^2 u}{\!\deriv x^2}(x_j) \simeq \dfrac{u(x_{j+1}) - 2 u(x_j) + u(x_{j-1})}{h^2} .
\end{align}
La m\'ethode des diff\'erences finies consiste donc \`a d\'efinir une suite $(u_j)_{0\leq j \leq N}$
qui reprenne les \'el\'ements du probl\`eme \eqref{eq:ell_1d}. En particulier, il faut remplacer
les d\'eriv\'ees continues par des d\'eriv\'ees discr\`etes.
Il existe plusieurs fa\c{c}ons de d\'efinir la suite $(u_j)_{0\leq j \leq N}$ et toutes
n'ont pas les m\^emes propri\'et\'es. La propri\'et\'e la plus importante est la convergence:
on veut que $\lim\limits_{N \to +\infty} \max_{0 \leq j \leq N} | u(x_j) - u_j | = 0$.
Nous aborderons plus en d\'etails ces aspects dans la section \ref{subsec:transport_DF}.


Nous proposons maintenant de discr\'etiser le prob\`eme \eqref{eq:ell_1d}
en utilisant \eqref{eq:dx2}.
La suite ainsi obtenue v\'erifie
\begin{equation}
  \label{eq:ell_DF}
  \left\{
    \begin{array}{l}
      \forall 1 \leq j \leq N-1 , - \dfrac{u_{j+1} - 2 u_j + u_{j-1}}{h^2} = f(x_j) ,
      \\
      u_0 = u_N = 0 .
    \end{array}
  \right.
\end{equation}

\begin{proposition}
  La relation \eqref{eq:ell_DF} d\'efinit une unique suite $(u_j)_{0\leq j \leq N}$ dont les valeurs
  peuvent \^etre d\'etermin\'ees en r\'esolvant le syst\`eme lin\'eaire
  $A U = F$, avec 
  \begin{equation}
    \label{eq:mat_DF_Laplacien}
    A = \dfrac{1}{h^2} 
    \begin{pmatrix} 
      2 & -1 & 0 & \cdots
      \\
      -1 & 2 & -1 & 0 & \cdots
      \\
      0 & -1 & 2 & \ddots &
      \\
      \vdots & & \ddots & \ddots & \ddots
      \\
      & & &  \ddots & 2 & -1
      \\
      & &  & & -1 & 2
    \end{pmatrix} , \qquad U = 
    \begin{pmatrix}
      u_1 \\ u_2 \\ \vdots \\ u_{N-1}
    \end{pmatrix} , \qquad F = 
    \begin{pmatrix}
      f(x_1) \\ f(x_2) \\ \vdots \\ f(x_{N-1})
    \end{pmatrix} .
  \end{equation}
\end{proposition}

\begin{proof}
  La relation \eqref{eq:ell_DF} est \'equivalente au syst\`eme lin\'eaire $AU = F$.
  Pour prouver cela, on \'ecrit les \'equations discr\`etes
  \begin{align*}
    &\dfrac{-u_0 + 2 u_1 - u_2}{h^2} = f(x_1) ,
    \\
    &\dfrac{-u_1 + 2 u_2 - u_3}{h^2} = f(x_2) ,
    \\
    &\dfrac{-u_2 + 2 u_3 - u_4}{h^2} = f(x_3) ,
    \\
    & \qquad \qquad \vdots
    \\
    &\dfrac{-u_{N-2} + 2 u_{N-1} - u_N}{h^2} = f(x_{N-1}) .
  \end{align*}
  En prenant en compte $u_0 = u_N = 0$, on obtient bien $AU = F$.
  La r\'eciproque se prouve de mani\`ere similaire.

  Maintenant, le fait que ce syst\`eme admet une unique solution vient de l'inversibilit\'e
  de la matrice $A$ qui est une cons\'equence de la proposition \ref{prop:matSDP}.
\end{proof}


\begin{remark}
  La matrice $A$ doit absolument \^etre connue par c\oe{}ur.
  En effet, ceci est exigible et le jury du concours se plaint dans ses rapports
  que beaucoup trop de candidats ne connaissent pas cette matrice.
\end{remark}

\begin{proposition}
  \label{prop:matSDP}
  La matrice $A$ est sym\'etrique d\'efinie positive.
\end{proposition}


!! Ajouter resultats de simulations !!

!! Ajouter exercice avec le probl\`eme de Neumann !!

!! La preuve des caracteristiques de la matrice $A$ en exercice ??

!! trouver une solution pour prouver la convergence de ce sch\'ema sans
introduire stabilit\'e et consistance !!


!! exercice : prendre $k$ constant et faire des simus avec ---- retrouver les 
r\'esultats de simu de la section 1 !!

%%%%%%%%%%%%%%%%%%%%%%%%%%%%%%%%%%%%%%%%%%%%%%%%%%%%%%%%%%%%%%%%%%%%%%%%%%%%%%%%%%%%
\section{\'Equation de transport}

Nous \'etudions ici l'\'equation de transport dans un espace \`a une dimension.
Le probl\`eme d\'ependra donc d'une variable d'espace $x \in [0,1]$ et d'une variable de temps
$t \in [0,T]$.
Nous consid\'erons le probl\`eme
\begin{equation}
  \label{eq:transport}
  \left\{
    \begin{array}{l}
     \dfrac{\partial u}{\partial t} + a \dfrac{\partial u}{\partial x} = 0 
      \dans (0,T) \times (0,1),
      \\
      \forall t \in (0,T), u(t,0) = \alpha(t) ,
      \\
      \forall x \in (0,1), u(0,x) = u_0(x) ,
    \end{array}
  \right.
\end{equation}
o\`u $a \geq 0$ correspond \`a la vitesse de transport de l'\'equation,
$\alpha(t)$ est la donn\'ee de Dirichlet \`a gauche
et $u_0$ est la donn\'ee initiale.
%%%%%%%%%%%%%%%%%%%%%%%%%%%%%%%%%%%%%%%
\subsection{Propri\'et\'es g\'en\'erales}

Le probl\`eme \eqref{eq:transport} admet une unique solution $u$.
Cette solution est obtenue en "d\'epla\c{c}ant" la donn\'ee initiale
vers la droite et en "faisant entrer" la donn\'ee de Dirichlet dans le domaine.
De mani\`ere plus rigoureuse, nous avons le th\'eor\`eme suivant.

\begin{theorem}
  Le probl\`eme \eqref{eq:transport} admet une unique solution donn\'ee par
  (...)
\end{theorem}

\begin{remark}
  On n'a besoin d'une donn\'ee de Dirichlet que d'un seul c\^ot\'e du domaine.
  Puisque l'on a choisi $a \geq 0$, il faut fixer la donn\'ee de Dirichlet \`a gauche.
  On aurait aussi pu prendre $a \leq 0$ et utiliser la condition de Dirichlet
  $u(t,1) = \alpha(t)$.
\end{remark}


Nous allons maintenant \'evoquer une autre propri\'et\'e int\'eressante de l'\'equation
de transport: la r\'eversibilit\'e de l'\'equation.
Cela signifie qu'en utilisant la solution au temps final, on peut reconstituer la solution
initiale.
Dans le cas pr\'esent, en utilisant la solution au temps final et la solution en sortie 
du domaine ($x=1$), on peut reconstituer la valeur de la solution \`a l'instant initial
ainsi que la valeur de la donn\'ee de Dirichlet $\alpha(t)$ en tout temps $t \in [0,T]$.

\begin{proposition}[R\'eversibilit\'e de l'\'equation de transport]
  (...)
\end{proposition}

Cette propri\'et\'e signifie que l'information est conserv\'ee au cours du temps:
le comportement de l'\'equation ne d\'egrade pas cette information.


%%%%%%%%%%%%%%%%%%%%%%%%%%%%%%%%%%%%%%%
\subsection{M\'ethode des caract\'eristiques}

La m\'ethode des caract\'eristiques consiste \`a montrer que la solution
se conserve sur certaines courbes que l'on appelle trajectoires caract\'eristiques.
Dans le cas monodimensionnel avec $a$ constant, ces courbes sont des droites.
On parlera donc de droites caract\'eristiques.
Cependant, dans le cas o\`u plusieurs dimensions d'espace sont consid\'er\'ees,
ces courbes peuvent \^etre plus complexes (voir la section \ref{subsec:transport_multiD}).

\begin{figure}
\centering
\begin{tikzpicture}[scale = 9]
  %% temps final
\def\T{0.9};

%% repere
\draw[->] (-0.05,0) -- (1.1,0);
\draw (1.1,-0.05) node{$x$};
\draw[->] (0,0.0) -- (0,\T+0.1);
\draw (-0.05,\T+0.1) node{$t$};

%% domaine
\draw[line width=1mm] (0,0) -- (1,0);
\draw (1,-0.05) node{$1$};
\draw (0.0,-0.05) node{$0$};
\draw[line width=1mm] (0,0) -- (0,\T);
\draw (-0.05, \T) node{$T$};
\draw[dashed] (1,0) -- (1,\T);
\draw[dashed] (0,\T) -- (1,\T);

%% courbes caract\'eristiques
\draw (0.75,0) -- (1,0.25);
\draw (0.5,0) -- (1,0.5);
\draw (0.25,0) -- (1,0.75);
\draw (0,0) -- (\T,\T);
\draw (0,0.25) -- (\T-0.25,\T);
\draw (0,0.5) -- (\T-0.5,\T);
\draw (0,0.75) -- (\T-0.75,\T);

%% ronds
\def\r{0.02};
\draw (0.75+\r,0) arc(0:360:\r);
\draw (0.5+\r,0) arc(0:360:\r);
\draw (0.25+\r,0) arc(0:360:\r);
\draw (\r,0) arc(0:360:\r);
\draw (\r,0.25) arc(0:360:\r);
\draw (\r,0.5) arc(0:360:\r);
\draw (\r,0.75) arc(0:360:\r);

\end{tikzpicture}
\caption{Les droites caract\'eristiques de l'\'equation de transport
  ($a=1$ et $T=0.9$).}
\label{fig:D_carac}
\end{figure}

Nous pouvons montrer que dans le cas consid\'er\'e dans cette section,
la solution se conserve sur les droites d'\'equation $t=ax+b$ avec $b \in \R$.
Ces droites sont donc les droites caract\'eristiques de notre probl\`eme,
nous les repr\'esentons sur la figure \ref{fig:D_carac}.

!! equation des droites a montrer en exo !!

Nous voyons donc que la valeur de la solution en $(t,x)$ est \`a aller chercher
sur le bord du domaine espace-temps. C'est-\`a-dire, en fonction de la valeur de 
$t$ et $x$, soit sur la condition initiale, soit sur la donn\'ee de Dirichlet
(l'endroit o\`u r\'ecup\'erer l'information est repr\'esent\'e sur la figure
\ref{fig:D_carac} par un cercle).
Nous retrouvons donc l'expression de la solution donn\'ee en (...).


!! preuve de l'unicit\'e ??

Pour montrer la r\'eversibilit\'e de l'\'equation, il suffit de parcourir ces droites 
caract\'eristiques dans l'autre sens (...)

%%%%%%%%%%%%%%%%%%%%%%%%%%%%%%%%%%%%%%%
\subsection{Discr\'etisation par les diff\'erences finies et analyse num\'erique}
\label{subsec:transport_DF}

Nous cherchons maintenant \`a discr\'etiser le probl\`eme \eqref{eq:transport}
par la m\'ethode des diff\'erences finies.
Dans la section \ref{subsec:ell_DF}, pour approcher l'\'equation de Poisson,
nous avons discr\'etis\'e l'espace $(0,1)$ et nous avons approch\'e la solution
$u(x)$ par une suite de terme g\'en\'eral $u_j$.

Dans le cas pr\'esent, la solution d\'epend de deux variables $t$ et $x$,
il faut donc discr\'etiser les deux dimensions.
On d\'ecoupe l'intervalle de temps $[0,T]$ en $N$ sous-intervalles $[t_n , t_{n+1}]$
avec pour tout $0\leq n \leq N$,  $t_n = n h_t$ et $h_t = T/N$.
De m\^eme, on d\'ecoupe l'intervalle d'espace $[0,1]$ en $M$ sous-intervalles
$[x_j , x_{j+1}]$ avec pour tout $0 \leq j \leq M$, $x_j = j h_x$ et $h_x = 1/M$. 
De plus, nous allons approcher la fonction $u(t,x)$ par une suite
$(u_j^n)_{\substack{0 \leq n \leq N \\ 0 \leq j \leq M}}$.
Ici, l'indice $j$ donne la position en espace et l'exposant $n$
donne le temps consid\'er\'e. Ainsi, $u_j^n$ devra \^etre une approximation
de $u(t_n, x_j)$.

%%%%%%%%%%%%%%%%%%
\subsubsection{Conditions p\'eriodiques et motivation de l'analyse}

Comme nous l'avons vu pr\'ec\'edemment, lorsque l'on consid\`ere
l'\'equation de transport avec une condition de Dirichlet,
l'information de la donn\'ee initiale sort progressivement du domaine
en \'etant remplac\'ee par la donn\'ee de Dirichlet.
On pourrait vouloir s'int\'eresser au traitement sur le temps long de 
l'information issue de la donn\'ee initiale.
Pour cela, on peut essayer de suivre le d\'eplacement de cette donn\'ee
initiale dans un domaine infini en consid\'erant des conditions p\'eriodiques.
Dans cette section, on s'int\'eresse au probl\`eme
\begin{equation}
  \label{eq:transport_per}
  \left\{
    \begin{array}{l}
     \dfrac{\partial u}{\partial t} + a \dfrac{\partial u}{\partial x} = 0 
      \dans (0,T) \times (0,1),
      \\
      \forall t \in (0,T), u(t,0) = u(t,1) ,
      \\
      \forall x \in (0,1), u(0,x) = u_0(x) ,
    \end{array}
  \right.
\end{equation}
avec $a>0$ et $u_0$ v\'erifie les conditions p\'eriodiques $u_0(0) = u_0(1)$.
Avec ces conditions fronti\`eres, l'information qui sort en 
$x=1$ r\'eentre imm\'ediatement en $x=0$.


En appliquant la m\'ethode des caract\'eristiques, on peut montrer
que la solution exacte de ce probl\`eme est
\begin{align}
  \label{eq:sol_transport_per}
  u(t,x) = u_0(f(x-at)) ,
\end{align}
o\`u $f(x)$ est la partie fractionnaire de $x$,
c'est-\`a-dire $x = E(x) + f(x)$ avec $E(x) \in \Z$
et $0 \leq f(x) < 1$.


!! Calcul de la solution du pb p\'eriodique avec m\'ethode des caract\'eristiques en exo ??

Nous souhaitons maintenant discr\'etiser l'\'equation
\eqref{eq:transport_per}
et comparer la solution obtenue num\'eriquement avec \eqref{eq:sol_transport_per}.
La m\'ethode consiste \`a approcher les d\'eriv\'ees partielles
(en temps et en espace) par des taux d'accroissement.
Nous proposons de comparer le comportement de deux discr\'etisations diff\'erentes.
Nous proposons tout d'abord d'approcher la d\'eriv\'ee en temps par une
approximation d\'ecentr\'ee et la d\'eriv\'ee en espace par une 
approximation centr\'ee comme suit
\begin{align*}
  \dfrac{\partial u}{\partial t}(t_n,x_j) \simeq \dfrac{u(t_{n+1},x_j) - u(t_n,x_j)}{h_t} ,
  \qquad \qquad 
  \dfrac{\partial u}{\partial x}(t_n,x_j) \simeq \dfrac{u(t_n,x_{j+1}) - u(t_n,x_{j-1})}{2 h_x} .
\end{align*}
On obtient le sch\'ema num\'erique
\begin{align}
  \label{eq:transport_DF_centre}
  \left\{
  \begin{array}{l}
    \forall 0 \leq j \leq M , 
    u_j^0 = u_0(x_j) ,
    \\
    \forall 1 \leq n \leq N, u_0^n = u_M^n ,
    \\
    \forall 1 \leq n \leq N, \forall 0 \leq j \leq M,
    u_j^n = u_j^{n-1} - \dfrac{a h_t}{2 h_x} (u_{j+1}^{n-1} - u_{j-1}^{n-1}) ,
  \end{array}
  \right.
\end{align}
o\`u nous consid\'erons $u_0^n = u_M^n$, $u_{M+1}^n = u_1^n$ et $u_{-1}^n = u_{M-1}^n$.
%En consid\'erant comme degr\'es de libert\'e $U^n = u_j^n$ pour $1 \leq n \leq N$,

Nous nous proposons \'egalement d'approcher les d\'eriv\'ees en temps et les d\'eriv\'ees
en espace respectivement par des approximations d\'ecentr\'ees aval et amont comme suit
\begin{align}
  \label{eq:der_decentre_AM}
  \dfrac{\partial u}{\partial t}(t_n,x_j) \simeq \dfrac{u(t_{n+1},x_j) - u(t_n,x_j)}{h_t} ,
  \qquad \qquad 
  \dfrac{\partial u}{\partial x}(t_n,x_j) \simeq \dfrac{u(t_n,x_{j}) - u(t_n,x_{j-1})}{h_x} .
\end{align}
On obtient le sch\'ema num\'erique
\begin{align}
  \label{eq:transport_DF_decentre_AM}
  \left\{
  \begin{array}{l}
    \forall 0 \leq j \leq M , 
    u_j^0 = u_0(x_j) ,
    \\
    \forall 1 \leq n \leq N, u_0^n = u_M^n ,
    \\
    \forall 1 \leq n \leq N, \forall 0 \leq j \leq M,
    u_j^n = u_j^{n-1} - \dfrac{a h_t}{h_x} (u_{j}^{n-1} - u_{j-1}^{n-1}) ,
  \end{array}
  \right.
\end{align}
o\`u nous consid\'erons $u_0^n = u_M^n$ et $u_{-1}^n = u_{M-1}^n$.

Les r\'esultats num\'eriques sont report\'es dans la
figure \ref{fig:trans_per}.
Nous observons des comportements num\'eriques tr\`es diff\'erents pour ces deux
sch\'emas qui sont pourtant similaires.
Il ne suffit donc pas de remplacer les d\'eriv\'ees partielles
par des taux d'accroissement pour obtenir des r\'esultats convenables.
Il faut aussi v\'erifier certaines propri\'et\'es que nous abordons dans
la section suivante.

\begin{figure}[h]
  \centering
  \includegraphics[width = 4cm]{Figures/A_venir.png}
  \includegraphics[width = 4cm]{Figures/A_venir.png}
  \includegraphics[width = 4cm]{Figures/A_venir.png}
  \caption{\'Equation de transport avec conditions p\'eriodiques}
  \label{fig:trans_per}
\end{figure}

!! exercice : coder les deux sch\'emas et retrouver les r\'esultats
de la figure \ref{fig:trans_per} !!

%%%%%%%%%%%%%%%%%%
\subsubsection{Outils d'analyse num\'erique}

Nous d\'efinissons maintenant des notions que nous
utiliserons ensuite pour analyser les deux sch\'emas introduits pr\'ec\'edemment.
Pour cela, consid\'erons une suite $(u_j^n)_{\substack{0 \leq n \leq N \\ 0 \leq j \leq M}}$
telle que $u_j^n$ soit \'egal \`a la $j$-\`eme coordonn\'ee du vecteur
$U^n$ g\'en\'er\'e par r\'ecurrence comme suit:
\begin{align}
  \label{eq:schema_1pas}
  U^0 = U_0 \qquad \text{et} \qquad \forall n \geq 0, \quad U^{n+1} = A U^n + h_t F^n ,
\end{align}
o\`u $U_0$ est donn\'e par $(U_0)_j = u(t_0,x_j)$.
Notons \'egalement que nous consid\'erons dans la suite
$U_0, F^n \in \R^{M+1}$ et $A \in \R^{(M+1)\times(M+1)}$.
Si ces vecteurs et matrice ont une autre dimension, cela ne change pas la suite de
notre propos.

\begin{remark}
  Dans le cas pr\'esent, nous consid\'erons une suite dont la formule de r\'ecurrence
  ne n\'ecessite que le pas de temps pr\'ec\'edent.
  En pratique, certains sch\'emas n\'ecessitent plusieurs pas de temps pr\'ec\'edents.
  Le sch\'ema \eqref{eq:schema_1pas} peut \^etre adapt\'e \`a ce cas de figure.
  Nous verrons ceci plus loin dans ce cours.
\end{remark}

\begin{definition}[Stabilit\'e]
  \label{def:stabilite}
  Soit $\Stab \subset \R_+^* \times \R_+^*$ tel que $(0,0) \in \overline{\Stab}$.
  On dit que le sch\'ema num\'erique \eqref{eq:schema_1pas} est stable 
  sous la condition $\Stab$ s'il existe $C_1 , C_2 > 0$ qui ne d\'ependent que de $T$
  tels que $\forall (h_x , h_t) \in \Stab$, $\forall U_0 \in \R^{M+1}$, 
  $\forall 0 \leq n \leq N$, $\forall F^n \in \R^{M+1}$, on a
  \begin{align}
    \max_{\substack{0\leq n \leq N\\ 0 \leq j \leq M}} | u_j^n | 
    \leq C_1 \max_{0 \leq j \leq M} | (U_0)_j | 
    + C_2 \max_{\substack{0\leq n \leq N\\ 0 \leq j \leq M}} | F_j^n | ,
  \end{align}
  o\`u $(u_j^n)$ est la suite g\'en\'er\'ee par la r\'ecurrence \eqref{eq:schema_1pas}.

  On dit \'egalement que le sch\'ema est inconditionnellement stable
  si la d\'efinition de stabilit\'e conditionnelle 
  s'applique avec $\Stab = \R_+^* \times \R_+^*$.
\end{definition}

\begin{definition}[Erreur de troncature]
  \label{def:troncature}
  L'erreur de troncature du sch\'ema \eqref{eq:schema_1pas}
  au temp $t_n$ est un vecteur $\varepsilon^n \in \R^{M+1}$
  d\'efini par
  \begin{align}
    \label{eq:troncature}
    \varepsilon^{n+1} := \tu^{n+1} - A \tu^n - h_t F^n ,
  \end{align}
  o\`u $\tu^n \in \R^{M+1}$ est d\'efini par
  $(\tu^n)_j = u(t_n,x_j)$ avec $u$ la solution du probl\`eme exact associ\'e \`a
  \eqref{eq:schema_1pas}.
\end{definition}

\begin{definition}[Consistance]
  \label{def:consistance}
  On dit que le sch\'ema num\'erique \eqref{eq:schema_1pas} est consistant si pour toute
  solution r\'eguli\`ere $u$ du probl\`eme exact on a
  \begin{align}
    \lim_{\substack{h_t \to 0\\ h_x \to 0}} \max_{\substack{0\leq n \leq N\\ 0 \leq j \leq M}} 
    \dfrac{| \varepsilon_j^n|}{h_t} = 0 ,
  \end{align}
  o\`u $\varepsilon_j^n = (\varepsilon^n)_j$ l'erreur de troncature.

  On dit de plus que le sch\'ema \eqref{eq:schema_1pas} est consistant d'ordre $p \in \N^*$
  en temps et $q \in \N^*$ en espace si pour toute solution r\'eguli\`ere $u$ (du probl\`eme exact)
  il existe une constante $C>0$ ind\'ependante de $h_t$ et $h_x$ telle que
  \begin{align}
    \forall h_t , h_x > 0 , \quad \max_{\substack{0\leq n \leq N\\ 0 \leq j \leq M}}
    \dfrac{| \varepsilon_j^n|}{h_t} \leq C (h_t^p + h_x^q) .
  \end{align}
\end{definition}

\begin{definition}[Convergence]
  \label{def:convergence}
  On dit que le sch\'ema num\'erique \eqref{eq:schema_1pas} converge (ou est convergent) 
  sous la condition $\Stab \subset \R_+^* \times \R_+^*$
  si 
  \begin{align}
    \lim_{\substack{(h_t,h_x) \in \Stab \\ (h_t,h_x) \to (0,0)}} \max_{\substack{0\leq n \leq N\\ 0 \leq j \leq M}} 
    | u(t_n , x_j) - u_j^n | = 0 ,
  \end{align}
  o\`u $u$ est la solution du probl\`eme exact.
  
  De mani\`ere similaire, on dit que le sch\'ema num\'erique 
  \eqref{eq:schema_1pas} converge (ou est convergent) 
  \`a l'ordre $p\in\N^*$ en temps et $q\in\N^*$ en espace
  sous la condition $\Stab \subset \R_+^* \times \R_+^*$
  s'il existe une constante $C>0$ ind\'ependante de $h_t$ et $h_x$
  telle que
  \begin{align}
    \forall (h_t , h_x) \in \Stab , \quad \max_{\substack{0\leq n \leq N\\ 0 \leq j \leq M}}
    | u(t_n,x_j) - u_j^n | \leq C (h_t^p + h_x^q) .
  \end{align}

  On dit enfin que le sch\'ema est convergent (resp. convergent \`a l'ordre $p$
  en temps et $q$ en espace) si la d\'efinition ci-dessus est v\'erifi\'ee
  avec $\Stab = \R_+^* \times \R_+^*$.
\end{definition}

Ces d\'efinitions sont reli\'ees par le th\'eor\`eme suivant.
\begin{theorem}
  \label{thm:convergence}
  Si le sch\'ema \eqref{eq:schema_1pas} est stable sous la condition
  $\Stab \subset \R_+^* \times \R_+^*$ et consistant 
  (respectivement consistant d'ordre $p$ en temps et $q$ en espace),
  alors le sch\'ema \eqref{eq:schema_1pas} est convergent
  (respectivement convergent d'ordre $p$ en temps et $q$ en espace)
  sous la condition $\Stab$.
\end{theorem}
Le th\'eor\`eme pr\'ec\'edent s'applique aussi avec $\Stab = \R_+^* \times \R_+^*$.
Avant d'exposer une preuve du th\'eor\`eme \ref{thm:convergence}
faisons quelques remarques sur les d\'efinitions pr\'ec\'edentes.

Concernant la stabilit\'e (cf d\'efinition \ref{def:stabilite}),
les quantit\'es $\max_{\substack{0\leq n \leq N\\ 0 \leq j \leq M}} | u_j^n |$, 
$\max_{0\leq j \leq M} | (U_0)_j |$ et $\max_{\substack{0\leq n \leq N\\ 0 \leq j \leq M}} | F_j^n |$ 
sont des normes portant sur $(u_j^n)$, $U_0$ et $(F^n)$.
On peut donner dire que la stabilit\'e du sch\'ema correspond \`a
la continuit\'e de l'application lin\'eaire qui \`a 
$U_0$ et $(F^n)$ associe $(u_j^n)$ en ajoutant que la constante
associ\'ee \`a cette continuit\'e ne doit pas d\'ependre
de $h_t$ et $h_x$.
On pourrait consid\'erer d'autres normes pour exprimer cette stabilit\'e,
il faudrait alors utiliser des poids adapt\'es en $h_t$ et $h_x$ pour 
avoir une d\'efinition \'equivalente (la constante $C$ ne doit
pas d\'ependre de $h_t$ et $h_x$).
Par exemple, on pourrait consid\'erer la norme
$\max_{0\leq n \leq N} \sqrt{h_x} (\sum_{0\leq j \leq M} (u_j^n)^2)^{1/2}$
pour $(u_j^n)$.
Pour ne pas avoir \`a consid\'erer des poids en $h_x$ et $h_t$,
nous consid\'erons les normes avec des "$\max$".
Cette continuit\'e permet d'obtenir le fait qu'une "petite"
modification de la donn\'ee initiale ou des termes source
ou fronti\`ere ne cr\'ee qu'une "petite" modification
du r\'esultat obtenu.
La condition $\Stab$ permet de traiter des cas o\`u le sch\'ema
n'est stable que si on restreint les $h_t$ et $h_x$ utilis\'es.
Nous verrons un tel exemple dans la suite de cette section.


L'erreur de troncature repr\'esente l'erreur que fait un sch\'ema
au cours d'un pas de temps.
En effet, on compare (cf d\'efinition \ref{def:troncature})
la solution exacte $\tu^{n+1}$ \`a la solution 
obtenue en faisant un pas de temps du sch\'ema
\`a partir de la solution exacte au pas de temps
pr\'ec\'edent $\tu^{n+1}$.


La consistance d'un sch\'ema (cf d\'efinition \ref{def:consistance})
signifie que l'erreur de troncature commise par unit\'e de temps
(on divise par $h_t$) tend vers $0$ lorsqu'on prend 
des pas de temps et d'espace de plus en plus petits.
De plus, on peut quantifier la notion de consistance
gr\^ace \`a l'ordre de consistance 
(plus les ordres de consistance en temps et en espace sont \'elev\'es,
plus l'erreur de troncature d\'ecro\^it vite vers $0$ quand les pas 
de temps et d'espace tendent vers $0$).
En quelque sorte, la consistance signifie que le sch\'ema est
coh\'erent avec le probl\`eme exact 
(les termes pr\'esents dans le sch\'ema num\'erique correspondent
\`a des termes du probl\`eme exact et vice-versa).


La convergence d'un sch\'ema signifie que la solution approch\'ee obtenue
correspond bien \`a une approximation de la solution exacte
au sens o\`u si l'on fait tendre les pas de temps et d'espace vers
$0$, l'erreur commise par le sch\'ema tend aussi vers $0$.
L\`a encore on peut quantifier cette d\'ecroissance de l'erreur commise 
avec la notion d'ordre de convergence.


En pratique, la convergence est la propri\'et\'e que l'on veut avoir
car elle garantit que le sch\'ema consid\'er\'e fournit une approximation
raisonnable du probl\`eme exact (sous r\'eserve que les pas de temps
et d'espace soient suffisamment petits).
Cependant, cette propri\'et\'e n'est pas ais\'ee \`a d\'emontrer.
C'est pourquoi nous utilisons le th\'eor\`eme \ref{thm:convergence}
qui assure la convergence \`a partir de la stabilit\'e et 
de la consistance.


\begin{remark}
  Les notions de consistance et de convergence sont diff\'erentes.
  Les jurys se plaignent des candidats qui confondent ces deux
  notions.
  Ne fa\^ites pas cette erreur.
\end{remark}

Nous allons maintenant prouver le th\'eor\`eme \ref{thm:convergence}.
\begin{proof}[Preuve du th\'eor\`eme \ref{thm:convergence}]
  Dans cette preuve, nous utilisons toutes les d\'efinitions pr\'ec\'edentes.
  Tout d'abord, par d\'efinition de l'erreur de troncature,
  $\tu^{n+1} = A \tu^n + h_t F^n + \varepsilon^{n+1}$.
  On montre ainsi que
  $w^n = \tu^{n} - U^{n}$ v\'erifie $w^0 = 0$ 
  (puisque $(U^0)_j = u(t_0,x_j)$) et
  $w^{n+1} = A w^n + \varepsilon^{n+1}$.

  Par stabilit\'e du sch\'ema, on obtient
  $\max_{\substack{0\leq n \leq N\\ 0 \leq j \leq M}} | w_j^n |
  \leq C_2 \max_{\substack{0\leq n \leq N\\ 0 \leq j \leq M}} \dfrac{| \varepsilon_j^n |}{h_t}$.
  On conclut en utilisant la d\'efinition de la consistance.
\end{proof}


Pour $A \in \R^{(M+1)\times (M+1)}$, on note 
\[
  \tn A \tn = \sup_{V \in \R^{M+1} \backslash\{0\}} \dfrac{\| A V \|}{\| V \|} ,
\]
o\`u $\| V \| = \max\limits_{0\leq j \leq M} | V_j |$.
Nous donnons maintenant une caract\'erisation de la stabilit\'e du sch\'ema.
\begin{proposition}
  \label{prop:carac_stab}
  Le sch\'ema \eqref{eq:schema_1pas} est stable si et seulement si
  il existe $C>0$ d\'ependant uniquement de $T$ tel que
  \begin{align}
    \label{eq:carac_stab}
    \forall n \in \N , \quad \tn A^n \tn \leq C .
  \end{align}
\end{proposition}

\begin{corrolary}
  Si $\tn A \tn \leq 1$, alors le sch\'ema est stable.
\end{corrolary}

\begin{proof}[Preuve de la proposition \ref{prop:carac_stab}]
  Supposons tout d'abord que le sch\'ema est stable.
  Dans ce cas, appliquons le sch\'ema \`a $U_0 = V \in \R^{M+1}$
  quelconque et $F^n = 0$. 
  La solution obtenue est $U^n = A^n V$.
  On a donc prouv\'e qu'il existe $C>0$ d\'ependant uniquement
  de $T$ tel que
  $\forall V \in \R^{M+1} , \quad \| A^n V \| \leq C \| V \|$
  et donc \eqref{eq:carac_stab}.

  Maintenant supposons que \eqref{eq:carac_stab} est \'etablie.
  Appliquons le sch\'ema \`a $U_0 \in \R^{M+1}$
  et $\forall n \in \N , F^n \in \R^{M+1}$.
  Nous pouvons prouver que la solution obtenue est
  \begin{align}
    \label{eq:sol_Un}
    U^n = A^n U_0 + h_t \sum_{\ell = 0}^{n-1} A^{n-\ell-1} F^{\ell} .
  \end{align}
  Ainsi, $\| U^n \| \leq \tn A^n \tn \| U_0 \| 
  + h_t \sum_{\ell = 0}^{n-1} \tn A^{n-\ell-1} \tn \| F^n \|$.
  En utilisant \eqref{eq:carac_stab}, on obtient
  $\| U^n \| \leq C \| U_0 \| + C h_t n \max_{0\leq \ell \leq N} \| F^{\ell} \|$.
  De plus, $n h_t = t_n \leq T$.
  On a donc \'etabli la stabilit\'e du sch\'ema.
\end{proof}


!! Exercice : \'etablir \eqref{eq:sol_Un} !!

%%%%%%%%%%%%%%%%%%
\subsubsection{Analyse des deux sch\'emas pour conditions p\'eriodiques}


Nous allons montrer dans cette section que le sch\'ema num\'erique 
propos\'e en \eqref{eq:transport_DF_centre} n'est pas stable
contrairement au sch\'ema \eqref{eq:transport_DF_decentre_AM}.

On consid\`ere comme degr\'es de libert\'e
$U^n = 
\begin{pmatrix}
  u_0^n \\ \vdots \\ u_{M-1}^n
\end{pmatrix}
\in \R^M
$.
Les deux sch\'emas pr\'ec\'edents peuvent \^etre \'ecrits sous la forme matricielle 
\eqref{eq:schema_1pas} avec $F^n = 0$ et pour matrices
\begin{align}
  A_{C} = \dfrac{1}{2}
  \begin{pmatrix}
    2 & -\co & 0 & \dots & 0 & \co
    \\
    \co & 2 & -\co & & & 0
    \\
    0 & \ddots & \ddots & \ddots & &
    \\
    \vdots & & \ddots & \ddots & \ddots
    \\
    0 & & & \ddots & \ddots & -\co
    \\
    -\co & 0 & & & \co & 2
  \end{pmatrix},
\end{align}
pour le sch\'ema centr\'e \eqref{eq:transport_DF_centre}
et
\begin{align}
  A_{AM} = 
  \begin{pmatrix}
    1-\co & 0 & \dots & 0 & \co
    \\
    \co & \ddots & & & 0
    \\
    & \ddots & \ddots &   &
    \\
    & & \ddots & \ddots &
    \\
    &&   &  \co & 1-\co
  \end{pmatrix},
\end{align}
pour le sch\'ema d\'ecentr\'e amont \eqref{eq:transport_DF_decentre_AM}.
Les deux matrices pr\'ec\'edentes ont \'et\'e \'ecrites avec 
$\co = \frac{a h_t}{h_x}$.
Notons que les deux matrices pr\'ec\'edentes sont dans
$\R^{M \times M}$.

\begin{proposition}
  Les sch\'emas \eqref{eq:transport_DF_centre} et \eqref{eq:transport_DF_decentre_AM}
  sont tous les deux consistants.
  De plus, le sch\'ema \eqref{eq:transport_DF_centre} est consistant d'ordre 1
  en temps et 2 en espace;
  le sch\'ema \eqref{eq:transport_DF_decentre_AM} est consistant d'ordre 1
  en temps et 1 en espace.
\end{proposition}


\begin{proof}
  Prouvons que le sch\'ema \eqref{eq:transport_DF_centre} est consistant
  d'ordre 1 en temps et 2 en espace.
  La preuve pour le sch\'ema d\'ecentr\'e amont est laiss\'ee en exercice.

  On calcule l'erreur de troncature. Ceci revient \`a 
  calculer une solution approch\'ee au temps $t_{n+1}$ par le sch\'ema \`a partir
  de la solution exacte au temps $t_n$ et de comparer le r\'esultat
  obtenu \`a la solution exacte au temps $t_{n+1}$.
  On a ainsi
  \[
    \varepsilon_j^{n+1} = u(t_{n+1},x_j) - u(t_n,x_j)
    + \dfrac{a h_t}{2 h_x} (u(t_n, x_{j+1}) - u(t_n, x_{j-1})) ,
  \]
  o\`u $u$ est la solution de \eqref{eq:transport_per}.

  On consid\`ere ensuite les d\'eveloppements de Taylor suivants:
  \begin{align*}
    u(t_{n+1} , x_j) = u(t_n , x_j) + h_t \dfrac{\partial u}{\partial t} (t_n , x_j)
    + O(h_t^2) ,
    \\
    u(t_n , x_{j+1} ) = u(t_n , x_{j} ) + h_x \dfrac{\partial u}{\partial x}(t_n, x_j)
    + \dfrac{h_x^2}{2} \dfrac{\partial^2 u}{\partial x^2}(t_n, x_j)
    + O(h_x^3) ,
    \\
    u(t_{n} , x_{j-1} ) = u(t_n , x_{j} ) - h_x \dfrac{\partial u}{\partial x}(t_n, x_j)
    + \dfrac{h_x^2}{2} \dfrac{\partial^2 u}{\partial x^2}(t_n, x_j)
    + O(h_x^3) .
  \end{align*}
  Nous obtenons donc
  \begin{align*}
    \dfrac{\varepsilon_j^{n+1}}{h_t} = \dfrac{\partial u}{\partial t}(t_n, x_j)
    + O (h_t) + a \dfrac{\partial u}{\partial x}(t_n, x_j) + O (h_x^2) .
  \end{align*}
  En utilisant le fait que $u$ est solution de \eqref{eq:transport_per},
  nous avons prouv\'e le fait que le sch\'ema \eqref{eq:transport_DF_centre}
  est consistant d'ordre 2 en espace et d'ordre 1 en temps.
\end{proof}

!! exercice : consistance du sch\'ema \eqref{eq:transport_DF_decentre_AM} !!


\begin{proposition}
  Le sch\'ema \eqref{eq:transport_DF_decentre_AM} est stable.
\end{proposition}

\begin{proposition}
  Le sch\'ema \eqref{eq:transport_DF_centre} est instable.
\end{proposition}

!! expliquer qu'il faut aller chercher l'information qui vient
pas celle qui part !!


%%%%%%%%%%%%%%%%%%
\subsubsection{Conditions de Dirichlet}

!! on s'int\'eresse au pb (...) !!

En appliquant \eqref{eq:der_decentre_AM}, 
on obtient le sch\'ema num\'erique
\begin{align}
  \label{eq:transport_DF_decentre_AM_dir}
  \left\{
  \begin{array}{l}
    \forall 0 \leq j \leq M , 
    u_j^0 = u_0(x_j) ,
    \\
    \forall 1 \leq n \leq N, u_0^n = \alpha(t_n) ,
    \\
    \forall 1 \leq n \leq N, \forall 1 \leq j \leq M,
    u_j^n = u_j^{n-1} - \dfrac{a h_t}{h_x} (u_{j}^{n-1} - u_{j-1}^{n-1}) .
  \end{array}
  \right.
\end{align}

On a donc 
\begin{align}
  A =
  \begin{pmatrix}
    1 - \co & 0 & & &
    \\
    \co & 1 - \co & \ddots & &
    \\
    0 & \ddots & \ddots & \ddots &
    \\
    \vdots & & \ddots & \ddots & 0
    \\
    0 & & & \co & 1-\co
  \end{pmatrix}
  \qquad \text{et} \qquad
  \forall n \geq 0,
  \quad h_t F^n = \co
  \begin{pmatrix}
    \alpha(t_n)
    \\
    0
    \\
    \vdots
    \\
    0
  \end{pmatrix}
  ,
\end{align}
avec $\co = \frac{a h_t}{h_x}$.
Remarquons que la condition de Dirichlet a \'et\'e prise en compte
dans le terme "second membre" $F^n$.

On obtient les r\'esultats (...)

!! est-ce qu'on ne ferait pas mieux de faire l'analyse de stabilit\'e
avec des conditions p\'eriodiques ??
-- Ca simplierait les choses en particulier pour les sch\'emas instables !!


!! remplacer les d\'eriv\'ees continues par des taux d'accroissement
assure la consistance mais pas forc\'ement la stabilit\'e !!
-- C'est tres interessant de faire ca sur l'\'equation de transport par ce que 
l'information ne se d\'eplace que dans une direction !!

!! faire des sous-sections dans cette section !!

!! prouver que le sch\'ema en intro n'est pas stable !!

!! donner un sch\'ema stable sous CFL et faire les preuves + simus num !!

!! introduire nombre de CFL + sens physique + simus avec diff\'erentes CFL !!

!! donner un sch\'ema inconditionnellement stable !!

!! comparer les r\'esultats avec ceux de l'intro !!

!! diffusion num\'erique : comparer avec l'\'equation de la chaleur !!

!! discr\'etisation implicite en exercice !!

!! dessin repr\'esentant l'endroit o\`u on va chercher l'information :
sch\'ema centr\'e / d\'ecentr\'e + effet de la cfl !!

%%%%%%%%%%%%%%%%%%%%%%%%%%%%%%%%%%%%%%%%%%%%%%%%%%%%%%%%%%%%%%%%%%%%%%%%%%%%%%%%%%%%
\section{\'Equation de la chaleur}

%%%%%%%%%%%%%%%%%%%%%%%%%%%%%%%%%%%%%%%%%%%%%%%%%%%%%%%%%%%%%%%%%%%%%%%%%%%%%%%%%%%%
\section{\'Equation des ondes}


%%%%%%%%%%%%%%%%%%%%%%%%%%%%%%%%%%%%%%%%%%%%%%%%%%%%%%%%%%%%%%%%%%%%%%%%%%%%%%%%%%%%
\section{Pour aller plus loin}

!! Ecrire tous les exercices au fur et \`a mesure et les reporter
dans une section \`a la fin du poly !!


!! les elements de cette section sont hors-programme mais ne sont pas 
d\'econnect\'es de celui-ci au sens o\`u ils peuvent \^etre utilis\'es 
dans un texte (ils seront alors pr\'esent\'es) !!
!! Ne travailler cette section qu'une fois que les autres sont parfaitement maitris\'ees !!

%%%%%%%%%%%%%%%%%%%%%%%%%%%%%%%%%%%%%%%
\subsection{\'Equation de transport dans un domaine multi-dimensionnel}
\label{subsec:transport_multiD}


%===================================================================================

\end{document}
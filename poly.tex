\documentclass[12pt,a4paper,twoside]{article}
\addtolength{\textheight}{80pt} \addtolength{\topmargin}{-50pt}
\textwidth 164mm \oddsidemargin -2.25mm \evensidemargin -2.25mm
\usepackage{amssymb}
\usepackage{amsmath}
\usepackage{amsthm} % theoremes
\usepackage[T1]{fontenc}
\usepackage[utf8]{inputenc} 
\usepackage[francais]{babel}
\usepackage{enumitem}
\usepackage{url}
\usepackage{graphicx}
\usepackage{comment}
\usepackage{xcolor}

%% pour les figures
\usepackage{tikz}
\usepackage{forloop}

\begin{document}

%%%%%%%%%%%%%%%%%%%%%%%%%%%%%%%%%%%%%

\newcommand{\Ni}[1]{{\left\|#1\right\|}_\infty}
\newcommand*{\R}{\mathbb{R}}
\newcommand*{\N}{\mathbb{N}}
\newcommand*{\Z}{\mathbb{Z}}
\newcommand*{\C}{\mathbb{C}}

\newcommand*{\DIV}{\nabla \cdot}
\newcommand*{\GRAD}{\nabla}
\newcommand*{\deriv}{\;\mathrm{d}}
\newcommand*{\SCAL}{\cdot}

\newcommand*{\dans}{\text{ dans }}
\newcommand*{\sur}{\text{ sur }}

\newcommand*{\st}{\; | \;}

\newcommand*{\calA}{\mathcal{A}}

%% maths
\newcommand*{\Stab}{\mathcal{S}}
\newcommand*{\tu}{\widetilde{u}}
\newcommand*{\tn}{|\!|\!|}
\newcommand*{\cfl}{\mathrm{cfl}}
\newcommand*{\co}{\mathrm{c}}
\newcommand*{\calF}{\mathcal{F}}
\newcommand*{\hu}{\hat u}
\newcommand*{\hf}{\hat f}

%% environnements
\newtheorem{theorem}{Th\'eor\`eme}
\newtheorem{proposition}{Proposition}
\newtheorem{remark}{Remarque}
\newtheorem{definition}{D\'efinition}
\newtheorem{corrolary}{Corrolaire}
\newtheorem{exercise}{Exercice}
\newtheorem{lemma}{Lemme}
% \noindent
% {\rule{\textwidth}{.2mm}}\\
% \renewcommand{\labelenumi}{(\alph{enumi})}
% \noindent
Polytech Sorbonne \hfill Ann{\'e}e universitaire 2023--2024\\
Analyse Numérique des EDP \hfill
MAIN4 \hfill mercredi 15 mai 2024
% {\rule{\textwidth}{.2mm}}\\

\begin{center}
{\bf \Huge Introduction aux \'equations aux d\'eriv\'ees partielles}
\end{center}

% \title{Introduction aux \'equations aux d\'eriv\'ees partielles}

%\vfill
%===================================================================================

%%%%%%%%%%%%%%%%%%%%%%%%%%%%%%%%%%%%%%%%%%%%%%%%%%%%%%%%%%%%%%%%%%%%%%%%%%%%%%%%%%%%
\section*{Avant-propos}

Le but de ce cours est de vous proposer une introduction \`a la th\'eorie
des \'equations aux d\'eriv\'ees partielles (EDP dans la suite).
Nous \'etudierons plusieurs \'equations ainsi que leur discr\'etisation
par la m\'ethode des diff\'erences finies.

Dans le cadre du programme officiel de la pr\'eparation \`a l'agr\'egation
de math\'ematiques (\'epreuve de mod\'elisation), nous aborderons notamment:
\begin{itemize}
\item des notions \'el\'ementaires portant sur les EDP classiques en dimension 1,
\item l'\'equation de transport lin\'eaire avec la m\'ethode des caract\'eristiques,
\item l'\'equation des ondes et de la chaleur; une r\'esolution par s\'erie de Fourier
  et transform\'ee de Fourier sera propos\'ee ainsi qu'une m\'ethode de s\'eparation
  des variables.
  Les aspects qualitatifs seront abord\'es.
\item les \'equations elliptiques avec l'utilisation du th\'eor\`eme de Lax--Milgram
\item des exemples de discr\'etisation des EDP en dimension 1 avec la m\'ethode des
  diff\'erences finies. L'\'etude des propri\'et\'es de ces discr\'etisations
  sera propos\'ee : notions de consistance, stabilit\'e, convergence et d'ordre.
\end{itemize}


Vous \^etes par ailleurs invit\'es \`a lire le rapport du jury (disponible sur internet).
Vous vous rendrez compte que le jury insiste notamment sur le fait que :
\begin{itemize}
\item l'\'epreuve de mod\'elisation, comme les autres, requiert de la rigueur math\'ematique,
\item il faut \'equilibrer sa pr\'esentation entre une pr\'esentation du mod\`ele \'etudi\'e,
  des preuves math\'ematiques rigoureuses, des illustrations informatiques,
\item il attend une prise de recul de la part des candidats.
  Il faudra donc notamment \^etre capable de critiquer les limites du mod\`ele pr\'esent\'e
  dans le texte, d'expliquer le comportement qualitatif de celui-ci 
  (par exemple expliquer ce qu'il se passe quand la valeur d'un param\`etre change)
  et \^etre capable de conclure sur la probl\'ematique de d\'epart.
\end{itemize}


Ce cours sera compos\'e de
\begin{itemize}
\item cinq s\'eances de cours de deux heures chacune,
\item une s\'eance de programmation de deux heures.
\end{itemize}
Dans une premi\`ere partie, nous pr\'esenterons les \'equations \'etudi\'ees
dans ce cours en donnant une id\'ee des probl\`emes physiques associ\'es.
Chacune des parties suivantes sera consacr\'ee \`a l'\'etude plus approfondie
d'une EDP. Nous y traiterons notamment les principales caract\'eristiques de cette EDP,
les outils utilis\'es pour mener des preuves ainsi qu'une discr\'etisation par diff\'erences finies.
Les EDP \'etudi\'ees dans la suite seront les \'equations elliptiques,
l'\'equation de transport, l'\'equation de la chaleur et enfin
l'\'equation des ondes.

%%%%%%%%%%%%%%%%%%%%%%%%%%%%%%%%%%%%%%%%%%%%%%%%%%%%%%%%%%%%%%%%%%%%%%%%%%%%%%%%%%%%
\section{Pr\'esentation des EDP du cours}

Nous pr\'esentons dans cette section les EDP \'etudi\'ees dans la suite du cours.
Nous essayons de donner une signification physique aux diff\'erents termes.

Nous nous int\'eressons \`a des EDP de la forme
\begin{align*}
  a \dfrac{\partial^2 u}{\partial x^2} + b \dfrac{\partial^2 u}{\partial x \partial y}
  + c \dfrac{\partial^2 u}{\partial y^2} + d \dfrac{\partial u}{\partial x}
  + e \dfrac{\partial u}{\partial y} + f u = F .
\end{align*}
Pour d\'eterminer la nature de l'EDP, on associe \`a chaque d\'eriv\'ee
la variable qui correspond \`a la direction de d\'erivation.
L'\'equation pr\'ec\'edente devient donc
\begin{align*}
  a x^2 + b xy + c y^2 + d x + e y + f = A ,
\end{align*}
avec $A$ un r\'eel tel que l'ensemble des solutions soit non vide.
S'il s'agit de l'\'equation:
\begin{itemize}
\item d'une ellipse, on dira que l'\'equation est elliptique;
\item d'une parabole, on dira que l'\'equation est parabolique;
\item d'une hyperbole, on dira que l'\'equation est hyperbolique.
\end{itemize}
Cette d\'enomination n'est pas juste esth\'etique.
En effet, comme nous le verrons plus loin dans ce cours,
chacun de ces types d'\'equations dispose de propri\'et\'es sp\'ecifiques.

Notons \'egalement que les \'equations pr\'ec\'edentes d\'ependent de deux
variables d'espace. La d\'enomination pr\'ec\'edente se g\'en\'eralise 
dans le cas o\`u on aurait une seule variable ou strictement plus de deux variables.
Dans le cadre de ce cours, nous nous concentrerons sur l'\'etude d\'equations
avec une seule dimension d'espace.
On consid\'erera donc une seule variable $x$ dans le cas d'un probl\`eme stationnaire
et deux variables $t$ (le temps) et $x$ (l'espace) dans le cas d'un probl\`eme
instationaire.

Dans la suite de ce cours, on notera $\Omega$ un ouvert born\'e de $\R^d$
avec $d=1,2,3$.

%%%%%%%%%%%%%%%%%%%%%%%%%%%%%%%%%%%%%%%
\subsection{\'Equations elliptiques}
\label{subsec:elliptique}

Les \'equations elliptiques apparaissent principalement dans deux contextes que nous
allons maintenant aborder.
Le premier est le cas o\`u des particules circulent dans un domaine.
Ce probl\`eme est repr\'esent\'e sur la figure \ref{fig:flux}.
Dans la partie droite de cette figure, nous nous int\'eressons \`a un probl\`eme
o\`u des particules circulent dans un milieu unidimensionnel. La position est
rep\'er\'ee par la coordonn\'ee d'espace $x$. On note $u(x)$ la densit\'e
de particules en $x$. Certaines particules entrent ou sortent du domaine
en $x$, on les note $f(x)$ le terme source les repr\'esentant.
De plus, les particules se d\'eplacent \`a travers le domaine,
on note $q(x)$ le flux de particules en $x$
(le nombre de particules qui traversent l'axe vertical d'abscisse $x$).
Ce flux est n\'egatif si les particules vont vers la gauche ($x$ d\'ecroissants)
et positif si elles vont vers la droite ($x$ croissants).

\begin{figure}
\begin{tikzpicture}[scale = 3]
  \def\h{0.1}
\def\N{11.0}
\newcounter{itx}
\newcounter{ity}

%% pente
\forloop{itx}{0}{\value{itx} < \N}{
\forloop{ity}{1}{\value{ity} < \value{itx} }{
\draw (\arabic{itx}*\h,\arabic{ity}*\h) node{$\circ$};
}
}

%% plateau
\forloop{itx}{0}{\value{itx} < 9.0}{
\forloop{ity}{1}{\value{ity} < 11.0 }{
\draw (1.0+\h+\arabic{itx}*\h,\arabic{ity}*\h) node{$\circ$};
}
}

%% axe des x
\draw[->] (0.0,0.0) -- (2.5 , 0.0);
\draw (2.5,-0.1) node{$x$};

%% terme source 1
\forloop{itx}{0}{\value{itx} < 4.0}{
\draw[->] (1.2 + 2*\h * \arabic{itx}, 1.5) -- (1.2 + 2*\h * \arabic{itx}, 1.1);
\draw (1.2 + 2*\h * \arabic{itx}, 1.3) node{$\circ$};
\draw (1.2 + 2*\h * \arabic{itx}, 1.5) node{$\circ$};
}
\draw (2.0, 1.3) node{$f(x)$};

%% terme source 2
\forloop{itx}{0}{\value{itx} < 2.0}{
\draw[->] (0.2 + 2*\h * \arabic{itx}, 0.4) -- (0.2 + 2*\h * \arabic{itx}, 0.8);
\draw (0.2 + 2*\h * \arabic{itx}, 0.6) node{$\circ$};
\draw (0.2 + 2*\h * \arabic{itx}, 0.4) node{$\circ$};
}
\draw (0.0, 0.6) node{$f(x)$};


%% densite u
\draw[<->] (2.0, \h/2.0) -- (2.0, 1.05);
\draw (2.2, 0.6) node{$u(x)$};

%% flux q
\draw[->] (0.95, 0.9) -- (0.6, 0.9);
\draw (0.8, 1.0) node{$q(x)$};
\end{tikzpicture}
\begin{tikzpicture}[scale = 2.5]
  \def\h{0.1}
\def\N{11.0}

%% rectangle
\draw (0,0) -- (2,0) -- (2,1) -- (0,1) -- (0,0);
\draw (1.0, 0.5) node{$u(x)$};

%% source term
\forloop{itx}{0}{\value{itx} < 9.0}{
\draw[->] (0.2 + 2*\h * \arabic{itx}, 1.5) -- (0.2 + 2*\h * \arabic{itx}, 1.1);
}
\draw (2.0, 1.3) node{$f(x)$};

%% flux
\draw[->] (-0.6,0.5) -- (-0.1,0.5);
\draw (-0.4, 0.3) node{$q(x)$};
\draw[->] (2.1,0.5) -- (2.8,0.5);
\draw (2.5, 0.3) node{$q(x+\delta x)$};

%% axe des abscisses
\draw[->] (-0.4,-0.1) -- (2.3,-0.1);
\draw (0.0, -0.1) -- (0.0, -0.2);
\draw (0.0, -0.3) node{$x$};
\draw (2.0, -0.1) -- (2.0, -0.2);
\draw (2.0, -0.3) node{$x+\delta x$};
\end{tikzpicture}
\caption{Un flux de particules. Gauche: repr\'esentation du probl\`eme.
  Droite: \'equilibre des flux.}
\label{fig:flux}
\end{figure}


On s'int\'eresse au cas o\`u les flux sont \`a l'\'equilibre, il n'y
a donc pas d'accumulation de particules en aucun point de l'espace.
Le probl\`eme ne d\'epend pas du temps.


Si l'on consid\`ere un \'el\'ement du domaine de taille $\delta x$ comme
sur la droite de la figure \ref{fig:flux}, le nombre de particules doit
rester constant au cours du temps.
On obtient donc la relation de conservation
$q(x) - q(x+\delta x) + f(x) \delta x = 0$,
ce qui donne
\begin{align}
  \label{eq:eq_flux}
  \dfrac{\partial q}{\partial x} = f .
\end{align}

De plus, on consid\`ere que les particules fuient les zones de forte densit\'e:
le flux $q(x)$ est orient\'e dans le sens inverse du gradient de $u$.
On note donc 
\begin{align}
  \label{eq:def_flux}
  q(x) = - k(x) \dfrac{\partial u}{\partial x}(x) . 
\end{align}
Ici $k$ est un coefficient positif qui peut d\'ependre de l'espace.
Il traduit le rapport de proportionnalit\'e entre le gradient de la densit\'e
et le flux qui en r\'esulte. Ainsi, pour une densit\'e fix\'ee,
si $k$ est grand alors les particules circuleront facilement et le flux sera important;
\`a l'inverse, un $k$ petit traduit le fait que les particules ont du mal \`a circuler dans
le milieu. On dit que $k$ est un coeeficient de diffusion.

L'\'equation finale sur $u$ est donc
\begin{align*}
  - \dfrac{\partial}{\partial x} \left(k \dfrac{\partial u}{\partial x} \right) = f .
\end{align*}

Si l'on consid\`ere plusieurs dimensions d'espace, cette \'equation devient
\begin{align*}
  - \DIV \left(k \GRAD u \right) = f ,
\end{align*}
o\`u $\DIV$ et $\GRAD$ sont respectivement les op\'erateurs divergence et gradient.


En pratique les particules peuvent r\'eellement repr\'esenter des particules physiques,
dans ce cas $u$ sera une densit\'e de particules, $q$ un flux de particules, 
$f$ un terme repr\'esentant l'apparition ou la disparition de particules et
$k$ sera un coefficient de diffusion des particules.
On peut aussi dire que les particules repr\'esentent de l'\'energie thermique.
Dans ce cas, $u$ sera la temp\'erature, $q$ un flux thermique, $f$ une source
ou un puit de chaleur et $k$ sera la conductivit\'e thermique du mat\'eriau consid\'er\'e.
Nous citerons une derni\`ere possibilit\'e selon laquelle les particules sont des
individus (humains ou animaux).
Dans ce cas, $u$ correspond \`a une densit\'e de population,
$q$ \`a un flux de population, $f$ repr\'esente des naissances ou des morts
dans la population et $k$ est un coefficient de diffusion repr\'esentant la
facilit\'e avec laquelle la population peut se d\'eplacer. 


!! TABLEAU !!


!! CONDITIONS FRONTIERES !!

Un autre cas o\`u ces \'equations apparaissent est le cas d'un mat\'eriau 
soumis \`a des contraintes m\'ecaniques.
Par exemple, on repr\'esente sur la figure \ref{fig:barre} une barre \'elastique
en \'equilibre.


\begin{figure}
\begin{tikzpicture}[scale = 1.5]
  %% barre
\draw (0,0) -- (6,0) -- (6,1) -- (0,1) -- (0,0);

%% fixation
\draw (0, -0.5) -- (0, 1.5);
\forloop{itx}{0}{\value{itx} < 8}{
\draw (0, -0.25+0.25*\arabic{itx}) -- (-0.25, -0.5+0.25*\arabic{itx});
}


%% forces
\forloop{itx}{0}{\value{itx} < 8}{
\draw[->] (0.5+0.7*\arabic{itx},0.5) node{$\times$} -- (0.9+0.7*\arabic{itx},0.5);
}
\draw (4.2,0.25) node{$f$};

%% axe des abscisses
\draw[->] (-0.5, 2.0) -- (6.0,2.0);
\draw (6.0,1.7) node{$x$};


%% barre de reference
\draw[dashed] (0.0,2.5) -- (4,2.5) -- (4,3.5) -- (0,3.5) -- (0,2.5);
\draw[dashed] (0.0,2.5) -- (0.0,1.0);
\draw[dashed] (4,2.5) -- (6,1);


%% deplacement u
\draw[dashed] (2,2.5) -- (2,-1);
\draw (2,-1.2) node{$x$};
\draw[dashed] (2,2.5) -- (2.8,1);
\draw[dashed] (2.8,1) -- (2.8,-1);
\draw[->] (2,-0.5) -- (2.8,-0.5);
\draw (2.4,-0.7) node{$u(x)$};

\end{tikzpicture}
\begin{tikzpicture}[scale = 2]
  \def\h{0.1}
\def\N{11.0}

%% rectangle 1
\draw[dashed] (-0.8,1.0) -- (1.2,1.0) -- (1.2,1.5) -- (-0.8,1.5) -- (-0.8,1.0);

%% rectangle 2
\draw (0,0) -- (2,0) -- (2,0.5) -- (0,0.5) -- (0,0);

%% axe des abscisses 1
\draw[dashed,->] (-1.2,0.9) -- (1.8,0.9);
\draw[dashed] (-0.8, 0.9) -- (-0.8, 0.8);
\draw (-0.8, 0.7) node{$x$};
\draw[dashed] (1.2, 0.9) -- (1.2, 0.8);
\draw (1.2, 0.7) node{$x+\delta x$};

%% axe des abscisses 2
\draw[->] (-0.4,-0.1) -- (2.3,-0.1);
\draw (0.0, -0.1) -- (0.0, -0.2);
\draw (0.0, -0.3) node{$x+u(x)$};
\draw (2.0, -0.1) -- (2.0, -0.2);
\draw (2.0, -0.3) node{$x + \delta x + u(x+\delta x)$};


%% autres traits
\draw[dashed] (-0.8,1.0) -- (0,0.5);
\draw[dashed] (1.2,1.0) -- (2,0.5);

%% bilan des forces
\draw[->] (-0.5,0.3) -- (-0.05,0.3) node[below left]{$q(x)$};
\draw[->] (2.1,0.3) -- (3,0.3) node[below left]{$q(x+\delta x)$};
\draw[->] (0.4,0.3) node{$\times$} -- (0.6,0.3);
\draw[->] (0.9,0.3) node{$\times$} -- (1.1,0.3);
\draw[->] (1.4,0.3) node{$\times$} -- (1.6,0.3);
\draw (1,0.15) node{$f$};
\end{tikzpicture}
\caption{Un flux de particules. Gauche: repr\'esentation du probl\`eme.
  Droite: \'equilibre des flux.}
\label{fig:barre}
\end{figure}


La barre dans son \'etat initial est repr\'esent\'ee en pointill\'es.
Sous l'effet d'une force lin\'eique $f$, cette barre s'allonge
et atteint l'\'etat d'\'equilibre repr\'esent\'e en bas de la figure.
On note $u(x)$ le d\'eplacement de mati\`ere qui a eu lieu entre
l'\'etat sous la charge $f$ et l'\'etat de r\'ef\'erence en pointill\'es.


On consid\`ere que la barre est \`a l'\'equilibre m\'ecanique.
On note $q(x)$ la force qu'exerce la section de gauche sur la section de droite
en $x$. En faisant un bilan de force comme sur la partie droite de la figure \ref{fig:flux},
les forces s'exer\c{c}ant sur une portion infinit\'esimale de barre sont
la force $q(x)$ \`a gauche, la force $-q(x+\delta x)$ \`a droite
et la force lin\'eique $f(x) \delta x$. La barre \'etant \`a l'\'equilibre la somme de ces forces
est nulle. On retrouve donc \eqref{eq:eq_flux}.


De plus, la force s'exer\c{c}ant en $x$ \`a travers la section de la barre est 
proportionnelle \`a l'\'elongation de la barre et s'oppose au mouvement impos\'e.
Ceci est intuitif, pensez \`a un \'elastique si vous l'allongez il s'exerce une force
qui tend \`a le faire revenir vers sa position initiale. De plus, plus l'\'elongation
est importante, plus l'intensit\'e de la force est grande.
On obtient donc la loi d'\'elasticit\'e \eqref{eq:def_flux}
o\`u $k$ est un coefficient de raideur:
plus $k$ est grand, plus la barre est raide, plus il faut forcer pour la d\'eformer.
En pratique, le coefficient de raideur d\'epend du mat\'eriau choisi et
de la g\'eom\'etrie de la section de la barre.

Notons que dans \eqref{eq:def_flux}, la d\'eriv\'ee en espace correspond \`a 
l'\'elongation de la barre en $x$. Pour s'en convaincre, on regardera la 
partie droite de la figure \ref{fig:barre}. Un \'el\'ement de mati\`ere de longueur
$\delta x$ dans sa position de r\'ef\'erence a pour longueur 
$x + \delta x + u(x+\delta x) - u(x) - x$ sous charge $f$.
La nouvelle longueur est donc de $\delta x + u(x+\delta x) - u(x) 
\simeq \left(1 + \dfrac{\partial u}{\partial x} \right) \delta x$
et la d\'eriv\'ee partielle en $x$ est donc bien une \'elongation lin\'eique.


!! CONDITIONS FRONTIERES !!
La partie hachur\'ee \`a gauche du dessin repr\'esente le fait que la barre est encastr\'ee
dans un mur. Ainsi son d\'eplacement est forc\'ement nul \`a gauche.


!! autres exemples de cette physique : electromagnetisme !!

!! solution pour diff\'erents $k$ !!

!! EDP elliptique !! A faire en exo !! -- A exprimer sous la forme d'une proposition !!

!! dire clairement ce qui est exigible !!

!! dire ce que sont les equations de Poisson et de Laplace !!
%%%%%%%%%%%%%%%%%%%%%%%%%%%%%%%%%%%%%%%
\subsection{\'Equation de transport}

La deuxi\`eme \'equation que nous \'etudions est l'\'equation de transport
(unidimensionelle).
Elle correspond \`a une quantit\'e qui est transport\'ee \`a vitesse
constante dans une direction.
Cela peut \^etre par exemple un polluant transport\'e par une rivi\`ere.
Dans cette section, nous nous int\'eressons au cas de tas de sable
sur un tapis roulant se d\'epla\c{c}ant \`a vitesse constante $a$.


\begin{figure}
\centering
\begin{tikzpicture}[scale = 0.5]
  %% longueur du tapis roulant
\def\L{15};

%% tapis roulant
\draw (1,0) arc (0:360:1);
\draw (0,1.1) arc (90:270:1.1);
\draw (\L+1.0,0) arc (0:360:1);
\draw (\L,1.1) arc (90:-90:1.1);
\draw (0,1.1) -- (\L,1.1);
\draw (0,-1.1) -- (\L,-1.1);

%% tas de sables (pointilles)
\draw[dashed] (1,1.1) -- (5,3.5) -- (10,2.3) -- (15,5.5);

%% distance parcourue
\def\a{3}

%% tas de sables (trait continu)
\draw (1+\a,1.1) -- (5+\a,3.5) -- (10+\a,2.3) -- (15+\a,5.5);

%% vecteur a
\draw[->] (5,3.9) -- (5+\a,3.9);
\draw (5+\a/2,4.5) node{$a$};

%% hauteur u
\draw[<->] (14,1.2) -- (14,2.8);
\draw (15.3,2.0) node{$u(t,x)$};

%% axe des abscisses
\draw[->] (-1.5,0.0) -- (19.0,0.0);
\draw (19.0,-1.0) node{$x$};

\end{tikzpicture}
\caption{Des tas de sable transport\'es sur un tapis roulant.
  La ligne discontinue repr\'esente les tas en $t=0s$ et la ligne
  continue en $t=1s$.}
\label{fig:transport}
\end{figure}

Sur la figure \ref{fig:transport}, nous repr\'esentons des tas de sables
qui se d\'eplacent \`a vitesse constante $a$ vers la droite.
On note $u(t,x)$ la hauteur du sable en $x$ \`a l'instant $t$.

Si l'on consid\`ere un temps infinit\'esimal $\delta t$,
le sable aura avanc\'e d'une distance $\delta x = a \delta t$.
La hauteur en $(t+\delta t , x + \delta x)$ sera la m\^eme qu'elle
\'etait en $(t,x)$, on traduit cela par $u(t+\delta t, x + a \delta t) = u(t,x)$.
On obtient ainsi l'\'equation
\begin{align}
  \dfrac{\partial u}{\partial t} + a \dfrac{\partial u}{\partial x} = 0 .
\end{align}
Ici, $a$ correspond \`a la vitesse du transport.
Si $a$ est positif, le sable bouge vers la droite; si $a$ est n\'egatif,
le sable bouge vers la gauche; plus $a$ est grand en valeur absolue, 
plus le mouvement est rapide.



%%%%%%%%%%%%%%%%%%%%%%%%%%%%%%%%%%%%%%%
\subsection{\'Equation de la chaleur}

L'\'equation de la chaleur est obtenue en consid\'erant une propagation 
de quantit\'es d'\'energie comme sur la figure \ref{fig:flux}.
La diff\'erence avec ce qui a \'et\'e fait dans la section \ref{subsec:elliptique}
est qu'ici les flux de chaleur ne sont pas n\'ecessairement \`a l'\'equilibre.
On permet donc \`a la temp\'erature de changer au cours du temps.

Nous allons maintenant refaire le raisonnement de la section \ref{subsec:elliptique}
avec cette fois-ci une temp\'erature $u(t,x)$ qui d\'epend du temps.
Si on consid\`ere la partie droite de la figure \ref{fig:flux}, l'accumulation
d'\'energie en $(t,x)$ est due au fait que les flux ne sont pas \'equilibr\'es
("ce qui entre n'est pas \'egal \`a ce qui sort").
Nous traduisons cela par l'\'equation
$c (x) \delta x \dfrac{\partial u}{\partial t}(t,x) = f(t,x) \delta x 
  + q(t,x) - q(t,x + \delta x)$, ce qui donne
\begin{align}
  \label{eq:flux_t}
  c (x) \dfrac{\partial u}{\partial t}(t,x) = f(t,x) - \dfrac{\partial q}{\partial x} (t,x) .
\end{align}
o\`u $c(x)$ est la capacit\'e thermique lin\'eique du mat\'eriau,
$q(t,x)$ et $f(t,x)$ sont respectivement le flux de chaleur 
et la source de chaleur en $(t,x)$.
Comme pr\'ec\'edemment, le flux de chaleur est proportionnel au gradient 
de temp\'erature (voir \eqref{eq:def_flux}).
Pour simplifier la pr\'esentation, nous consid\'erons que la capacit\'e
thermique $c$ et la conduction thermique $k$ sont des constantes 
(qui ne d\'ependent ni du temps, ni de l'espace).
En combinant les \'equations \eqref{eq:def_flux} et \eqref{eq:flux_t},
on obtient l'\'equation de la chaleur
\begin{align}
  \dfrac{\partial u}{\partial t} - \nu \dfrac{\partial^2 u}{\partial x^2} = f ,
\end{align}
o\`u $\nu = k/c$ est la diffusion thermique.


!! courbes avec $\nu$ qui varie !!

!! exercice : prouver qu'il s'agit d'une equation parabolique !!

De mani\`ere similaire \`a ce qui a \'et\'e expliqu\'e dans la section \ref{subsec:elliptique},
cette l'\'equation de la chaleur, en plus de repr\'esenter l'\'evolution de la temp\'erature
dans un milieu peut aussi repr\'esenter l'\'evolution de la densit\'e d'une population
ou encore l'\'evolution d'une concentration chimique d'une esp\`ece au sein d'un diluant.


!! rappel sur les conditions fronti\`eres !!

%%%%%%%%%%%%%%%%%%%%%%%%%%%%%%%%%%%%%%%
\subsection{\'Equation des ondes}

L'\'equation des ondes est obtenue en consid\'erant un mod\`ele m\'ecanique
comme celui de la figure \ref{fig:barre} o\`u cette fois-ci les forces ne
sont pas n\'ecessairement \`a l'\'equilibre et o\`u les d\'eplacements
$u(t,x)$ peuvent varier au cours du temps.

La principe fondamental de la dynamique appliqu\'e \`a une tranche de mati\`ere
de longueur $\delta x$ nous dit que l'acc\'el\'eration de la position de cette tranche
($x+u(t,x)$) multipli\'ee par sa masse est \'egale \`a la somme des forces qui
agissent sur elle.
Ainsi, $\rho(x) \delta x \dfrac{\partial^2 u}{\partial t^2} = 
f(t,x) \delta x + q(t,x) - q(t, x + \delta x)$, ce qui nous donne l'\'equation
\begin{align}
  \label{eq:barre_t}
  \rho(x) \dfrac{\partial^2 u}{\partial t^2} + \dfrac{\partial q}{\partial x}(t,x) = f(t,x) ,
\end{align}
o\`u $\rho(x)$ correspond \`a la masse de la barre par unit\'e de longueur.
Plus $\rho$ est grand, plus la barre a d'inertie et plus elle acc\'el\`ere lentement.
Notons \`a ce stade qu'il y a, dans cette \'equation, une d\'eriv\'ee seconde en temps
\`a la place de la d\'eriv\'ee premi\`ere qu'il y avait dans \eqref{eq:flux_t}.

La force de compression \`a travers la barre est donn\'ee par la loi d'\'elasticit\'e
\eqref{eq:def_flux}.
Pour simplifier la pr\'esentation, on consid\`ere que la raideur $k$ et la masse
lin\'eique $\rho$ sont constantes (ne d\'ependent ni de $t$ ni de $x$).
En combinant \eqref{eq:def_flux} et \eqref{eq:barre_t} on obtient l'\'equation des ondes
\begin{align}
  \label{eq:ondes}
  \dfrac{\partial^2 u}{\partial t^2} - c^2 \dfrac{\partial^2 u}{\partial x^2} = f ,
\end{align}
o\`u $c>0$ d\'efini par $c^2 = k/\rho$ correspond \`a la vitesse de propagation
des ondes dans ce milieu.

!! differentes valeurs de $c$ -> utiliser des valeurs reelles (eau/air) !!

!! prouver que l'\'equation est hyperbolique !!

!! autres physiques : electromagnetisme, corde !!

!! rappel sur les conditions fronti\`eres !!


%%%%%%%%%%%%%%%%%%%%%%%%%%%%%%%%%%%%%%%%%%%%%%%%%%%%%%%%%%%%%%%%%%%%%%%%%%%%%%%%%%%%
\section{\'Equations elliptiques}

Comme \'evoqu\'e pr\'ec\'edemment, on s'int\'eresse au probl\`eme
\begin{equation}
  \label{eq:ell_Dir}
  \left\{
    \begin{array}{l}
      - \DIV (k \GRAD u) = f \dans \Omega ,
      \\
      u = 0 \sur \partial \Omega ,
    \end{array}
  \right.
\end{equation}
avec $0 < k_0 \leq k(x) \leq k_1$.

%%%%%%%%%%%%%%%%%%%%%%%%%%%%%%%%%%%%%%%
\subsection{Propri\'et\'es g\'en\'erales de l'\'equation}

\begin{theorem}[Existence et unicit\'e de la solution]
  \label{thm:ell_existence}
  Si $f \in C^0(\Omega)$, alors il existe une unique solution
  $u \in C^2(\Omega)$ au probl\`eme \eqref{eq:ell_Dir}.
  De plus, pour $\ell \in \N$, si $f \in C^{\ell}(\Omega)$, alors
  $u \in C^{\ell+2}(\Omega)$.
\end{theorem}

!! commenter + donner lien vers preuve !!

\begin{proposition}[Principe du maximum]
  On se place dans le cadre du th\'eor\`eme \ref{thm:ell_existence}.
  Si $f \leq 0$, alors $u \leq 0$. En particulier,
  $\max_{x\in \Omega} u(x) = \max_{x\in \partial \Omega} u(x) = 0$.
  Si de plus il existe $m\in \Omega$ tel que $u(m) = 0$,
  alors $\forall x \in \Omega$, $u(x) = 0$ 
  (principe du maximum fort).
\end{proposition}

!! commenter + ref livre !!

\begin{remark}
  D'autres conditions fronti\`ere sont possibles.
  Condition de Neumann, de Robin, conditions mixtes, conditions de Dirichlet non homog\`enes.
\end{remark}

!! expliquer le principe d'un rel\`evement !!

\begin{definition}[Fonction harmonique]
  Si $u \in C^2(\Omega)$ et $- \Delta u = 0$ dans $\Omega$,
  on dit que $u$ est une fonction harmonique.
\end{definition}

!! lien avec les fonctions holomorphes !!

%%%%%%%%%%%%%%%%%%%%%%%%%%%%%%%%%%%%%%%
\subsection{Formulation variationnelle, th\'eor\`eme de Lax--Milgram}

Le but de cette section est de pr\'esenter une preuve du th\'eor\`eme \ref{thm:ell_existence}
en utilisant le th\'eor\`eme de Lax--Milgram.

\begin{theorem}[Lax--Milgram]
  \label{thm:Lax-Milgram}
  On fait les hypoth\`eses suivantes.
  \begin{itemize}
  \item Soit $V$ un espace de Hilbert.
  \item Soit $\ell$ une forme lin\'eaire continue sur $V$
    (il existe $C_{\ell} > 0$ tel que $\forall v \in V, |\ell(v)| \leq C_{\ell} \| v \|_{V}$).
  \item Soit $a$ une forme bilin\'eaire continue sur $V$
    (il existe $C_{a} > 0$ tel que $\forall v,w \in V, |a(v,w)| \leq C_{a} \| v \|_{V} \| w \|_{V}$).
  \item On suppose de plus que $a$ est coercive:
    il existe $\alpha > 0$ tel que $\forall v \in V, a(v,v) \geq \alpha \| v \|_{V}^2$.
  \end{itemize}
  Sous ces hypoth\`eses le probl\`eme suivant est bien pos\'e:
  \begin{equation}
    \label{eq:Lax-Milgram}
    \text{Trouver $u \in V$, tel que $\forall v \in V$, } a(u,v) = \ell(v) .
  \end{equation}
  Ceci signifie que le probl\`eme \eqref{eq:Lax-Milgram} admet une unique solution $u \in V$
  et que celle-ci v\'erifie $\| u \|_{V} \leq \dfrac{C_{\ell}}{\alpha}$.
\end{theorem}

!! lien vers un livre !!

Pour utiliser ce th\'eor\`eme, \'ecrivons le probl\`eme \eqref{eq:ell_Dir} sous
sa forme variationnelle.
On introduit l'espace de Sobolev
\begin{align}
  H_0^1(\Omega) := \{ v \in L^2(\Omega) \st \GRAD v \in L^2(\Omega) 
  \text{ et } v_{| \partial \Omega} = 0 \sur \partial \Omega \} ,
\end{align}
o\`u $\GRAD v$ est d\'efini au sens des distributions et $v_{\partial \Omega}$
est la trace de $v$ sur le bord du domaine.


Si $u$ est une solution de \eqref{eq:ell_Dir}, alors pour tout $v \in H^1_0(\Omega)$
on peut \'ecrire
\begin{align*}
  \int_{\Omega} - \DIV (k \GRAD u) v \deriv x = \int_{\Omega} f v \deriv x .
\end{align*}
En int\'egrant par parties on obtient 
\begin{align*}
  \int_{\Omega} k \GRAD u : \GRAD v \deriv x 
  - \int_{\partial \Omega} k (\GRAD u \SCAL n) v \deriv s = \int_{\Omega} f v \deriv x .
\end{align*}
Puisque $v_{\partial \Omega} = 0$, le deuxi\`eme terme de cette expression est nul.
Nous avons donc \'etabli la proposition suivante 
(la r\'eciproque utilise des arguments similaires).
\begin{proposition}
  Le probl\`eme \eqref{eq:ell_Dir} est \'equivalent au probl\`eme suivant.
  \begin{equation}
    \label{eq:LM2}
    \text{Trouver $u \in H_0^1(\Omega)$, tel que $\forall v \in H_0^1(\Omega)$, } 
    \int_{\Omega} k \GRAD v : \GRAD v \deriv x = \int_{\Omega} f v \deriv x .
  \end{equation}
\end{proposition}

\begin{proposition}
  Le probl\`eme \eqref{eq:LM2} est bien pos\'e.
\end{proposition}
\begin{proof}
  Le probl\`eme \eqref{eq:LM2} correspond au probl\`eme \eqref{eq:Lax-Milgram}
  avec $V = H_0^1(\Omega)$, 
  $\forall v, w \in H_0^1(\Omega), a(v,w) = \int_{\Omega} k \GRAD v : \GRAD v \deriv x$
  et $\ell(v) = \int_{\Omega} f v \deriv x$.
  Nous allons montrer que toutes les hypoth\`eses du th\'eor\`eme de Lax--Milgram
  sont v\'erifi\'ees.
  On admet le fait que $H_0^1(\Omega)$ \'equip\'e de la norme 
  $\| v \|_{H_0^1(\Omega)} := (\int_{\Omega} v^2 + \GRAD v : \GRAD v \deriv x)^{1/2}$ 
  est un espace de Hilbert (se reporter \`a un cours sur les distributions).
  D'apr\`es l'in\'egalit\'e de Cauchy--Schwarz, on a 
  $|\ell(v)| \leq (\int_{\Omega} f^2 \deriv x)^{1/2} (\int_{\Omega} v^2 \deriv x)^{1/2}
  \leq (\int_{\Omega} f^2 \deriv x)^{1/2} \| v \|_{H_0^1(\Omega)}$.
  La forme lin\'eaire $\ell$ est donc continue. 
  De la m\^eme fa\c{c}on, l'in\'egalit\'e de Cauchy--Schwarz permet de prouver
  que la forme bilin\'eaire $a$ est continue.
  Pour finir, nous utilisons l'in\'egalit\'e de Poincar\'e: il existe $c>0$
  tel que $\forall v \in H_0^1(\Omega), 
  \int_{\Omega} v^2 \deriv x \leq c \int_{\Omega} \GRAD v : \GRAD v \deriv x$.
  Avec cette in\'egalit\'e, on peut prouver que $a$ est coercive.
  Toutes les hypoth\`eses du th\'eor\`eme de Lax--Milgram sont r\'eunies,
  le probl\`eme \eqref{eq:LM2} est donc bien pos\'e.
\end{proof}



%%%%%%%%%%%%%%%%%%%%%%%%%%%%%%%%%%%%%%%
\subsection{Discr\'etisation par la m\'ethode des diff\'erences finies}

Le but de la m\'ethode des diff\'erences finies est de 

%%%%%%%%%%%%%%%%%%%%%%%%%%%%%%%%%%%%%%%%%%%%%%%%%%%%%%%%%%%%%%%%%%%%%%%%%%%%%%%%%%%%
\section{\'Equation de transport}

%%%%%%%%%%%%%%%%%%%%%%%%%%%%%%%%%%%%%%%%%%%%%%%%%%%%%%%%%%%%%%%%%%%%%%%%%%%%%%%%%%%%
\section{\'Equation de la chaleur}

%%%%%%%%%%%%%%%%%%%%%%%%%%%%%%%%%%%%%%%%%%%%%%%%%%%%%%%%%%%%%%%%%%%%%%%%%%%%%%%%%%%%
\section{\'Equation des ondes}


%%%%%%%%%%%%%%%%%%%%%%%%%%%%%%%%%%%%%%%%%%%%%%%%%%%%%%%%%%%%%%%%%%%%%%%%%%%%%%%%%%%%
\section{Pour aller plus loin}


%===================================================================================

\end{document}
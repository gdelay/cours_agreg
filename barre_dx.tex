\def\h{0.1}
\def\N{11.0}

%% rectangle 1
\draw[dashed] (-0.8,1.0) -- (1.2,1.0) -- (1.2,1.5) -- (-0.8,1.5) -- (-0.8,1.0);

%% rectangle 2
\draw (0,0) -- (2,0) -- (2,0.5) -- (0,0.5) -- (0,0);

%% axe des abscisses 1
\draw[dashed,->] (-1.2,0.9) -- (1.8,0.9);
\draw[dashed] (-0.8, 0.9) -- (-0.8, 0.8);
\draw (-0.8, 0.7) node{$x$};
\draw[dashed] (1.2, 0.9) -- (1.2, 0.8);
\draw (1.2, 0.7) node{$x+\delta x$};

%% axe des abscisses 2
\draw[->] (-0.4,-0.1) -- (2.3,-0.1);
\draw (0.0, -0.1) -- (0.0, -0.2);
\draw (0.0, -0.3) node{$x+u(x)$};
\draw (2.0, -0.1) -- (2.0, -0.2);
\draw (2.0, -0.3) node{$x + \delta x + u(x+\delta x)$};


%% autres traits
\draw[dashed] (-0.8,1.0) -- (0,0.5);
\draw[dashed] (1.2,1.0) -- (2,0.5);

%% bilan des forces
\draw[->] (-0.5,0.3) -- (-0.05,0.3) node[below left]{$q(x)$};
\draw[->] (2.1,0.3) -- (3,0.3) node[below left]{$q(x+\delta x)$};
\draw[->] (0.4,0.3) node{$\times$} -- (0.6,0.3);
\draw[->] (0.9,0.3) node{$\times$} -- (1.1,0.3);
\draw[->] (1.4,0.3) node{$\times$} -- (1.6,0.3);
\draw (1,0.15) node{$f$};
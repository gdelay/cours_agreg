%% Fichier contenant des sources pour des parties non conservees sur l'equation des ondes
%% (discretisation par les differences finies)

\section{Equation des ondes}
\subsection{Analyse num\'erique}

!! INTRO DU SCHEMA EXPLICITE CENTRE !!

Plus g\'en\'eralement, les sch\'emas \`a deux pas sont de la forme
\begin{align}
  \label{eq:schema_2pas}
  U^0 = U_0 , \quad U^1 = U_0 + h_t V_0 \quad \text{et} 
  \quad \forall n \geq 1, \quad U^{n+1} = A_1 U^n + A_2 U^{n-1} + h_t^2 F^n ,
\end{align}
avec $A_1$ et $A_2$ deux matrices et $U_0$ et $V_0$ des vecteurs tels que (...)

La but de cette section est d'adapter les d\'efinitions des sections \ref{subsubsec:analyse_def}
et !!(...)!! au contexte des sch\'emas \`a deux pas de temps.
Nous adaptons tout d'abord les d\'efinitions et les th\'eor\`emes d'analyse num\'erique.
Nous analysons ensuite diff\'erents sch\'emas dont \eqref{eq:ondes_DF_exp}.
%%%%%%%%%%%%%%%%%
\subsubsection{D\'efinitions et th\'eor\`emes 
  pour les sch\'emas \`a deux pas de temps}

Nous donnons maintenant sans commentaire les d\'efinitions et th\'eor\`emes de la section
\ref{subsubsec:analyse_def} adapt\'es au cadre d'un sch\'ema \`a deux pas de temps.
Notons les similitudes avec les d\'efinitions pour un pas de temps.
Nous consid\'erons $\|\cdot\|$ une norme de $\R^M$ (cette norme d\'epend de $M$).

\begin{definition}[Stabilit\'e en norme $\|\cdot\|$]
  \label{def:stabilite_2pas}
  Soit $\Stab \subset \R_+^* \times \R_+^*$ tel que $(0,0) \in \overline{\Stab}$.
  On dit que le sch\'ema num\'erique \eqref{eq:schema_2pas} est stable 
  (en norme $\|\cdot\|$)
  sous la condition $\Stab$ s'il existe $C_1 , C_2 , C_3 > 0$ qui ne d\'ependent que de $T$
  tels que $\forall (h_x , h_t) \in \Stab$, $\forall U_0 , U^1 \in \R^{M}$, 
  $\forall (F^n)_{0 \leq n \leq N} \in \R^{(N+1) \times M}$, on a
  \begin{align}
    \max_{0\leq n \leq N} \| U^n \| 
    \leq C_1 \| U^0 \| + C_2 \| U^1 \| 
    + C_3 \max_{0\leq n \leq N} \| F^n \| ,
  \end{align}
  o\`u $(U^n)$ est l'unique suite v\'erifiant \eqref{eq:schema_2pas}.

  On dit \'egalement que le sch\'ema \eqref{eq:schema_2pas} 
  est inconditionnellement stable
  si la d\'efinition pr\'ec\'edente
  s'applique avec $\Stab = \R_+^* \times \R_+^*$.
\end{definition}

\begin{definition}[Erreur de troncature]
  \label{def:troncature_2pas}
  L'erreur de troncature (ou erreur de consistance) du sch\'ema \eqref{eq:schema_2pas}
  au temps $t_n$ est un vecteur $\varepsilon^n \in \R^{M}$
  ($2 \leq n \leq N$)
  d\'efini par
  \begin{align}
    \label{eq:troncature_2pas}
    \varepsilon^{n+1} := \tu^{n+1} - A_1 \tu^n - A_2 \tu^{n-1} - h_t^2 F^n ,
  \end{align}
  o\`u $\tu^n \in \R^{M}$ est d\'efini par
  $(\tu^n)_j = u(t_n,x_j)$ avec $u$ la solution du probl\`eme exact associ\'e \`a
  \eqref{eq:schema_2pas}.
\end{definition}

\begin{definition}[Consistance en norme $\|\cdot\|$]
  \label{def:consistance_2pas}
  On dit que le sch\'ema num\'erique \eqref{eq:schema_2pas} est consistant 
  (en norme $\|\cdot\|$) si pour toute
  solution r\'eguli\`ere $u$ du probl\`eme exact on a
  \begin{align}
    \lim_{\substack{N \to +\infty\\ M \to +\infty}} \max_{0\leq n \leq N} 
    \dfrac{\| \varepsilon^n \|}{h_t^2} = 0 ,
  \end{align}
  o\`u $\varepsilon^n$ est l'erreur de troncature (et $h_t = T/N$).

  On dit de plus que le sch\'ema \eqref{eq:schema_2pas} est consistant d'ordre $p \in \N^*$
  en temps et $q \in \N^*$ en espace (en norme $\|\cdot\|$)
  si pour toute solution r\'eguli\`ere $u$ (du probl\`eme exact)
  il existe une constante $C>0$ ind\'ependante de $h_t$ et $h_x$ telle que
  \begin{align}
    \forall N, M \geq 2 , \quad \max_{0\leq n \leq N}
    \dfrac{\| \varepsilon^n \|}{h_t^2} \leq C (h_t^p + h_x^q) .
  \end{align}
\end{definition}

\begin{remark}
  Notons que la constante $C$ dans la d\'efinition \ref{def:consistance}
  peut d\'ependre de la solution $u$ du probl\`eme exact.
\end{remark}

\begin{definition}[Convergence en norme $\| \cdot \|$]
  \label{def:convergence_2pas}
  On dit que le sch\'ema num\'erique \eqref{eq:schema_2pas} converge (ou est convergent) 
  en norme $\| \cdot \|$
  sous la condition $\Stab \subset \R_+^* \times \R_+^*$
  si 
  \begin{align}
    \lim_{\substack{(h_t,h_x) \in \Stab \\ (h_t,h_x) \to (0,0)}} \max_{0\leq n \leq N} 
    \| \tu^n - u^n \| = 0 ,
  \end{align}
  avec $Nh_t = T$ et $Mh_x = 1$ o\`u $u$ est la solution du probl\`eme exact
  et $(\tu^n)_j = u(t_n,x_j)$.
  
  De mani\`ere similaire, on dit que le sch\'ema num\'erique 
  \eqref{eq:schema_2pas} converge (ou est convergent) en norme $\|\cdot\|$
  \`a l'ordre $p\in\N^*$ en temps et $q\in\N^*$ en espace
  sous la condition $\Stab \subset \R_+^* \times \R_+^*$
  s'il existe une constante $C>0$ ind\'ependante de $h_t$ et $h_x$
  telle que
  \begin{align}
    \forall (h_t , h_x) \in \Stab , \quad \max_{0\leq n \leq N}
    \| \tu^n - u_j^n \| \leq C (h_t^p + h_x^q) .
  \end{align}

  On dit enfin que le sch\'ema est convergent (resp. convergent \`a l'ordre $p$
  en temps et $q$ en espace) en norme $\|\cdot\|$ 
  si la d\'efinition ci-dessus est v\'erifi\'ee
  avec $\Stab = \R_+^* \times \R_+^*$.
\end{definition}

Ces d\'efinitions sont reli\'ees par le th\'eor\`eme de Lax.
\begin{theorem}[Th\'eor\`eme de Lax]
  \label{thm:convergence_2pas}
  Si le sch\'ema \eqref{eq:schema_2pas} est stable en norme $\|\cdot\|$ sous la condition
  $\Stab \subset \R_+^* \times \R_+^*$ et consistant en norme $\|\cdot\|$
  (respectivement consistant d'ordre $p$ en temps et $q$ en espace),
  alors le sch\'ema \eqref{eq:schema_2pas} est convergent en norme $\|\cdot\|$
  (respectivement convergent d'ordre $p$ en temps et $q$ en espace)
  sous la condition $\Stab$.


  De plus, si le probl\`eme exact (le probl\`eme v\'erifi\'e par $u$) 
  est bien pos\'e, alors la consistance et la stabilit\'e du sch\'ema 
  sont n\'ecessaires \`a sa convergence.
\end{theorem}
Le th\'eor\`eme \ref{thm:convergence_2pas} s'applique aussi avec $\Stab = \R_+^* \times \R_+^*$.


\begin{remark}
  On peut imaginer des sch\'emas avec un nombre de pas de temps sup\'erieur \`a deux.
  Ces d\'efinitions peuvent alors \^etre adapt\'ees \`a ce cas de figure.
\end{remark}


!! FAIRE ATTENTION A CE QUE CETTE SECTION SOIT ADAPTEE POUR L'ETUDE DES SCHEMAS
MULTI-PAS DANS LE CAS D'UN PB D'ORDRE UN EN TEMPS 
(en particulier faut-il un $h_t^2$ devant $F^n$ et dans la consistance) !!


!! Remarque : on pourrait construire des caract\'erisations de la stabilit\'e comme
nous l'avons fait en section 3 !!


!! adapter ce qui vient d'\^etre fait \`a l'analyse des sch\'emas \`a venir !!
---- Enlever toute cette partie ... 
(en la reportant dans la partie \'equation de la chaleur)
(faire quand m\^eme une remarque quelque part sur ce qu'est la convergence)

%%%%%%%%%%%%%%%%%
\subsubsection{Analyse de quelques sch\'emas}



Une m\'ethode classique pour \'etudier des sch\'emas \`a plusieurs pas de temps est
de les r\'e\'ecriture comme un sch\'ema "vectoriel" \`a un pas de temps.
Pour cela, introduisons les inconnues 
\begin{align}
  v_j^n = \dfrac{u_j^n - u_j^{n-1}}{h_t} , \qquad w_j^n = c \dfrac{u_j^n - u_{j-1}^n}{h_x} .
\end{align}
Nous pouvons montrer que ces inconnues v\'erifient
\begin{align}
  \dfrac{v_j^{n+1} - v_j^n}{h_t} - c \dfrac{w_{j+1}^n - w_j^n}{h_x} = f(t_n,x_j) ,
  \\
  \dfrac{w_j^{n+1} - w_j^n}{h_t} = c \dfrac{v_j^{n+1} - v_{j-1}^{n+1}}{h_x} .
\end{align}
Nous pouvons donc r\'e\'ecrire le sch\'ema \eqref{eq:ondes_DF_exp}
comme le sch\'ema \`a un pas de temps de la forme \eqref{eq:schema_1pas}
(nous ne donnons pas ici l'expression de la matrice et du second membre).

Il s'agit d'un sch\'ema explicite o\`u \`a chaque pas de temps,
$V^{n+1}$ et $W^{n+1}$ sont calcul\'es comme
\begin{align*}
  &v_j^{n+1} = v_j^n + \dfrac{c h_t}{h_x} ( w_{j+1}^n - w_j^n ) + h_t f_j^n ,
  \\
  &w_j^{n+1} = w_j^n + \dfrac{c h_t}{h_x} ( v_j^n - v_{j-1}^n ) 
  + \left( \dfrac{c h_t}{h_x} \right)^2 (w_{j+1}^n - 2 w_j^n + w_{j-1}^n ) 
  + \dfrac{c h_t^2}{h_x} (f_j^n - f_{j-1}^n ) . 
\end{align*}

!! Donner l'initialisation de ce sch\'ema !!


Nous voulons \'etudier la stabilit\'e de ce sch\'ema au sens de Von Neumann.
On consid\`ere donc que $V^n$ et $W^n$ sont donn\'es par des ondes spatiales et
on calcule le facteur d'amplification qui appara\^it au pas de temps suivant.
Si pour $k \in \Z$, $v_j^n = w_j^n = e^{2 i \pi k x_j}$ et $f_j^n = 0$, alors
$v_j^{n+1} = v_j^n + \dfrac{c h_t}{h_x} ( e^{2 i \pi k h_x} - 1 ) w_j$
et $w_j^{n+1} = w_j^n + \dfrac{c h_t}{h_x} (1-e^{-2 i \pi k h_x}) v_j^n
+ 2 \left(\dfrac{c h_t}{h_x}\right)^2 (\cos(2 \pi k h_x) - 1) w_j^n$.
Ainsi, nous avons la relation
\begin{align*}
  \begin{pmatrix}
    v_j^{n+1} \\ w_j^{n+1}
  \end{pmatrix}
  = A(k)
  \begin{pmatrix}
    v_j^{n} \\ w_j^{n}
  \end{pmatrix} ,
\end{align*}
o\`u $A(k)$ est la matrice d'amplification associ\'ee \`a ce probl\`eme.
Elle est donn\'ee par
\begin{align*}
  A(k) = 
  \begin{pmatrix}
    1 & i e^{i \pi k h_x} a(k)
    \\
    i e^{-i \pi k h_x} a(k) & 1 - (a(k))^2
  \end{pmatrix} ,
\end{align*}
avec $a(k) = 2 \frac{c h_t}{h_x} \sin(\pi k h_x)$.


Nous admettons le fait qu'une condition n\'ecessaire de stabilit\'e soit
que les valeurs propres de cette matrice aient un module $\leq 1$.
Le discriminant de cette matrice vaut
$\Delta(k) = (a(k))^2 [(a(k))^2 - 4]$.
\'Etant donn\'e que le d\'eterminant de cette matrice vaut $1$,
le produit de ses deux valeurs propres vaut aussi $1$.
Il est donc n\'ecessaire que le discrimant reste $\leq 0$ pour tout $k \in \Z$.
Nous avons donc la condition n\'ecessaire de stabilit\'e
\begin{align*}
  \frac{c h_t}{h_x} \leq 1 .
\end{align*}


Cette condition ne correspond pas \`a une condition suffisante de stabilit\'e.
On peut prouver qu'une condition suffisante est
\begin{align*}
  \exists \delta \in ]0,1[, \qquad
  \text{$\delta$ ind\'ependant de $h_t$ et $h_x$ tel que  }
  \dfrac{c h_t}{h_x} \leq \delta .
\end{align*}
Pour plus de d\'etails, le lecteur peut se reporter au livre de B. Lucquin
(p. 196--201).


!! difficult\'e : dans le livre de B. Lucquin, il faut r\'e\'ecrire les 
sch\'emas comme des sch\'emas d'ordre 1 !!
-- Utilise-t-on les def au dessus ??
!! Il y aurait moyen de les utiliser mais il faudrait refaire toutes les 
caract\'erisations !!

!! Une tr\`es bonne fa\c{c}on de faire serait de faire les premi\`eres d\'emos avec
les nouvelles defs et de laisser en exercice un raisonnement sur le syst\`eme vectoriel
---- On pourrait ainsi inclure la partie "continue" dans l'exo !!


!! Une autre fa\c{c}on de proc\'eder serait de faire toute l'analyse num\'erique comme dans
le livre puis de faire une section "liens avec l'\'equation de transport"
o\`u on ferait l'\'etude continue avec la m\'ethode des caract\'eristiques !!
---- On va choisir cette option : l'analyse avec plusieurs pas de temps est tr\`es compliqu\'ee !!



!! toute cette analyse a l'air compliqu\'ee : une possibilit\'e serait 
de juste \'evoquer les sch\'emas sans les analyser !!

!! analyse sch\'ema intro !!

!! interpretation CFL !!

!! v\'erifier qu'il n'y a pas besoin de stabilit\'e dans d'autres normes
que celle propos\'ee !!


!! sch\'ema implicite !!
+ $\theta$-sch\'ema en exo


!! exo final sur formulation mixte !!
+ derniere question : legere modif rend le sch\'ema instable !!

!! exercice avec conditions de Dirichlet non homogenes ??




%%%%%%%%%%%%%%%%%%
\subsubsection{D\'etermination pratique des ordres de convergence}


!! expliquer comment ca marche + figures + exo : retrouver les figures !!


!! exercice sur les ordre de convergence : figure avec echelle logarithmique !!
